%! TEX root = ../tor.tex

\chapter{Funcționarea, pe scurt}

\section{Relee și \qq{foi de ceapă}}
\indent\indent Într-o formă simplificată, Tor funcționează prin transmiterea conexiunii 
printr-o serie de \emph{relee} (eng.\ \texttt{relay}) de la computerul inițiator pînă la
destinație.

Actualmente, există peste 6000 de relee în toată lumea, care se ocupă cu această
redirecționare a traficului. Releele sînt localizate în întreaga lume și puse la
dispoziție de voluntari.

Într-o conexiune standard, Tor realizează conexiunea cu 3 relee, fiecare dintre
acestea avînd cîte un rol standard.
\begin{figure}[!htbp]
  \centering
  \includegraphics[scale=0.5]{fig/3relays.png}
  \caption{Cele 3 relee standard (\cite{jw1})}
  \label{fig:3rel}
\end{figure}

\begin{itemize}
  \item \textit{Releul de intrare} (eng.\ \texttt{entry/guard}), prin care conexiunea
    intră în rețeaua Tor. Asemenea relee sînt alese după ce au dovedit o vechime
    în rețea, stabilitate și lățime de bandă corespunzătoare.
  \item \emph{Relee intermediare} sînt cele care transmit conexiunea mai departe.
    Totodată, în ideea anonimizării, rolul lor este ca releele de intrare și cele
    de ieșire să nu se cunoască între ele.
  \item \emph{Releul de ieșire}, care se află la capătul rețelei Tor și trimit 
    traficul către destinația finală dorită de client.
\end{itemize}
\index{releu}
\index{releu!de intrare}
\index{releu!intermediar}
\index{releu!de ieșire}

De remarcat este faptul că, dacă releele intermediare pot fi orice calculator, server
etc., care nu se compromit în niciun fel, deoarece ele nu fac decît să transporte
trafic deja criptat, releele de ieșire au o responsabilitate deosebită. În cazul unei
conexiuni ilicite, traficul către destinație apare ca fiind transmis de la releul
de ieșire, ceea ce-i expune în mod deosebit.

La fiecare pas se realizează o decriptare, dacă traficul circulă de la client către
server și o criptare, dacă traficul circulă invers. Așa cum am menționat în secțiunea
anterioară, fiecare nod intermediar adaugă sau elimină cîte un strat criptografic,
realizînd \qq{rutarea în foi de ceapă}.

\begin{figure}[!htbp]
  \centering
  \includegraphics{fig/4onions.png}
  \caption{\qq{Foile de ceapă} (\cite{bs})}
  \label{fig:4on}
\end{figure}


Prin acest mecanism, fiecare releu cunoaște doar minimul necesar: nodul anterior și
nodul care va urma, realizînd \emph{criptarea telescopică} despre care am mai vorbit.
De remarcat este faptul că releul de ieșire vede datele inițiale trimise de client.
Astfel, dacă se trimit date sensibile prin protocoale care folosesc text clar, precum
HTTP sau FTP, releul de ieșire poate să intercepteze traficul.

%%%%%%%%%%%%%%%%%%%%%%%%%%%%%%%%%%%%%%%%%%%%%%%%%%%%%%%%%%%%%%%%%%%%%%

\section{Poduri} \index{poduri}

\indent\indent Utilizarea releelor, în forma descrisă mai sus, ridică o vulnerabilitate
serioasă. Astfel, atunci cînd un client se conectează la rețea, trebuie să aibă acces
la lista tuturor releelor de intrare, mediane și de ieșire, pentru a ști unde s-ar putea
conecta. Lista releelor nu este secretă, ceea ce ridică o potențială problemă de securitate.

O soluție pentru această problemă este utilizarea \emph{podurilor} (eng.\ \texttt{bridges}).
Într-o formă simplificată, putem privi podurile ca intrări secrete în relee, pe care le
pot accesa, de exemplu, utilizatori care se află în spatele unor rețele cenzurate.

Există și o listă completă de poduri, deținută de proiectul Tor, dar această listă
nu este publicată. Cei de la Tor au găsit o soluție prin care utilizatorii să aibă
acces doar la o porțiune mică a listei podurilor, suficiente pentru a iniția o conexiune.
Astfel, utilizatorul nici nu are nevoie de toate podurile disponibile, ci doar de cîteva
pentru a-și face intrarea în rețea.

Cercetătorii au reușit să identifice între 79\% și 86\% din lista totală a podurilor,
realizînd o analiză a întregului spațiu de adrese IPv4, dar chiar și așa, putem considera
că secretul listei podurilor este suficient de bine păstrat.

%%%%%%%%%%%%%%%%%%%%%%%%%%%%%%%%%%%%%%%%%%%%%%%%%%%%%%%%%%%%%%%%%%%%%%
\section{Autorități directoare} \index{autorități directoare}

\indent\indent Am menționat în prima parte faptul că în rețea există o serie
de autorități directoare, care sînt nodurile cele mai de încredere și, într-un fel,
țin funcționare proiectului în spate.

Statusul releelor Tor este ținut într-un document dinamic, numit \emph{consens}. \index{consens}
Acest document este întreținut de autoritățile directoare (AD) și actualizat
în fiecare oră prin voturi, în următorul mod:
\begin{itemize}
  \item fiecare AD face o listă de relee cunoscute;
  \item fiecare AD calculează și ceilalți parametri necesari despre relee (țara,
    lățimea de bandă etc.);
  \item AD transmite această informație sub forma unui status celorlalte AD;
  \item fiecare AD primește acest status, din care își actualizează propria
    listă;
  \item toți parametrii adunați de la toate AD sînt combinate și se calculează
    un vot, care este transmis cu o semnătură de către fiecare AD;
  \item releele care primesc majoritatea voturilor sînt păstrate în consens, care
    se actualizează și se transmite tuturor AD.
\end{itemize}

Procesul de vot și actualizare de mai sus este public, transmis prin HTTP, astfel
încît poate fi accesat de orice utilizator.
