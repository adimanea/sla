%! TEX root = ../tor.tex

\chapter{Generalități}

\section{Idei principale}

\indent\indent \qq{Rutarea în foi de ceapă} (eng.\ \textit{onion routing}, OR) a fost
concepută pentru anonimizarea aplicațiilor care funcționează pe baza protocolului
TCP, precum browsere, servicii de mesagerie sau \texttt{ssh}. În acest model
de rutare, clienții aleg un drum prin rețea, construind un circuit, dar
astfel încît fiecare nod cunoaște doar predecesorul și succesorul său, fără restul
nodurilor care ar putea exista în rețea.
\index{onion!routing (OR)}

Traficul propriu-zis criculă în pachete de mărime fixată, numite \emph{celule},
\index{celule} care sînt decodate cu o cheie simetrică în fiecare nod, apoi
transmise mai departe.

După înființarea proiectului, a apărut destul de repede o rețea de astfel de
noduri, dar chiar și la început, atunci cînd singurul nod era doar o mașină de teste,
procesa conexiuni de la peste 60 000 IP-uri din întreaga lume, cu un flux de
aproximativ 50 000 conexiuni pe zi.

Tor a fost conceput pentru a rezolva deficiențele găsite în modelul inițial OR,
iar principalele îmbunătățiri privesc:
\begin{enumerate}[(1)]
  \item \textit{Secretizarea perfectă a înaintării traficului} (eng.\ perfect
    forward secrecy): în modelul inițial, dacă exista un nod ostil într-un punct
    al conexiunii, putea să refuze înaintarea traficului și să decripteze
    conținutul care a ajuns pînă la el. În modelul actual, se folosește o
    \emph{criptare telescopică}, prin care nodul curent adaugă chei
    de sesiune la criptarea de pînă atunci și totodată pierde accesul la cheile din
    urmă, astfel încît mesajul să nu mai poată fi decriptat (nodul curent are
    acces doar la ultima parte a cheii, cu care contribuie).
    \index{criptare!telescopică}
  \item \textit{Separarea \qq{curățirii protocolului} de anonimitate:} în varianta
    inițială, OR cerea proxy-uri pentru aplicații, corespunzătoare fiecărei
    aplicații suportate. Dacă proxy-urile nu erau scrise (fiind un proces laborios),
    protocoalele suportau doar puține aplicații. Tor folosește o interfață
    proxy standard, SOCKS și proxy-uri de filtrare pentru aplicații de tip
    Privoxy, care sînt bine cunoscute și dezvoltate.
    \todo[inline,noline,backgroundcolor=green!40]{citări pentru SOCKS și Privoxy}
  \item \textit{Nu influențează traficul}: OR cerea gruparea și reordonarea
    celulelor care ajungeau la norduri și adăugarea de padding între utilizatori
    și OR-uri. \todo[inline,noline,backgroundcolor=green!40]{padding?}
    Această prelucrare suplimentară cerea standardizarea și găsirea unor algoritmi
    care să facă operațiunile sigure --- algoritmi care încă nu s-au găsit. Pînă
    la găsirea unor implementări satisfăcătoare, Tor nu \qq{modelează} în niciun
    fel traficul.
  \item \textit{Mai multe fire TCP pot folosi același circuit}: în implementarea
    originală, OR cerea construirea cîte unui circuit separat pentru
    fiecare cerere. Dar aceasta ridica probleme de securitate, deoarece
    în cazul unui volum foarte mare de trafic, cheile generate pentru
    anonimizare și criptare puteau să se repete. Tor permite utilizarea
    simultană a aceluiași circuit de către mai multe fire TCP.
  \item \textit{Implementarea unei topologii de tip \qq{țeavă cu scurgeri}}:
    clientul poate forța celulele să iasă din circuit pe oriunde, nu doar
    pe la capete, în cazul în care observă vreo problemă sau nu dorește
    să continue rutarea.
    \index{topologie!țeavă cu scurgeri}
  \item \textit{Controlul blocajelor:} Tor implementează un control și
    management al blocajelor și a\-glo\-me\-ră\-ri\-lor de tip decentralizat,
    prin intermediul celor care participă în rețea (devenind noduri).
  \item \textit{Servere directoare:} În forma inițială, OR răspîndea
    informația prin rețea fără o anume organizare. Tor stabilește
    existența unor servere directoare, care sînt noduri de încredere
    și conțin semnături care asigură integritatea rețelei. Clienții
    descarcă periodic aceste semnături prin HTTP, pentru a se asigura
    de integritatea rețelei și a traficului.
    \index{server!director}
  \item \textit{Politici standardizate:} pentru intrarea și ieșirea
    traficului în și din fiecare nod --- un element-cheie în cazul
    rețelelor construite din voluntari.
  \item \textit{Puncte de întîlnire și servicii ascunse:} în forma
    originală, în OR existau noduri speciale folosite pentru a construi
    circuite speciale către servere ascunse. Dar utilitatea lor era nulă dacă
    vreunul se defecta. În varianta Tor, clienții negociază \qq{puncte de %
    întîlnire} (eng.\ rendezvous-points) pe parcurs, care ghidează
    traficul pe drum și nu se bazează doar pe \qq{autorități speciale}.
    \index{punct!de întîlnire (rendezvous)}
\end{enumerate}

Ca implementare, Tor nu necesită instalarea de module speciale în kernel.
Dezavantajul este că nu se pot anonimiza decît protocoale TCP, dar avantajul
principal este dat de portabilitate.

De asemenea, o altă caracteristică de implementare este aceea că Tor se 
încadrează in categoria design-urilor cu latență scăzută, deoarece este
folosit pentru anonimizarea traficului în timp real.


%%%%%%%%%%%%%%%%%%%%%%%%%%%%%%%%%%%%%%%%%%%%%%%%%%%%%%%%%%%%%%%%%%%%%%

\section{Obiective de design și asumpții}

Obiectivul principal este împiediarea atacurilor care ar viza un singur
utilizator, ca țintă unică. Totodată, următoarele aspecte esențiale au
contribuit la dezvoltarea Tor în forma actuală:
\begin{itemize}
  \item Ușurința lansării în aplicații reale (eng.\ \emph{deployability}): este
    necesar ca implementarea și utilizarea să fie cît mai simple și
    economice, iar cerința de anonimitate este imperioasă. Singurele
    noduri care nu satisfac această cerință de anonimitate sînt punctele
    de întîlnire, pentru asigurarea integrității traficului.
  \item Ușurința de utilizare: simplitatea contribuie la securitate, iar
    anonimatul nu trebuie să fie dependent de sistemul de operare.
  \item Flexibilitate: protocolul trebuie să fie suficient de flexibil și
    în același timp suficient de bine specificat astfel încît să poată
    servi drept tester pentru cercetări în domeniu.
\end{itemize}

De asemenea, s-au trasat și \emph{non-obiective}, aspecte care sînt
în mod intenționat ignorate pe parcursul dezvoltării serviciului:
\begin{itemize}
  \item Nu servește drept conexiune peer-to-peer: abordarea are probleme
    serioase de securitate, chiar dacă a fost implementată de alte servicii.
    \todo[inline,noline,backgroundcolor=green!40]{probleme la p2p --- exemple}
  \item Nu este sigur împotriva atacurilor end-to-end: Tor nu pretinde
    să rezolve această problemă, împotriva atacurilor \emph{end-to-end timing}
    sau \emph{intersection attacks}. Problema nu este rezolvată satisfăcător
    în domeniu, dar ar putea să ajute ca utilizatorii să-și ruleze propriile
    OR.
    \index{atac!end-to-end}
    \index{atac!intersection}
    \todo[inline,noline,backgroundcolor=green!40]{detalii despre aceste atacuri!}
  \item Nu oferă normalizarea protocoalelor: dacă se dorește securizarea
    informației transmise folosind protocoale complexe precum HTTP sau UDP,
    trebuie să se folosească servicii suplimentare; Tor nu normalizează securitatea.
  \item Nu este steganografic --- nu ascunde cine este conectat în rețea.
    \todo[inline,noline,backgroundcolor=green!40]{interesant cuvînt... detalii?}
\end{itemize}

%%%%%%%%%%%%%%%%%%%%%%%%%%%%%%%%%%%%%%%%%%%%%%%%%%%%%%%%%%%%%%%%%%%%%%

\section{Modelul general al amenințărilor}

\indent\indent Design-urile de anonimizare au drept adversar tipic
unul \emph{global, pasiv}, adică unul care poate asculta tot traficul.
\index{adversar!global} \index{adversar!pasiv}

Ca orice alt proiect pentru sisteme cu latență scăzută, Tor nu
poate proteja împotriva unui asemenea adversar. Vom analiza însă, cazul
adversarului care poate observa o porțiune a traficului, care poate genera,
modifica, șterge sau întîrzia traficul, care ar putea să ruleze propriile OR
și care ar putea compromite o parte a OR existente.

În sistemele de anonimizare cu latență scăzută care folosesc criptarea
pe straturi (eng.\ \emph{layered encryption}), \index{criptare!pe straturi}
obiectivul tipic al adversarului este să observe inițiatorul și respondentul
traficului. Astfel, prin observare, ar putea confirma că Alice vorbește cu Bob
dacă în trafic sînt prezente anumite variabile de flux și temporizare.
Un atacator activ poate introduce asemenea variabile pentru a se asigura.

Tor urmărește să împiedice nu atacurile de confirmare a traficului, ci pe
cele de \emph{învățare}, pentru ca adversarul să nu poată folosi informații
preluate pasiv pentru a afla structura rețelei sau a routerelor. De exemplu,
învățînd structura rețelei, ar putea afla poziția nodurilor de încredere (e.g.\
serverele directoare sau punctele de întîlnire)
și ar putea destabiliza întreaga rețea atacîndu-le pe acelea.
\index{atac!de confirmare a traficului}
\index{atac!de învățare a traficului}
