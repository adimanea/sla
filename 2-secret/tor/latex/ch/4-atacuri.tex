%! TEX root = ../tor.tex

\chapter{Atacuri și strategii de apărare}

\indent\indent În lucrarea inițială de prezentare a tehnologiei Tor (\cite{whitepaper}),
apar diverse tipuri de atacuri pe care cercetătorii le prevăzuseră, împreună cu
modurile de apărare.

\section{Atacuri pasive}

\indent\indent Un atac pasiv clasic este reprezentat de \emph{observarea tiparelor %
din traficul utilizatorilor}. Astfel, deși identitatea utilizatorilor rămîne
necunoscută, se poate profila traficul și găsi diverse tipare în modul lor
de utilizare. Totuși, sarcina este îngreunată de faptul că pot circula mai
multe fire prin același circuit, așa cum am menționat în prima parte a lucrării,
deci conexiunile interceptate pot să nu aparțină utilizatorului-țintă.

Un alt tip de atac tipic este dat de \emph{observarea conținutului utilizatorului}.
Deși la nivelul utilizatorului, conținutul este criptat, conexiunile către
serverele care trimit răspunsul, pot să nu fie (de exemplu, conexiunea de
la releul final la serverul adresat). Se poate chiar ca serverul accesat să
fie ostil și să încerce să atace utilizatorul. Tor nu urmărește să filtreze
conținutul, dar se poate apăra împotriva acestui atac folosind servicii terțe
precum Privoxy sau altele similare pentru filtrare.

Faptul că este permis fiecărui utilizator să-și configureze conexiunea poate
fi o vulnerabilitate. De exemplu, utilizatorii mai avansați sau cei care
ar avea nevoie de anonimitate sau intimitate mai pronunțată, pot cere schimbarea
circuitului foarte des. Dar aceasta poate fi o vulnerabilitate pentru utilizatorii
\qq{normali}, deoarece îi evidențiază ca pe o posibilă minoritate.

Așa cum am mai menționat, un alt atac clasic poate fi \emph{corelația la ambele capete}.
Un observator global ar putea să identifice și să profileze mai bine traficul,
dar în realitate, acest atac este aproape imposibil. Mai mult decît atît,
cum Tor folosește topologia țevii cu scurgeri, se poate întîmpla ca un utilizator
să ceară ieșirea din rețea înainte de capăt, iar atunci observatorul, chiar cu
puteri globale, este compromis.

\index{atac!de confirmare}
\index{atac!website fingerprinting}
\index{atac!de corelație}
Atacurile de mai sus se încadrează în categoria \emph{atacurilor de confirmare}.
Prin observarea traficului, se poate confirma că s-a realizat o anumită
conexiune de interes. Dar există și un alt atac cu potențial mai periculos,
de care Tor nu poate apăra foarte eficient: \emph{amprentarea site-urilor}
(eng.\ \texttt{website fingerprinting}). În locul căutării conexiunilor de
intrare sau ieșire pentru profilare, atacatorul poate să-și facă o bază
de date de \qq{amprente} care conțin tipare de acces și mărimi de fișiere
schimbate de client cu site-uri țintă. În cazul Tor, eficiența acestui
atac nu este foarte mare, deoarece se folosesc mai multe fire în format
multiplex prin aceeași rețea și mai mult, amprentele sînt limitate de
mărimea celulelor de trafic, adică de 512 octeți. Sugestii pentru îmbunătățirea
securității din acest punct de vedere ar fi să se utilizeze scheme de grupare
a traficului către anumite site-uri sau padding aplicat link-urilor.


%%%%%%%%%%%%%%%%%%%%%%%%%%%%%%%%%%%%%%%%%%%%%%%%%%%%%%%%%%%%%%%%%%%%%%
\section{Atacuri active}

\indent\indent Un atacator care află cheia de sesiune TLS ar putea
să controleze celulele și releele din circuit. Totodată, aflarea
acestei chei îi permite să desfacă un strat din criptare și, dacă
află cheia unui OR, poate să \qq{devină} acel OR pe durata vieții
cheii. Însă atacul nu este foarte util, deoarece, pentru a putea crea
într-adevăr impresia că este acel OR, trebuie să aibă acces și la cheia
onion pentru a decripta celulele \texttt{create}. Dar, cum conexiunea
are securitate perfectă la înaintare, de îndată ce s-a făcut o legătură
în rețea, ea nu mai poate fi compromisă. Totodată, rotația cheilor
limitează perioada în care asemenea atacuri pot acționa. Astfel că,
în realitate, un asemenea atac poate fi eficient doar dacă se poate
lansa pe durata vieții unei conexiuni. Situația este puțin probabilă,
dar pot exista cazuri legale sau extralegale cînd autorități de un
anume rang pot cere accesul la anumite noduri. A fost cazul în 2003,
cînd autoritățile germane au obținut un mandat de spargere a protocoalelor
serviciului de anonimizare \texttt{Java Anon Proxy} (\cite{jap}).

În plus, dacă atacatorul află cheia de identitate a unui nod, poate
contacta serverele directoare și să trimită descrieri false, infiltrîndu-se
în aceste servere.
\index{atac!compromiterea cheilor}

Un alt tip de atac activ care poate fi lansat este prin rularea unui server.
Dacă un atacator creează un server web care să ofere conținut prin care
să atragă utilizatori ai rețelei Tor, atunci el are șanse să intre în rețea
și să fie accesat de noduri de ieșire. În acest caz, poate afla diverse
tipare de trafic sau poate trimite conținut malițios, deoarece are control
asupra unui capăt al conexiunii. Similar este și cazul cînd un atacator
deține un OR ostil. Dar, dacă atacatorul deține controlul asupra $ m > 1 $
din $ N $ noduri, se poate arăta că el poate corela cel mult $ \Big(\dfrac{m}{n}\Big)^2 $
din trafic, conform \cite{bs}. Acest număr, deși poate fi neglijabil 
(amintim că $ N > 6000 $), poate fi grupat cu alte atacuri, precum acela 
al rulării unui server care să atragă trafic.

În ce privește atacurile asupra celulelor, amintim că la fiecare nod se
fac verificări de integritate, deci acestea nu pot fi atacate cu succes.

În cazul în care utilizatorul vrea să se conecteze la un server printr-un
protocol nesecurizat, precum HTTP, se poate ca un atacator să se dea drept
serverul-țintă. De aceea, utilizatorii se pot proteja de o asemenea situație
conectîndu-se cu protocoale care fac autentificarea la ambele capete.

Desigur, alte tipuri de atacuri care țin de faptul că rețeaua este menținută
activă de voluntari pot fi distribuirea de cod ostil din partea unui utilizator
de încredere sau acțiuni interzise din punctul de vedere al politicilor
de ieșire sau de rutare ori din punct de vedere social.

%%%%%%%%%%%%%%%%%%%%%%%%%%%%%%%%%%%%%%%%%%%%%%%%%%%%%%%%%%%%%%%%%%%%%%

\section{Atacuri asupra directoarelor}

\index{atac!pe directoare}
\indent\indent În cazul în care se distrug servere directoare, rețeaua
este proiectată pentru a continua să funcționeze, deoarece încă se
primesc semnături și voturi în consens din partea serverelor încă
active. Însă, din design, dacă se distrug peste jumătate din serverele
directoare, este necesară intervenția umană, deoarece consensul nu mai
are suficiente date.

Atacul asupra unei singure autorități directoare, însă, nu este tocmai
eficient, deoarece consensul înregistrează voturi, iar acest atac ar
putea, în cel mai rău caz (dar total nepractic) să influențeze aceste
voturi, înclinînd balanța către autorizarea unor servere ostile.
În practică, asemenea atacuri nu au apărut, deci nu există precedențe
pentru a se face analize de impact.

Și în aceste cazuri, pot apărea atacuri datorate naturii umane. De exemplu,
se poate păcăli o autoritate directoare că un OR funcționează, deși este
stricat sau că este valid, deși este malițios sau convingerea pentru listarea unui
server drept autoritate directoare. Tor nu apără împotriva unor asemenea
atacuri.

%%%%%%%%%%%%%%%%%%%%%%%%%%%%%%%%%%%%%%%%%%%%%%%%%%%%%%%%%%%%%%%%%%%%%%

\section{Cenzură și atacuri politice}
\index{atac!cenzură}
\indent\indent Un atac special, care poate fi lansat de un atacator
ostil sau comandat politic (sub formă de cenzură, de exemplu), ar putea
să vizeze traficul care intră sau care iese din Tor, adică să împiedice
utilizatorii să intre în rețea sau să iasă către destinație.

Blocarea nodurilor de ieșire este o situație care poate avea loc, deoarece
lista nodurilor de ieșire este publică, iar Tor nu se poate apăra de
asemenea atacuri. Un atacator poate pur și simplu să dezactiveze nodul
de ieșire sau să-l inunde cu cereri, lucru care l-ar face inutilizabil.

În celălalt caz, însă, blocarea nodurilor de intrare este imposibilă,
mulțumită podurilor. Aceasta deoarece lista tuturor nodurilor de intrare 
nu este disponibilă, iar lista podurilor poate fi descoperită doar parțial.
Se poate, cel mult, combina acest tip de atac cu cel de corelarea traficului.
Dacă atacatorul are date acumulate despre anumiți utilizatori, despre care
se crede că se conectează la anumite noduri de intrare (sau acest lucru este
surprins în timp ce se întîmplă), se pot lansa atacuri asupra acelui nod.
Atacurile de tip \qq{inundație} (eng.\ \texttt{flood}) sînt cele mai ușoare
și ar putea dezactiva nodul de intrare.

