\chapter*{Introducere}

\indent\indent Proiectul prezentat este \texttt{Monocypher} (\cite{gh1}), care este o bibliotecă
criptografică scrisă de Loup Vaillant, începînd cu 2016. După cum menționează
autorul pe pagina \cite{loupcrypto}, intenția a fost de a concepe o bibliotecă
de criptare suficient de sigură, de rapidă, dar și de simplă pentru a fi folosită
atît de el, cît și de alții. Autorul a considerat că bibliotecile actuale ori nu
sînt suficient de sigure, ori sînt supraîncărcate pentru necesitățile sale,
așa că a pornit în a-și concepe propria soluție. Proiectul este scris în C.

Verificarea proiectului \texttt{Monocypher} se va face folosind pachetul
\texttt{FramaC} (\cite{framac}), specific pentru limbajul C. Cum \texttt{FramaC}
este, de fapt, mai curînd un mediu în care se pot rula mai multe unelte de verificare
formală, am ales verificarea \texttt{Monocypher} cu \texttt{Eva}.

Detaliile despre modul de funcționare a proiectului \texttt{Monocypher}, cît și
a verificării prin \texttt{Eva} sînt date în secțiunile următoare.

\vspace{1cm}

Toate exemplele și capturile proprii au fost realizate pe o mașină Core i7 8th gen și:
\begin{itemize}
\item \texttt{frama-c Chlorine 20180502}, instalat via \texttt{opam};
\item \texttt{Eva 18 Argon};
\item sistemul de operare Manjaro 18 (Illyria), Community Edition i3 (\cite{mi3});
\item editorul text Emacs 26.2 \cite{emacs} cu \texttt{acsl-mode} pentru adnotări ACSL (\cite{acsl-mode});
\item emulatorul de terminal \texttt{st} (\cite{st}), cu modificările lui Luke Smith
  (\cite{lukest}).
\end{itemize}

\todo[inline,noline,backgroundcolor=green!40]{bucăți de cod și comenzi centrate, pe linie separată!}

%%% Local Variables:
%%% mode: latex
%%% TeX-master: "../mono-frama"
%%% End:
