\chapter{Uneltele de verificare}

\section{\texttt{FramaC}}

\indent\indent \texttt{FramaC} este o unealtă de verificare formală dezvoltată
de un grup de cercetători de la Commisariat \'a l'\'Energie Atomique (CEA-list)
și INRIA din Franța.

În esență, \texttt{FramaC} verifică programe scrise în C, dar este, de fapt,
un cadru în care se pot include și utiliza unelte de verificare cu scopuri precise.
În varianta cea mai simplă, \texttt{FramaC} folosește un dialect al limbajului C
(de fapt, o variantă simplificată numită \emph{C Intermediate Language}) și oferă
metode de verificare similare cu uneltele bazate pe \emph{adnotarea codului},
precum Dafny și altele. Limbajul se numește \texttt{ACSL} și este descris detaliat
la \cite{acsl}.

\texttt{FramaC} însuși a fost implementat în limbajul OCaml, astfel că poate
fi instalat atît individual, cît și prin managerul de pachete \texttt{opam}.

Odată instalat, \texttt{FramaC} oferă două moduri de interacțiune: la nivel
de linie de comandă (CLI) sau prin interfață grafică (GUI). Este de menționat
faptul că verificatorul nu oferă un editor, astfel că modalitatea recomandată
de utilizare este:
\begin{enumerate}[(1)]
\item se scrie programul C cu editorul preferat;
\item se rulează din linia de comandă pe fișierele-sursă
  de verificat;
\item fie se redirecționează rezultatul într-un fișier separat, de exemplu
  \texttt{frama-c *.c >> log}, fie se salvează într-un format binar, specific,
  cu comanda \texttt{frama-c *.c -save mysession.sav};
\item se consultă fișierul \texttt{log}, dacă s-a folosit prima variantă,
  care este un fișier text simplu, sau, preferabil, se încarcă formatul
  salvat specific în interfața grafică, cu comanda \texttt{frama-c-gui -load mysession.sav}.
\end{enumerate}

\todo[inline,noline,backgroundcolor=green!40]{POZE CLI}

Dacă se alege varianta simplă, se poate observa că informațiile din \texttt{log}
nu sînt foarte explicite mereu, astfel că trebuie să știm să interpretăm rezultatele.
În general, ele sînt afișate sub forma \texttt{[nivel] fișier rezultat}, unde:
\begin{itemize}
\item \texttt{[nivel]} este nivelul la care se face verificarea (e.g.\ nivelul de bază, adică
direct codul sursă, dacă s-au folosit adnotări și nu se apelează vreun plugin;
\item \texttt{fișier} este fișierul verificat, despre care se raportează rezultatele;
\item \texttt{rezultat} este diagnosticul sau concluzia la care a ajuns analizatorul,
  la nivelul \texttt{[nivel]} asupra fișierului \texttt{fișier}.
\end{itemize}

\todo[inline,noline,backgroundcolor=green!40]{POZĂ log}

În varianta cu interfață grafică, însă, putem vedea mult mai multe informații, într-o
variantă mult mai flexibilă:
\begin{itemize}
\item În panoul din stînga se pot vedea fișierele încărcate spre analizare, care au fost
  \qq{desfăcute} în funcțiile conținute, care pot fi analizate individual;
\item Panoul median arată modul în care \texttt{FramaC} a interpretat codul-sursă,
  eventual scriindu-și singur adnotări;
\item Panoul din dreapta arată codul-sursă încărcat (în care \emph{nu} se poate scrie);
\item Panoul din stînga-jos arată uneltele folosite pentru verificare, eventual
  plugin-urile încărcate;
\item Panoul de jos arată mesajele și erorile generate, cu cîmpurile populate în
  funcție de uneltele de verificare folosite.
\end{itemize}

Deși poate fi utilizat în această formă, este recomandabil ca \texttt{FramaC} să fie
apelat folosind plugin-uri specifice. În verificarea proiectului \texttt{Monocypher},
vom folosi plugin-ul \texttt{Eva} (varianta actualizată [\emph{Evolved}] a
\emph{Value Analysis}), care este descris în secțiunea următoare.


%%%%%%%%%%%%%%%%%%%%%%%%%%%%%%%%%%%%%%%%%%%%%%%%%%%%%%%%%%%%%%%%%%%%%%

\section{\texttt{Eva}}

\subsection{Motivație}
\indent\indent Ca majoritatea uneltelor criptografice, \texttt{Monocypher}
se bazează pe expresii și operații numerice, eventual cu valori foarte
mari și prelucrări pe biți. Aceasta îl face să fie posibil vulnerabil
atît la operații matematice interzise (precum împărțirea la 0 sau depășirea
intervalului de valori permise de tipul de date folosit) sau la scurgeri de memorie.

\texttt{Eva} se bazează pe o teorie matematică numită \emph{interpretare abstractă}
(\cite{wikiabs}), conform și \cite{eva}, care îl face în special potrivit
pentru verificarea programelor cu conținut aritmetic bogat. De asemenea,
\texttt{Eva} poate detecta și scurgeri de memorie sau operații nepermise la nivel
de memorie alocată, dar pentru verificări mai avansate în special asupra memoriei,
este recomandabil să se folosească \texttt{Valgrind} (\cite{valgrind}).

%%% Local Variables:
%%% mode: latex
%%% TeX-master: "../mono-frama"
%%% End:
