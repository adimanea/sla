\documentclass[12pt, a4paper]{article}

\usepackage{libertine}
\usepackage{libertinust1math}
\usepackage{inconsolata}
\usepackage{amsthm, amsmath}
\usepackage[style=german]{csquotes}
\usepackage[romanian]{babel}
\usepackage{enumerate}
\usepackage{xcolor}
\usepackage{graphicx}
\usepackage[all]{xy}
\usepackage{hyperref}
\hypersetup{citecolor=blue}
\usepackage{ebproof} % for proof trees
\usepackage{natbib} % for citekey hyphenation

\newcommand\NN{\mathbb{N}}
\newcommand\ZZ{\mathbb{Z}}
\newcommand\QQ{\mathbb{Q}}
\newcommand\RR{\mathbb{R}}
\newcommand\CC{\mathbb{C}}
\newcommand\FF{\mathbb{F}}
\newcommand\bb{\mathbb}
\newcommand\dr{\mathrm}
\newcommand\kal{\mathscr}
\newcommand\fk{\mathfrak}
\newcommand\op{\oplus}
\newcommand\ot{\otimes}
\newcommand\ds{\displaystyle}
\newcommand\id{\indent}
\newcommand\nid{\noindent}
\newcommand\seq{\subseteq}
\newcommand\sneq{\subsetneq}
\newcommand\speq{\supseteq}
\newcommand\spneq{\supsetneq}
\newcommand\rar{\rightarrow}
\newcommand\Rar{\Rightarrow}
\newcommand\xrar{\xrightarrow}
\newcommand\lar{\leftarrow}
\newcommand\Lar{\Leftarrow}
\newcommand\xlar{\xleftarrow}
\newcommand\lrar{\leftrightarrow}
\newcommand\Lrar{\Leftrightarrow}
\newcommand\ideal{\trianglelefteq}
\newcommand\such{\text{ s.t. }}
\newcommand\sk{(\textit{Sketch}) }
\newcommand\rhup{\rightharpoonup}
\newcommand\rhdn{\rightharpoondown}
\newcommand\lhup{\leftharpoonup}
\newcommand\lhdn{\leftharpoondown}
\newcommand\incl{\hookrightarrow}
\newcommand\vid{\emptyset}
\newcommand\wt{\widetilde}
\newcommand\wh{\widehat}
\newcommand\ol{\overline}
\newcommand\Lor{\Longrightarrow}
\newcommand\Lol{\Longleftarrow}
\newcommand\Lolr{\Longleftrightarrow}
\newcommand\ld{{}^\ast}
\newcommand\lgr{{}^{\ast-\text{gr}}}
\newcommand\kap{\Bbbk^a[\kal{P}]}
\newcommand\kcp{\Bbbk^c[\kal{P}]}
\newcommand\fc{\mathfrak{c}}
\newcommand\qq{\enquote}
\newcommand\coli{\protect\varinjlim}
\newcommand\li{\protect\varprojlim}
\newcommand\injr{\rightarrowtail}
\newcommand\surjr{\twoheadrightarrow}
\newcommand\injl{\leftarrowtail}
\newcommand\surjl{\twoheadleftarrow}
\newcommand\sqsb{\sqsubseteq}
% hfill in math mode (e.g. align)
\makeatletter
\newcommand{\pushright}[1]{\ifmeasuring@#1\else\omit\hfill$\displaystyle#1$\fi\ignorespaces}
\newcommand{\pushleft}[1]{\ifmeasuring@#1\else\omit$\displaystyle#1$\hfill\fi\ignorespaces}
\makeatother
% FONTS
\usepackage[mathscr]{euscript}
\usepackage[normalem]{ulem}
\usepackage{eurosym}
\usepackage{sectsty}
\allsectionsfont{\rmfamily}
\usepackage{anyfontsize}

% PAGE SETUP
\usepackage[portrait]{geometry}
\geometry{
  a4paper,
  left=20mm,
  right=25mm,
  % marginparwidth=40mm,
}
\usepackage{afterpage}
\newcommand\blankpage{
  \null
  \thispagestyle{empty}
  \addtocounter{page}{-1}
  \newpage}
\newenvironment{myenv}[1]
  {\begin{spacing}{#1}}
    {\end{spacing}}

% FANCY SETUP
\usepackage[Glenn]{fncychap}
\usepackage{fancyhdr}
\addto\captionsromanian{\renewcommand{\chaptername}{Seminar}}

% THEOREMS
\newtheoremstyle{dotless}{}{}{\itshape}{}{\bfseries}{:}{ }{}
\theoremstyle{dotless}
\newtheorem{theorem}{Teorem\u{a}}[chapter]
\newtheorem{proposition}{Propozi\c{t}ie}[chapter]
\newtheorem{lemma}{Lem\u{a}}[chapter]
\newtheorem{corollary}{Corola}[chapter]
\newtheorem{axiom}{Axiom\u{a}}[chapter]
\newtheorem{principle}{Principiu}[chapter]
\newtheorem{reg}{Regul\u{a}}[chapter]
\renewcommand*{\proofname}{Dem.:}

\newtheoremstyle{dotlessdr}{}{}{}{}{\bfseries}{:}{ }{}
\theoremstyle{dotlessdr}
\newtheorem{example}{Exemplu}[chapter]
\newtheorem{definition}{Defini\c{t}ie}[chapter]
\newtheorem{remark}{Observa\c{t}ie}[chapter]
\newtheorem{exercise}{Exerci\c{t}iu}[chapter]
\numberwithin{equation}{chapter}

\begin{document}

\begin{center}
  {{\Large\textbf{Introducere în semantica denotațională}} \\
    --- \textit{raport de cercetare, anul 1, semestrul II} --- \\
    Adrian Manea \\
    coordonator: Conf.\ dr.\ Denisa Diaconescu
  }
\end{center}

\vspace{2cm}

\tableofcontents

\section{Introducere}
\indent\indent Lucrarea de față continuă studiul corespondenței Curry--Howard--Lambek,
expusă pe scurt într-o prezentare de ansamblu în semestrul anterior. De data aceasta,
ne concentrăm asupra noțiunilor corespunzătoare \emph{semanticii denotaționale a %
  limbajelor de programare} și introducem cîteva dintre uneltele matematice utile
pentru a studia \emph{teoria domeniilor}. Vom insista pe mulțimile parțial ordonate
complete și stricte, cărora le vom justifica utilitatea în specificații recursive
în topologia Scott (cf.\ \cite{scottstrachey}) și pe care le vom lega și de categorii
cartezian închise.

În particular, prezentarea de față pune accentul pe partea computațională a
co\-res\-pon\-den\-ței Curry--Howard--Lambek, prin semantica denotațională, dar totodată
evidențiază și formalismul teoriei categoriilor pentru aceste scopuri, formalism
care, conform \cite{lambek}, reprezintă un mediu excelent pentru interpretarea
sistemelor logice deductive.

%%%%%%%%%%%%%%%%%%%%%%%%%%%%%%%%%%%%%%%%%%%%%%%%%%%%%%%%%%%%%%%%%%%%%%

\section{Semantica punctelor fixe}

\subsection{Motivație}

\indent\indent Conform \cite{schm}, \emph{semantica denotațională} își propune să pornească de la
spe\-ci\-fi\-ca\-ții sintactice ale limbajelor, în varianta simplă folosindu-se
forma Backus-Naur (BNF) și să asocieze operații (în general, funcții),
care să arate înțelesul expresiilor din sintaxă. Totodată, este de dorit
uneori ca semantica denotațională să țină cont și de aspecte \emph{operaționale},
adică cele privitoare la implementarea limbajului și a compilatoarelor.

Deoarece accentul cade pe funcții, dar și ținîndu-se cont de aspectele
operaționale, semantica denotațională are un \emph{caracter compozițional}. În acest
sens, devine extrem de importantă specificarea matematică a așa-numitelor
\emph{algebre semantice}, care constau în domeniile de definiție și de valori
ale funcțiilor implementate și funcțiile propriu-zise (de arități diferite, deci
în particular includem și operații binare, unare și funcții de mai multe variabile).

Dana Scott (\cite{scottstrachey}) a arătat, într-o serie de articole publicate în
anii '70-'80 că, pentru majoritatea necesităților \qq{practice}, i.e.\ a limbajelor deja
existente, implementate sau teoretice, sînt necesare constrîngeri speciale asupra
naturii mulțimilor ce devin domenii semantice, dar și asupra funcțiilor dintre ele.
Astfel, dacă structuri, instrucțiuni și operații aritmetice simple se pot defini
fără constrîngeri speciale asupra mulțimilor și funcțiilor, în afara proprietăților
de tip morfism (i.e.\ compunerea funcțiilor să fie compatibilă cu operațiile),
recursivitatea omniprezentă este cea care ridică probleme.

În particular, \emph{combinatorii din $\lambda$-calcul} de forma $ \lambda x . xx $ sînt
cei care ridică probleme, prin auto-aplicare.

În continuare, vom prelua din expunerile \cite{abrj} și \cite{schm} pentru a justifica
și exemplifica semantica funcțiilor recursive și a domeniilor necesare pentru a le
susține.

%%%%%%%%%%%%%%%%%%%%%%%%%%%%%%%%%%%%%%%%%%%%%%%%%%%%%%%%%%%%%%%%%%%%%%

\subsection{Ecuația de punct fix}

\indent\indent Presupunem că avem o definiție recursivă, de forma:
\[
  X = \dots X \dots,
\]
căreia vrem să-i asociem un sens (semantică), i.e.\ o metodă de interpretare, care
să fie compatibilă cu restul limbajului din care ar face parte această expresie.
În particular, vom fi interesați de proprietățile compoziționale, deci semantica
va trebui să fie dată funcțional.

Presupunem că ecuația este formulată într-un domeniu $ D $, deci căutăm
un element $ d \in D $ astfel încît, făcînd o substituție convenabilă
în $ X $ cu $ d $, să obținem o expresie cu sens. Membrul drept al unei
expresii recursive are o formă funcțională, de tipul $ f(d) $, iar membrul
stîng trebuie să fie un rezultat, pentru ca expresia să returneze ceva,
deci sîntem conduși la o \emph{ecuație de punct fix}, de forma generală:
\[
  f(d) = d, \quad d \in D,
\]
unde urmează să definim funcția $ f $ și proprietățile domeniului $ D $.

În plus, pentru generalitatea teoriei, avem nevoie de o metodă canonică
sau măcar generală de a găsi puncte fixe pentru funcții (cu anumite proprietăți),
definite între domenii de un anumit fel.

Ajungem astfel la ecuația fundamentală rezolvată de Scott în semantica denotațională,
anume:
\[
  D \simeq [ D \to D ].
\]
Această ecuație afirmă că sîntem în căutarea unei corespondențe biunivoce
între elementele \qq{spațiului de funcții} ($ [D \to D] $) și elementele
lui $ D $, care pot fi rezultate ale definițiilor recursive pentru aceste
funcții.

Ecuația a fost formulată și rezolvată de Scott în cazul unor domenii
înzestrate cu o topologie discretă, numită \emph{topologia Scott}, iar
funcțiile din $ [D \to D] $ au o proprietate de continuitate în
raport cu această topologie.

Discuția detaliată asupra topologiei Scott depășește cadrul acestei
expuneri, dar vom folosi cele de mai sus ca motivație pentru studiul
unor proprietăți ale domeniilor semantice, dintre care cea mai importantă
este \emph{ordonarea parțială}. De fapt, structura de poset a domeniului $ D $
este punctul de pornire pentru definiția generală a topologiei Scott, deci
putem considera cele ce urmează drept o introducere elementară în acest subiect.

\begin{remark}\label{rk:dom-cat}
  Deoarece tema aleasă urmărește și legătura cu teoria categoriilor, facem
  observația că ecuația domeniilor Scott,
  \[
    D \simeq [D \to D]
  \]
  poate fi formulată și în context categorial. Fie $ \kal{C} $ o categorie.
  Spațiul de funcții $ [D \to D] $ poate fi gîndit ca un (bi)functor
  $ F : \kal{C}^{\text{op}} \times \kal{C} \to \kal{C} $.

  Atunci, ecuația Scott ajunge la găsirea acelor categorii $ \kal{C} $ și obiecte
  $ D \in \kal{C} $ astfel încît $ D \simeq F(D, D) $.

  Vom detalia acest punct de vedere categorial într-o oarecare măsură în
  \S\ref{sec:ccc}.
\end{remark}

%%%%%%%%%%%%%%%%%%%%%%%%%%%%%%%%%%%%%%%%%%%%%%%%%%%%%%%%%%%%%%%%%%%%%%

\subsection{Exemplu: Semantica factorialului}

\indent\indent Necesitatea folosirii domeniilor înzestrate cu o ordine parțială
poate fi văzută printr-un exemplu, preluat din \cite{schm}, \S 6.

Fie \texttt{Nat} domeniul numerelor naturale,
$\texttt{Nat}_\perp = \texttt{Nat} \cup \{ \bot \} $ și funcția definită prin:
\begin{align*}
  \code{fac n} = \lambda \code{n}.&\code{n equals zero} \to \code{one} \\
               &\square \code{ n times (fac n minus one)}
\end{align*}

Am folosit sintaxa preluată (și adaptată) din \cite{schm}, iar notațiile
în litere pentru constante și funcții elementare arată faptul că ele pot fi
definite separat, în așa-numitele domenii semantice primitive, cum este cazul
\code{Nat}. Detalii se găsesc, de exemplu, în \cite{schm}, \S 3. În continuare,
însă, vom face abuz de notație și vom folosi simbolurile obișnuite.

Evaluînd funcția \code{fac} pe argumente concrete, observăm că avem 3 cazuri posibile:
\begin{itemize}
\item nicio explicitare (eng.\ \emph{unfolding}) --- caz imposibil, pe care îl notăm
  cu $ \code{fac}_0 $ și care are graficul $ \emptyset $ (formal, este
  $ \Gamma \code{fac}_0 = \{ (\emptyset, \emptyset) \} $);
\item o explicitare, pentru $ n = 0 $, caz pe care îl notăm cu $ \code{fac}_1 $, care are
  graficul $ \Gamma \code{fac}_1 = \{ (0, 1) \} $;
\item $ i + 1 $ explicitări, pentru $ i $ apeluri recursive, caz pe care îl notăm cu
  $ \code{fac}_i $ și care, pentru $ i $ fixat, are graficul
  $ \Gamma \code{fac}_i = \{ (0, 1), (1, 1), \dots, (i, i!) \} $.
\end{itemize}

Aceste grafice sînt ordonate parțial cu incluziunea (admițînd că $ \{ (\emptyset, \emptyset) \} $
este submulțime pentru orice mulțime de perechi. În mod evident, ele modelează cazuri
particulare ale funcției \code{fac}, deci:
\[
  \Gamma \code{fac} \supseteq \bigcup_{i \geq 0} \Gamma \code{fac}_i.
\]
Dar are loc și incluziunea reciprocă: dacă o pereche $ (a, b) \in \Gamma \code{fac} $, ea
s-a obținut printr-un anumit număr de explicitări, deci există $ i $ cu $ (a, b) \in \Gamma \code{fac}_i $.
Rezultă că avem egalitate:
\[
  \Gamma \code{fac} = \bigcup_{i \geq 0} \Gamma \code{fac}_i.
\]

Aceste rezultat stă la baza semanticii pentru puncte fixe, care afirmă exact că
înțelesul total al unei funcții recursive este dat de cazurile sale particulare.
De aceea, putem extrage și o reprezentare nerecursivă pentru fiecare $\code{fac}_i $:
\begin{align*}
  \code{fac}_i &: \code{Nat} \to \code{Nat}_\bot, \quad i \geq 0 \\
  \code{fac}_0 &= \lambda \code{n} . \bot \\
  \code{fac}_{i + 1} &= \lambda \code{n.n = 0} \to 1\,\, \square \code{ n} \cdot \code{fac}_i(n-1), \quad i \geq 1,
\end{align*}
ceea ce ne conduce chiar la funcționala:
\begin{align*}
  \code{F} &: (\code{Nat} \to \code{Nat}_\bot) \to (\code{Nat} \to \code{Nat}_\bot) \\
  \code{F} &= \lambda \code{f}\lambda \code{n.n = 0} \to 1 \,\, \square \code{ n} \cdot \code{f(n-1)},
\end{align*}
cu $ \code{fac}_{i + 1} = \code{F fac}_i $.

Rezultă:
\[
  \Gamma \code{fac} = \ds\bigcup_{i \geq 0} \Gamma \code{F}^i \emptyset,
\]
unde $ \emptyset = \lambda \code{n.} \bot $ și mai mult,
\[
  \Gamma \code{F fac} = \Gamma \code{fac},
\]
deci $ \code{F fac} = \code{fac} $, de unde rezultă că \code{fac} este un punct
fix pentru \code{F}.

Dar, dacă pornim cu funcția:
\begin{align*}
  \code{q} &: \code{Nat} \to \code{Nat}_\bot \\
  \code{q} &= \lambda \code{n.n=0} \to 1 \,\, \square \code{ q(n+1)},
\end{align*}
se poate arăta (\cite{schm}, pp. 97-98) că ecuația $ \code{Qq = q} $ are
o infinitate de soluții, unde \code{Q} este funcționala corespunzătoare:
\begin{align*}
  \code{Q} &: (\code{Nat} \to \code{Nat}_\bot) \to (\code{Nat} \to \code{Nat}_\bot) \\
  \code{Q} &= \lambda \code{g} \lambda \code{n.n=0} \to 1\,\, \square \code{ g(n+1)}.
\end{align*}

Avem nevoie, așadar, de \emph{cel mai mic punct fix}, un motiv suplimentar de a lucra
cu structuri de poset, atît pe $ D $, cît și pe $ [ D \to D ] $. Mai facem observația
că pe $ [D \to D] $ vom lucra cu \emph{definiții extensionale}, adică:
\[
  f \geq g \Leftrightarrow \Gamma f \seq \Gamma g.
\]

%%%%%%%%%%%%%%%%%%%%%%%%%%%%%%%%%%%%%%%%%%%%%%%%%%%%%%%%%%%%%%%%%%%%%%

\subsection{Domenii parțial ordonate}

\indent\indent Fără a intra în detalii, rezultatele esențiale care arată utilitatea
poset-urilor pentru semantica definițiilor recursive urmează.

\begin{definition}\label{def:cpo}
  Fie $ (D, \sqsubseteq) $ o mulțime parțial ordonată.

  (1) O submulțime $ X \seq D $ se numește \emph{lanț} dacă $ X \neq \emptyset $ și
  pentru orice $ a, b \in X $, $ a \sqsubseteq a $ sau $ b \sqsubseteq a $.

  (2) Poset-ul $ (D, \sqsubseteq) $ se numește \emph{complet} (eng.\ \emph{cpo}) dacă
  orice lanț în $ D $ are cea mai mică margine superioară în $ D $.

  (3) $ D $ se numește \emph{punctat} sau \emph{strict} (eng.\ \emph{pcpo/scpo}) dacă
  este complet și $ D $ însuși are un cel mai mic element.

  (4) Fie $ A, B $ două cpo. O funcție $ f : A \to B $ se numește \emph{monotonă}
  (crescătoare) dacă respectă ordinea, i.e.\ $ a_1 \sqsubseteq a_2 $ în $ A $ implică
  $ f(a_1) \sqsubseteq f(a_2) $ în $ B $, relațiile de ordine parțială fiind cele
  ale domeniilor respective.

  (5) O funcție monotonă $ f : A \to B $ se numește \emph{continuă} dacă pentru orice
  lanț $ X \seq A $ are loc:
  \[
    f \Big( \bigsqcup X \Big) = \bigsqcup \{ f(x) \mid x \in X \},
  \]
  unde $ \sqcup = \sup $.
\end{definition}

\begin{remark}\label{rk:top}
  (1) Adesea, este foarte ușor de transformat un cpo într-unul strict, prin reuniune
  cu un cel mai mic element $ \{ \bot \} $.

  (2) Noțiunile de mai sus pot fi definite și mai riguros, într-o topologie, numită
  \emph{topologia Scott} (detalii în \cite{abrj}).
\end{remark}

Rezultatul fundamental care justifică utilitatea celor de mai sus în semantica definițiilor
recursive este:

\begin{theorem}\label{thm:fix}
  Fie $ D $ un pcpo și $ F : D \to D $ o funcție continuă. Atunci $ F $ are un cel mai
  mic punct fix:
  \[
    \code{fix} F = \bigsqcup \{ F^i \bot \mid i \geq 0 \}.
  \]
\end{theorem}

\begin{proof}
  Calculăm succesiv:
  \begin{align*}
    F \code{fix} F &= F \Big( \bigsqcup \{ F^i \bot \mid i \geq 0 \} \Big) \\
                   &= \bigsqcup \Big\{ F(F^i \bot) \mid i \geq 0 \Big\} \quad \text{(continutate)} \\
                   &= \bigsqcup \Big\{ F^i \bot \mid i \geq 1 \Big\} \\
                   &= \bigsqcup \Big\{ F^i \bot \mid i \geq 0 \Big\} \\
                   &= \code{fix} F.
  \end{align*}
  Penultimul pas este justificat de $ F^0 \bot = \bot \sqsubseteq F \bot $, din definiția celui
  mai mic element, $ \bot $.

  Fie acum $ e $ un alt punct fix. Rezultă:
  \[
    \bot \sqsubseteq e \Rightarrow F^i \bot \sqsubseteq F^i e = e, \forall i \Rightarrow %
    \sqcup \big\{ F^i \bot \mid i \geq 0 \big\} \sqsubseteq e \Rightarrow \code{fix} F \sqsubseteq e.
  \]
\end{proof}

Putem acum formula:

\begin{definition}\label{def:semfix}
  În condițiile și cu notațiile de mai sus, semantica specificației recursive $ f = Ff $
  este $ \code{fix} F $, cel mai mic punct fix al lui $ F $.
\end{definition}

O observație importantă pentru semantica unor limbaje de programare simple este:
se poate arăta că, dacă $ D_1, D_2 $ sînt pcpo, atunci și domeniile compuse $ D_1 + D_2 $,
$ D_1 \times D_2 $, $ [D_1 \to D_2] $ sînt pcpo, iar constructorii sînt operații continue.

%%%%%%%%%%%%%%%%%%%%%%%%%%%%%%%%%%%%%%%%%%%%%%%%%%%%%%%%%%%%%%%%%%%%%%

\section{Categorii cartezian închise}
\label{sec:ccc}

\indent\indent Prezentarea de mai jos urmează \cite{bw} și arată cum domeniile semantice
introduse mai sus pot fi interpretate în contextul categoriilor cartezian
închise.

Începem cu o observație simplă: mulțimea cpo stricte și cu funcții continue
și stricte (i.e. $ f\bot = \bot $) formează o categorie, care este o subcategorie
a cazului nestrict.

Vom vedea cum cazul nestrict ne permite, de fapt, să lucrăm cu categorii
cartezian închise. Totodată, prin aceasta vom putea lega teoria domeniilor
de tip cpo de semantica $ \lambda $-calculului cu tipuri simple, asociat
canonic unei categorii cartezian închise.

\begin{definition}\label{def:ccc}
  Fie $ \kal{C} $ o categorie. Ea se numește \emph{cartezian închisă} (CCC)
  dacă:
  \begin{enumerate}[(CCC1)]
  \item are un obiect terminal $ 1 $;
  \item orice pereche de obiecte $ A, B \in \kal{C} $ are un produs direct
    $ A \times B $, împreună cu proiecțiile canonice:
    \[
      A \xleftarrow{p_1} A \times B \xrightarrow{p_2} B;
    \]
  \item pentru orice obiecte $ A, B \in \kal{C} $, există un obiect
    $ [A \to B] $ și o săgeată
    \[
      \code{eval} : [A \to B] \times A \to B,
    \]
    cu proprietatea de universalitate: pentru orice săgeată $ f : C \times A \to B $,
    există o unică săgeată $ \lambda f : C \to [A \to B] $ astfel încît
    compunerea:
    \[
      C \times A \xrar{\lambda f \times A} [A \to B] \times A \xrar{\code{eval}} B
    \]
    să fie exact $ f $.

    Cu alte cuvinte, avem diagrama comutativă din figura \ref{fig:ccc}.
    \begin{figure}[!htbp]
      \centerline{
        \xymatrixcolsep{50px}
        \xymatrixrowsep{50px}
        \xymatrix{
          [A \to B] \times A \ar[r]^-{\code{eval}} & B \\
          C \times A \ar[u]^-{\lambda f \times A} \ar[ur]_-{f} &
        }
      }
      \caption{Proprietatea de universalitate a obiectului $ [A \to B] $}
      \label{fig:ccc}
    \end{figure}
  \end{enumerate}

  Obiectul $ [A \to B] $ se mai numește \emph{exponențială}, se mai notează $ B^A $,
  iar $ A $ se numește \emph{exponent}.
\end{definition}

Continuăm cu cîteva exemple esențiale.

\textbf{Categoria mulțimilor}, $ \code{Set} $ este cartezian închisă, cu structura
$ [A \to B] = \dr{Func}(A, B) $, mulțimea funcțiilor $ A \to B $, iar săgeata
\code{eval} este:
\[
  \code{eval} : [A \to B] \times A \to B, \quad \code{eval}(f)(a) = f(a).
\]

\vspace{1cm}

Orice \textbf{algebră Boole} $ B $ este CCC. Mai întîi, ea este un poset, deci are o categorie
asociată canonic $ \kal{C}(B) $, cu cel mult o săgeată între oricare două elemente
(definită dacă și numai dacă elementele sînt comparabile).

$ B $ are un obiect terminal, $ 1 $, iar produsul este $ \land $.

Pentru structura de CCC, definim acum $ [a \to b] = \lnot a \lor b $. Existența
săgeții $ \code{eval} $ înseamnă $ [a \to b] \land a \leq b $, deci, folosind
proprietatea de latice distributivă, obținem succesiv:
\begin{align*}
  [a \to b] \land a &= (\lnot a \lor b) \land a \\
                     &= (\lnot a \land a) \lor (b \land a) \\
                     &= 0 \lor (b \land a) \\
                     &= b \land a \leq b.
\end{align*}

Existența săgeții $ \lambda f : c \to [a \to b] $ asociate unei săgeți $ f : c \land a \to b $
înseamnă, de fapt:
\[
  c \land a \leq b \Rightarrow c \leq [a \to b].
\]
Să remarcăm că unicitatea va fi automată, deoarece în orice poset există cel mult o
săgeată între oricare două obiecte.

Mai departe, presupunem $ c \land a \leq b $. Rezultă:
\begin{align*}
  c &= c \land 1 \\
    &= c \land (a \lor \lnot a) \\
    &= (c \land a) \lor (c \land \lnot a) \\
    &\leq b \lor (c \land \lnot a) \\
    &\leq  b \lor \lnot a \\
    &= [a \to b].
\end{align*}

\vspace{1cm}

\textbf{Categoria cpo} cu funcții continue este CCC, dar \textbf{subcategoria strictă nu este}.

Demonstrația este destul de tehnică și folosește esențialmente detalii privitoare
la continuitate în topologia Scott, arătîndu-se faptul că strictețea nu se
păstrează obligatoriu în proprietatea de universalitate a obiectului exponențial.
Detalii se pot găsi în \cite{rey}, pp.\ 37-39.

\vspace{1cm}

Următoarele două exemple sînt prezentate nu tocmai riguros, dar ele pot servi drept
punct de plecare pentru o interpretare formală.

Putem privi un \textbf{limbaj simplu de programare funcțională ca alcătuind o categorie}.
Obiectele sînt tipurile, presupunem că fiecare tip are o operație \code{id}, iar
funcțiile sînt morfisme.

Dacă vrem ca acest limbaj să descrie o CCC, trebuie să ținem cont că:
\begin{itemize}
\item obiectul exponențial $\code{[A} \to \code{B]}$ va fi obiectul de funcții \code{func :: A -> B}.
  Axiomele CCC impun \qq{funcții clasa I}, adică obiectul $\code{[A} \to \code{B]}$ să se comporte
  ca orice alt obiect din categorie, i.e.\ tip, i.e.\ să poată fi luat ca argument
  de funcții;
\item proprietatea de universalitate a CCC poate ridica probleme de implementare, mai
  ales în forma generalizată cu produse $ n $-are.
\end{itemize}

\vspace{1cm}

Conform \cite{lambek}, o categorie poate fi gîndită ca un \textbf{sistem deductiv},
obiectele fiind formule, iar morfismele fiind demonstrații.

Dacă vrem ca această categorie să fie cartezian închisă, obținem reguli de
deducție binecunoscute.

Fie $ A, B $ formule. Atunci $ [A \to B] $ ar trebui să fie tot o formulă,
pe care o putem gîndi ca pe o implicație. Atunci \code{eval} este o
demonstrație care deduce $ B $ din $ A $ și $ [A \to B] $. Ea joacă,
de fapt, rolul \emph{modus ponens}.

Proprietatea de universalitate pornește cu $ f : C \times A \to B $, adică
o deducție a lui $ B $ din $ C $ și $ A $ (intuitiv, dar și formal, este chiar
conjuncția) și produce o deducție $ \lambda f : C \to [A \to B] $, care se mai
numește \emph{regula detașării}.

Mai multe detalii se pot găsi în \cite{lambek} și un rezumat succint în notițele
din semestrul anterior.


%%%%%%%%%%%%%%%%%%%%%%%%%%%%%%%%%%%%%%%%%%%%%%%%%%%%%%%%%%%%%%%%%%%%%%

\section{Concluzii și perspective}


\indent\indent În această prezentare, am oferit cîteva detalii privitoare la aplicații concrete
ale co\-res\-pon\-den\-ței de tip Curry--Howard--Lambek în semantica denotațională.

Fiecare dintre cele trei ramuri ale corespondenței fac obiectul studiului meu
ulterior, iar cîteva referințe bibliografice de urmărit sînt:
\begin{itemize}
\item îmbinarea teoriei domeniilor în special cu $\lambda$-calculul, dar și cu
  categoriile cartezian închise este detaliată excelent în \cite{amacur};
\item aspecte logice fundamentale, atît în ce privește logica clasică de ordinul
  întîi, dar și trecerea către sisteme logice moderne, cu aplicații în programarea
  funcțională sînt conținute în \cite{girlog} și \cite{girpf};
\item \qq{traducerea} logicii clasice, via algebre universale, către teoria
  categoriilor este prezentată succint, dar suficient de riguros în \cite{pitts}.
\end{itemize}

Desigur, în afară de aceste referințe punctuale, rămîn de interes deosebit monografiile:
\begin{itemize}
\item \cite{barls} și \cite{barlt} pentru toate noțiunile fundamentale privitoare la
  $ \lambda $-calcul, varianta tipizată sau nu;
\item \cite{jac}, pentru aspecte esențiale privitoare la interpretarea logicii în categorii;
\item \cite{lascott}, care pune laolaltă contribuțiile lui J.\ Lambek privitoare la
  interpretarea categoriilor drept sisteme deductive;
\item \cite{ch}, în care este sumarizată corespondență Curry--Howard, plină de detalii
  și demonstrații complete ale acestei legături.
\end{itemize}


\newpage
%%%%%%%%%%%%%%%%%%%%%%%%%%%%%%%%%%%%%%%%%%%%%%%%%%%%%%%%%%%%%%%%%%%%%%
% BIBLIOGRAFIE
\bibliography{densem.bib}
\bibliographystyle{apalike}
\nocite{*}
\addcontentsline{toc}{section}{\protect\numberline{}Bibliografie}

\end{document}
