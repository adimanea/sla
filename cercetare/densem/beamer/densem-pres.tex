\documentclass[xcolor=dvipsnames,handout]{beamer}

\newcommand\NN{\mathbb{N}}
\newcommand\ZZ{\mathbb{Z}}
\newcommand\QQ{\mathbb{Q}}
\newcommand\RR{\mathbb{R}}
\newcommand\CC{\mathbb{C}}
\newcommand\AF{\mathbb{A}}
\newcommand\PP{\mathbb{P}}
\newcommand\FF{\mathbb{F}}
\newcommand\bb{\mathbb}
\newcommand\dr{\mathrm}
\newcommand\kal{\mathscr}
\newcommand\fk{\mathfrak}
\newcommand\op{\oplus}
\newcommand\ot{\otimes}
\newcommand\ds{\displaystyle}
\newcommand\id{\indent}
\newcommand\nid{\noindent}
\newcommand\seq{\subseteq}
\newcommand\sneq{\subsetneq}
\newcommand\speq{\supseteq}
\newcommand\spneq{\supsetneq}
\newcommand\rar{\rightarrow}
\newcommand\Rar{\Rightarrow}
\newcommand\xrar{\xrightarrow}
\newcommand\lar{\leftarrow}
\newcommand\Lar{\Leftarrow}
\newcommand\xlar{\xleftarrow}
\newcommand\lrar{\leftrightarrow}
\newcommand\Lrar{\Leftrightarrow}
\newcommand\ideal{\trianglelefteq}
\newcommand\such{\text{ s.t. }}
\newcommand\sk{(\textit{Sketch}) }
\newcommand\rhup{\rightharpoonup}
\newcommand\rhdn{\rightharpoondown}
\newcommand\lhup{\leftharpoonup}
\newcommand\lhdn{\leftharpoondown}
\newcommand\incl{\hookrightarrow}
\newcommand\vid{\emptyset}
\newcommand\wt{\widetilde}
\newcommand\wh{\widehat}
\newcommand\ol{\overline}
\newcommand\Lor{\Longrightarrow}
\newcommand\Lol{\Longleftarrow}
\newcommand\Lolr{\Longleftrightarrow}
\newcommand\ld{{}^\ast}
\newcommand\lgr{{}^{\ast-\text{gr}}}
\newcommand\kap{\Bbbk^a[\kal{P}]}
\newcommand\kcp{\Bbbk^c[\kal{P}]}
\newcommand\fc{\mathfrak{c}}
\newcommand\qq{\enquote}
\newcommand\coli{\protect\varinjlim}
\newcommand\li{\protect\varprojlim}
\newcommand\injr{\rightarrowtail}
\newcommand\surjr{\twoheadrightarrow}
\newcommand\injl{\leftarrowtail}
\newcommand\surjl{\twoheadleftarrow}
\newcommand\sqsb{\sqsubseteq}

\usepackage{soul}
\newcommand{\code}[1]{\texttt{#1}}
\soulregister{\code}{1}

% font setup
\usepackage{libertine}
\renewcommand*\familydefault{\sfdefault}    % Linux Libertine = default sans serif
\usepackage{inconsolata}                    % Inconsolata = monospaced
\usepackage[utf8]{inputenc}
\usepackage[T1]{fontenc}
\usepackage{mathrsfs}

% Graphics and other packages
\usepackage[romanian]{babel}
\usepackage{graphicx}
\addto\captionsromanian{\renewcommand{\figurename}{Ilustrație}}
\usepackage{caption}
\usepackage{subcaption}
\usepackage[style=german]{csquotes}

% Custom macros
\newtheorem{bl}{}
\newcommand{\bloc}[3]{\begin{bl}<#1->{{\large\color{Gray}{\hrulefill}}\\ \color{bleumarin}{\large \emph{#2}}}\\ \vspace*{-2mm}{\color{Gray}{\hrulefill}}\\ #3 \end{bl}} 
\newcommand{\fr}[1]{\frame{#1}}
\newcommand{\ft}[1]{\frametitle{\color{bleumarin}{\hfill #1 \hfill}}}
\newcommand{\lin}[3]{\uncover<#1->{\alert<#1>{#2}}{\vspace*{#3 ex}}}
\newcommand{\ite}[2]{\uncover<#1->{\alert<#1>{\item #2}}}
\newcommand{\vs}[1]{\vspace*{#1 ex}}
\definecolor{bleumarin}{RGB}{30,30,150} 
\definecolor{firebrick}{RGB}{178,34,34}

% Theme setup
\useoutertheme{shadow} 
\usetheme{CambridgeUS} 
\usecolortheme[named=bleumarin]{structure} 
\useoutertheme[compress]{smoothbars}

% Theme finetuning
\setbeamertemplate{items}[ball]
\setbeamertemplate{blocks}[rounded][shadow=true]
\setbeamertemplate{navigation symbols}{}
\setbeamertemplate{headline}{}  


%%%%%%%%%%%%%%%%%%%%%%%%%%%%%%%%%%%%%%%%%%%%%%%%%%%%%%%%%%%%%%%%%%%%%% 
% TITLE PAGE
\title[DenSem]{Introducere în semantică denotațională}
\author{Adrian Manea}
\institute{SLA, 410}

\date{}

\begin{document}

\maketitle
%%%%%%%%%%%%%%%%%%%%%%%%%%%%%%%%%%%%%%%%%%%%%%%%%%%%%%%%%%%%%%%%%%%%%% 


% SLIDES START HERE
\fr{
  \ft{Context}

  \lin{1}{Tema: Corespondența Curry-Howard-Lambek.}{3}

  \lin{2}{Acum: Aplicații în semantică ale $\lambda$-calculului + CCC.}{3}

  \lin{3}{Resurse principale: \cite{schm} și \cite{abrj}.}{0}
}

\fr{
  \ft{Semantica punctelor fixe}

  \lin{1}{Problema: Specificațiile recursive}{0}

  \lin{2}{\begin{center}$\lambda x . xx$\end{center}}{0}

  \lin{3}{\begin{center}$ X = \dots X \dots $ \end{center}}{0}

  \lin{4}{Avem nevoie de o \emph{metodă canonică} de a rezolva ecuațiile pentru orice $ x, X $}{3}

  \lin{5}{$ \Rightarrow $ ecuația domeniilor (Scott) $ D \simeq [D \to D] $, cu $ D $ = \emph{domeniu semantic}}{3}

  \lin{6}{În categorii, $ F : \kal{C}^{op} \times \kal{C} \to \kal{C} \Rightarrow D \simeq F(D, D) $.}{1}
}

\fr{
  \ft{Exemplu: Semantica factorialului}

  \lin{1}{
    \begin{align*}
      \code{fac n} = \lambda \code{n}.&\code{n equals zero} \to \code{one} \\
                                      &\square \code{ n times (fac n minus one)}
    \end{align*}}{1}

  \lin{2}{După explicitare pe cazuri particulare $ \Rightarrow \Gamma \code{fac} = \ds\bigcup_{n \geq 0} \code{fac}_i $}{0}

  \lin{3}{Mai general, obținem funcționala:
    \begin{align*}
      \code{F} &: (\code{Nat} \to \code{Nat}_\bot) \to (\code{Nat} \to \code{Nat}_\bot) \\
      \code{F} &= \lambda \code{f}\lambda \code{n.n = 0} \to 1 \,\, \square \code{ n} \cdot \code{f(n-1)},
    \end{align*}
    cu $ \code{fac}_{i + 1} = \code{F fac}_i $.}{3}

  \lin{4}{Rezultă $ \Gamma \code{fac} = \ds\bigcup_{i \geq 0} \Gamma F^i \emptyset $ și
    $ \Gamma \code{F fac} = \Gamma \code{fac}$.}{1}

  \lin{5}{Folosind \emph{definiții extensionale}, \code{F fac = fac} , deci \code{fac}
    este \emph{punct fix} pentru \code{F}.}{1}
}

\fr{
  \ft{Domenii parțial ordonate}

  \bloc{1}{Definiții}
  {
    \lin{1}{$(D, \sqsubseteq)$ = \textbf{CPO} dacă orice submulțime total
      ordonată (\textbf{lanț}) are $\sup$ în $ D $.}{0}

    \lin{2}{$D$ = \textbf{punctat/strict} dacă este complet și
       are cel mai mic element.}{0}

    \lin{3}{$ f : A \to B $ monotonă între CPO = \textbf{continuă} dacă
      pentru orice lanț $ X \seq A, f(\sup(X)) = \sup_{x \in X}(f(x)) $.}{0}
  }

  \bloc{4}{Teoremă}
  {
    \lin{5}{$ D $ = PCPO, $ F : D \to D $ continuă.}{0}

    \lin{6}{$ F $ are un cel mai mic
      punct fix: \code{fix F} $ = \sup_{i \geq 0}(F^i \bot) $.}{0}
  }

  \bloc{7}{Definiție}
  {
    \lin{8}{Semantica specificației recursive $ f = Ff $ este \code{fix F}.}{0}
  }
}

\fr{
  \ft{Categorii cartezian închise}

  \bloc{1}{Definiție}
  {
    \lin{1}{Categoria $ \kal{C} $ se numește \textbf{cartezian închisă} (\textbf{CCC}) dacă:}{0}
    \begin{itemize}
      \ite{2}{are un obiect terminal, $ 1 $;}
      \ite{3}{orice pereche de obiecte are un produs direct și proiecții canonice;}
      \ite{4}{$\forall A, B \in \kal{C}, \exists [A \to B] \in \kal{C} $ și săgeata
        \[
          \code{eval} : [A \to B] \times A \to B,
        \]
        a.î.\ $ \forall f : C \times A \to B, \exists! {} \lambda f : C \to [A \to B] $,
        \[
          f = C \times A \xrar{\lambda f \times A} [A \to B] \times A \xrar{\code{eval}} B.
        \]
      }
    \end{itemize}
  }
}


\fr{
  \ft{Exemple de CCC}

  \lin{1}{\code{Set} este CCC, $ [A \to B] = \dr{Func}(A, B) $ și
    \[
      \code{eval} : [A \to B] \times A \to B, \quad \code{eval f a = f a}.
    \]
  }{1}

  \lin{2}{Orice algebră Boole este poset, deci categorie. Def.\ $ [a \to b] = \lnot a \lor b $.}{0}

  \lin{3}{\code{eval} asigură $ [a \to b] \land a \leq b $.}{0}

  \lin{4}{$ \lambda f : c \to [a \to b] $ asociată $ f : c \land a \to b $ înseamnă:
    \[
      c \land a \leq b \Rightarrow c \leq [a \to b].
    \]}{1}

  \lin{5}{Categoria \code{CPO} cu funcții continue este CCC, dar subcategoria strictă, nu.
    \cite{rey}.}{1}
}

\fr{
  \ft{Exemple de CCC (informale)}

  \lin{1}{Un limbaj simplu de programare funcțională = categorie (tipuri și funcții).}{3}

  \lin{2}{Existența $ [A \to B] $ cere \qq{funcții clasa I}.}{3}

  \lin{3}{Proprietatea de universalitate ridică probleme de implementare, mai ales pentru
    produse $ n $-are. (\cite{bw})}{3}
}

\fr{
  \ft{Exemple de CCC (informale)}

  \lin{1}{\cite{lambek}: Categorie = sistem deductiv (formule și demonstrații).}{3}

  \lin{2}{CCC: $[A \to B]$ = implicație. $ \code{eval} : [A \to B] \times A \to B $
    este \emph{modus ponens}.}{3}

  \lin{3}{Proprietatea de universalitate asociază $ f : C \times A \to B $
    o \qq{deducție} $ \lambda f : C \to [A \to B] $, care este
    \emph{regula detașării}.}{3}
}



%%%%%%%%%%%%%%%%%%%%%%%%%%%%%%%%%%%%%%%%%%%%%%%%%%%%%%%%%%%%%%%%%%%%%% 
% BIBLIOGRAFIE
\begin{frame}[allowframebreaks]
  \ft{Citări}
  \bibliography{../densem.bib}
  \bibliographystyle{apalike}
\end{frame}

\end{document}

