\chapter*{Concluzii și perspective}


\indent\indent În această prezentare, am oferit cîteva detalii privitoare la aplicații concrete
ale co\-res\-pon\-den\-ței de tip Curry--Howard--Lambek în semantica denotațională.

Fiecare dintre cele trei ramuri ale corespondenței fac obiectul studiului meu
ulterior, iar cîteva referințe bibliografice de urmărit sînt:
\begin{itemize}
\item îmbinarea teoriei domeniilor în special cu $\lambda$-calculul, dar și cu
  categoriile cartezian închise este detaliată excelent în \cite{amacur};
\item aspecte logice fundamentale, atît în ce privește logica clasică de ordinul
  întîi, dar și trecerea către sisteme logice moderne, cu aplicații în programarea
  funcțională sînt conținute în \cite{girlog} și \cite{girpf};
\item \qq{traducerea} logicii clasice, via algebre universale, către teoria
  categoriilor este prezentată succint, dar suficient de riguros în \cite{pitts}.
\end{itemize}

Desigur, în afară de aceste referințe punctuale, rămîn de interes deosebit monografiile:
\begin{itemize}
\item \cite{barls} și \cite{barlt} pentru toate noțiunile fundamentale privitoare la
  $ \lambda $-calcul, varianta tipizată sau nu;
\item \cite{jac}, pentru aspecte esențiale privitoare la interpretarea logicii în categorii;
\item \cite{lascott}, care pune laolaltă contribuțiile lui J.\ Lambek privitoare la
  interpretarea categoriilor drept sisteme deductive;
\item \cite{ch}, în care este sumarizată corespondență Curry--Howard, plină de detalii
  și demonstrații complete ale acestei legături.
\end{itemize}

%%% Local Variables:
%%% mode: latex
%%% TeX-master: "../densem"
%%% End:
