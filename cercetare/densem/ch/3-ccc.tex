\chapter{Categorii cartezian închise}
\label{sec:ccc}

\indent\indent Prezentarea de mai jos urmează \cite{bw} și arată cum domeniile semantice
introduse mai sus pot fi interpretate în contextul categoriilor cartezian
închise.

Începem cu o observație simplă: mulțimea cpo stricte și cu funcții continue
și stricte (i.e. $ f\bot = \bot $) formează o categorie, care este o subcategorie
a cazului nestrict.

Vom vedea cum cazul nestrict ne permite, de fapt, să lucrăm cu categorii
cartezian închise. Totodată, prin aceasta vom putea lega teoria domeniilor
de tip cpo de semantica $ \lambda $-calculului cu tipuri simple, asociat
canonic unei categorii cartezian închise.

\begin{definition}\label{def:ccc}
  Fie $ \kal{C} $ o categorie. Ea se numește \emph{cartezian închisă} (CCC)
  dacă:
  \begin{enumerate}[(CCC1)]
  \item are un obiect terminal $ 1 $;
  \item orice pereche de obiecte $ A, B \in \kal{C} $ are un produs direct
    $ A \times B $, împreună cu proiecțiile canonice:
    \[
      A \xleftarrow{p_1} A \times B \xrightarrow{p_2} B;
    \]
  \item pentru orice obiecte $ A, B \in \kal{C} $, există un obiect
    $ [A \to B] $ și o săgeată
    \[
      \code{eval} : [A \to B] \times A \to B,
    \]
    cu proprietatea de universalitate: pentru orice săgeată $ f : C \times A \to B $,
    există o unică săgeată $ \lambda f : C \to [A \to B] $ astfel încît
    compunerea:
    \[
      C \times A \xrar{\lambda f \times A} [A \to B] \times A \xrar{\code{eval}} B
    \]
    să fie exact $ f $.

    Cu alte cuvinte, avem diagrama comutativă din figura \ref{fig:ccc}.
    \begin{figure}[!htbp]
      \centerline{
        \xymatrixcolsep{50px}
        \xymatrixrowsep{50px}
        \xymatrix{
          [A \to B] \times A \ar[r]^-{\code{eval}} & B \\
          C \times A \ar[u]^-{\lambda f \times A} \ar[ur]_-{f} &
        }
      }
      \caption{Proprietatea de universalitate a obiectului $ [A \to B] $}
      \label{fig:ccc}
    \end{figure}
  \end{enumerate}

  Obiectul $ [A \to B] $ se mai numește \emph{exponențială}, se mai notează $ B^A $,
  iar $ A $ se numește \emph{exponent}.
\end{definition}

Continuăm cu cîteva exemple esențiale.

Categoria $ \code{Set} $ este cartezian închisă, cu structura $ [A \to B] = \dr{Func}(A, B) $,
mulțimea funcțiilor $ A \to B $, iar săgeata \code{eval} este:
\[
  \code{eval} : [A \to B] \times A \to B, \quad \code{eval}(f)(a) = f(a).
\]

\vspace{1cm}

Orice algebră Boole $ B $ este CCC. Mai întîi, ea este un poset, deci are o categorie
asociată canonic $ \kal{C}(B) $, cu cel mult o săgeată între oricare două elemente
(definită dacă și numai dacă elementele sînt comparabile).

$ B $ are un obiect terminal, $ 1 $, iar produsul este $ \land $.

Pentru structura de CCC, definim acum $ [a \to b] = \lnot a \lor b $. Existența
săgeții $ \code{eval} $ înseamnă $ [a \to b] \land a \leq b $, deci, folosind
proprietatea de latice distributivă, obținem succesiv:
\begin{align*}
  [ a \to b] \land a &= (\lnot a \lor b) \land a \\
                     &= (\lnot a \land a) \lor (b \land a) \\
                     &= 0 \lor (b \land a) \\
                     &= b \land a \leq b.
\end{align*}

Existența săgeții $ \lambda f : c \to [a \to b] $ asociate unei săgeți $ f : c \land a \to b $
înseamnă, de fapt:
\[
  c \land a \leq b \Rightarrow c \leq [a \to b].
\]
Să remarcăm că unicitatea va fi automată, deoarece în orice poset există cel mult o
săgeată între oricare două obiecte.

Mai departe, presupunem $ c \land a \leq b $. Rezultă:
\begin{align*}
  c &= c \land 1 \\
    &= c \land (a \lor \lnot a) \\
    &= (c \land a) \lor (c \land \lnot a) \\
    &\leq b \lor (c \land \lnot a) \\
    &\leq  b \lor \lnot a \\
    &= [a \to b].
\end{align*}

\vspace{1cm}

Categoria cpo cu funcții continue este CCC, dar subcategoria strictă nu este.
\todo[inline,noline,backgroundcolor=green!40]{DEMONSTRAȚIE}

\vspace{1cm}

Următoarele două exemple sînt prezentate nu tocmai riguros, dar ele pot servi drept
punct de plecare pentru o interpretare formală.

Putem privi un \emph{limbaj simplu de programare funcțională} ca alcătuind o categorie.
Obiectele sînt tipurile, presupunem că fiecare tip are o operație \code{id}, iar
funcțiile sînt morfisme.

Dacă vrem ca acest limbaj să descrie o CCC, trebuie să ținem cont că:
\begin{itemize}
\item obiectul exponențial $ [A \to B] $ va fi obiectul de funcții $ func :: A -> B $.
  Axiomele CCC impun \qq{funcții clasa I}, adică obiectul $ [A \to B] $ să se comporte
  ca orice alt obiect din categorie, i.e.\ tip, i.e.\ să poată fi luat ca argument
  de funcții;
\item proprietatea de universalitate a CCC poate ridica probleme de implementare, mai
  ales în forma generalizată cu produse $ n $-are.
\end{itemize}

\vspace{1cm}

Conform \cite{lambek}, o categorie poate fi gîndită ca un sistem deductiv,
obiectele fiind formule, iar morfismele fiind demonstrații.

Dacă vrem ca această categorie să fie cartezian închisă, obținem reguli de
deducție binecunoscute.

Fie $ A, B $ formule. Atunci $ [A \to B] $ ar trebui să fie tot o formulă,
pe care o putem gîndi ca pe o implicație. Atunci \code{eval} este o
demonstrație care deduce $ B $ din $ A $ și $ [A \to B] $. Ea joacă,
de fapt, rolul \emph{modus ponens}.

Proprietatea de universalitate pornește cu $ f : C \times A \to B $, adică
o deducție a lui $ B $ din $ C $ și $ A $ (intuitiv, dar și formal, este chiar
conjuncția) și produce o deducție $ \lambda f : C \to [A \to B] $, care se mai
numește \emph{regula detașării}.

\todo[inline,noline,backgroundcolor=green!40]{detalii din Lambek?}

%%% Local Variables:
%%% mode: latex
%%% TeX-master: "../densem"
%%% End:
