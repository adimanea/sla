\chapter*{Introducere}


\indent\indent Lucrarea de față continuă studiul corespondenței Curry--Howard--Lambek,
expusă pe scurt într-o prezentare de ansamblu în semestrul anterior. De data aceasta,
ne concentrăm asupra noțiunilor corespunzătoare \emph{semanticii denotaționale a %
  limbajelor de programare} și introducem cîteva dintre uneltele matematice utile
pentru a studia \emph{teoria domeniilor}. Vom insista pe mulțimile parțial ordonate
complete și stricte, cărora le vom justifica utilitatea în specificații recursive
în topologia Scott (cf.\ \cite{scottstrachey}) și pe care le vom lega și de categorii
cartezian închise.

În particular, prezentarea de față pune accentul pe partea computațională a
co\-res\-pon\-den\-ței Curry--Howard--Lambek, prin semantica denotațională, dar totodată
evidențiază și formalismul teoriei categoriilor pentru aceste scopuri, formalism
care, conform \cite{lambek}, reprezintă un mediu excelent pentru interpretarea
sistemelor logice deductive.

%%% Local Variables:
%%% mode: latex
%%% TeX-master: "../densem"
%%% End:
