\documentclass[a4paper]{article}

\usepackage{libertine}
\usepackage{libertinust1math}
\usepackage{inconsolata}
\usepackage{amsthm, amsmath}
\usepackage[style=german]{csquotes}
\usepackage[romanian]{babel}
\usepackage{enumerate}
\usepackage{xcolor}
\usepackage{graphicx}
\usepackage[all]{xy}
\usepackage{hyperref}
\hypersetup{citecolor=blue}
\usepackage{ebproof} % for proof trees
\usepackage{natbib} % for citekey hyphenation

% THEOREMS
\newtheoremstyle{dotless}{}{}{\itshape}{}{\bfseries}{:}{ }{}
\theoremstyle{dotless}
\newtheorem{theorem}{Teorem\u{a}}[chapter]
\newtheorem{proposition}{Propozi\c{t}ie}[chapter]
\newtheorem{lemma}{Lem\u{a}}[chapter]
\newtheorem{corollary}{Corola}[chapter]
\newtheorem{axiom}{Axiom\u{a}}[chapter]
\newtheorem{principle}{Principiu}[chapter]
\newtheorem{reg}{Regul\u{a}}[chapter]
\renewcommand*{\proofname}{Dem.:}

\newtheoremstyle{dotlessdr}{}{}{}{}{\bfseries}{:}{ }{}
\theoremstyle{dotlessdr}
\newtheorem{example}{Exemplu}[chapter]
\newtheorem{definition}{Defini\c{t}ie}[chapter]
\newtheorem{remark}{Observa\c{t}ie}[chapter]
\newtheorem{exercise}{Exerci\c{t}iu}[chapter]
\numberwithin{equation}{chapter}
\newcommand\NN{\mathbb{N}}
\newcommand\ZZ{\mathbb{Z}}
\newcommand\QQ{\mathbb{Q}}
\newcommand\RR{\mathbb{R}}
\newcommand\CC{\mathbb{C}}
\newcommand\FF{\mathbb{F}}
\newcommand\bb{\mathbb}
\newcommand\dr{\mathrm}
\newcommand\kal{\mathscr}
\newcommand\fk{\mathfrak}
\newcommand\op{\oplus}
\newcommand\ot{\otimes}
\newcommand\ds{\displaystyle}
\newcommand\id{\indent}
\newcommand\nid{\noindent}
\newcommand\seq{\subseteq}
\newcommand\sneq{\subsetneq}
\newcommand\speq{\supseteq}
\newcommand\spneq{\supsetneq}
\newcommand\rar{\rightarrow}
\newcommand\Rar{\Rightarrow}
\newcommand\xrar{\xrightarrow}
\newcommand\lar{\leftarrow}
\newcommand\Lar{\Leftarrow}
\newcommand\xlar{\xleftarrow}
\newcommand\lrar{\leftrightarrow}
\newcommand\Lrar{\Leftrightarrow}
\newcommand\ideal{\trianglelefteq}
\newcommand\such{\text{ s.t. }}
\newcommand\sk{(\textit{Sketch}) }
\newcommand\rhup{\rightharpoonup}
\newcommand\rhdn{\rightharpoondown}
\newcommand\lhup{\leftharpoonup}
\newcommand\lhdn{\leftharpoondown}
\newcommand\incl{\hookrightarrow}
\newcommand\vid{\emptyset}
\newcommand\wt{\widetilde}
\newcommand\wh{\widehat}
\newcommand\ol{\overline}
\newcommand\Lor{\Longrightarrow}
\newcommand\Lol{\Longleftarrow}
\newcommand\Lolr{\Longleftrightarrow}
\newcommand\ld{{}^\ast}
\newcommand\lgr{{}^{\ast-\text{gr}}}
\newcommand\kap{\Bbbk^a[\kal{P}]}
\newcommand\kcp{\Bbbk^c[\kal{P}]}
\newcommand\fc{\mathfrak{c}}
\newcommand\qq{\enquote}
\newcommand\coli{\protect\varinjlim}
\newcommand\li{\protect\varprojlim}
\newcommand\injr{\rightarrowtail}
\newcommand\surjr{\twoheadrightarrow}
\newcommand\injl{\leftarrowtail}
\newcommand\surjl{\twoheadleftarrow}
\newcommand\sqsb{\sqsubseteq}
% hfill in math mode (e.g. align)
\makeatletter
\newcommand{\pushright}[1]{\ifmeasuring@#1\else\omit\hfill$\displaystyle#1$\fi\ignorespaces}
\newcommand{\pushleft}[1]{\ifmeasuring@#1\else\omit$\displaystyle#1$\hfill\fi\ignorespaces}
\makeatother
% FONTS
\usepackage[mathscr]{euscript}
\usepackage[normalem]{ulem}
\usepackage{eurosym}
\usepackage{sectsty}
\allsectionsfont{\rmfamily}
\usepackage{anyfontsize}

% PAGE SETUP
\usepackage[portrait]{geometry}
\geometry{
  a4paper,
  left=20mm,
  right=25mm,
  % marginparwidth=40mm,
}
\usepackage{afterpage}
\newcommand\blankpage{
  \null
  \thispagestyle{empty}
  \addtocounter{page}{-1}
  \newpage}
\newenvironment{myenv}[1]
  {\begin{spacing}{#1}}
    {\end{spacing}}

% FANCY SETUP
\usepackage[Glenn]{fncychap}
\usepackage{fancyhdr}
\addto\captionsromanian{\renewcommand{\chaptername}{Seminar}}



\begin{document}
\begin{center}
  {{\Large\textbf{Corespondența Curry--Howard--Lambek}} \\
    --- \textit{raport de cercetare, anul 1, semestrul I} --- \\
    Adrian Manea \\
    coordonator: Conf.\ dr.\ Denisa Diaconescu
  }
\end{center}

\vspace{2cm}

\tableofcontents

%%%%%%%%%%%%%%%%%%%%%%%%%%%%%%%%%%%%%%%%%%%%%%%%%%%

\section{Introducere}

\indent\indent Tema de cercetare pe care o abordez este, în termeni
largi, \emph{corespondența Curry--Howard--Lambek}, cu mul\-te\-le sale fațete
din logică, filosofie, matematică și informatică. Prezentarea curentă
urmărește, într-o oarecare măsură, articolul \cite{wad}, în mod special în
partea introductivă și cea despre logică. \\


În multe situații din istoria științei și nu numai s-a dovedit că un punct de
vedere dualist asupra unor noțiuni nu aduce decît beneficii, clarificări
suplimentare și relevă similitudini care în primă fază nu erau evidente.

A fost cazul, de-a lungul istoriei, în mai multe instanțe, precum:
\begin{itemize}
  \item sistemul de coordonate al lui Descartes, care a legat geometria de
    algebră, introducînd totodată și un mod intuitiv de a vizualiza entitățile
    algebrice abstracte (care în cam aceeași perioadă au primit notațiile
    pe care le utilizăm astăzi, tot de la Descartes, Vi\`{e}te și alții);
  \item teoria cuantică a lui Planck, ce a introdus și evidențiat dualitatea undă-corpuscul;
  \item teoria informației a lui Shannon, cu conexiuni nebănuite
    cu termodinamica, prin teoria entropiei.
\end{itemize}

Un astfel de rezultat este și cel pe care îl prezentăm aici, anume legătura
între propoziții (logice) și tipuri. Acestea din urmă, la rîndul lor, și-au
arătat polivalența, producînd ecouri în algebră, teoria limbajului, filosofie
și informatică, pentru a da doar cîteva exemple. Conexiunea, așa cum a apărut
ea inițial, poartă numele celor trei matematicieni care au evidențiat-o,
anume logicienii Haskell Curry, William Howard și algebristul Joakim Lambek.

Legătura pe care cei trei au evidențiat-o (lucrînd independent și pentru a
rezolva probleme aparent diferite) își are originile în filosofie și matematică,
mai precis în așa numita \emph{interpretare Brouwer--Heyting--Kolmogorov} (BHK)
a logicii (intuiționiste), denumită după filosofii și matematicienii
L.\ E.\ J.\ Brouwer, A.\ Heyting și A. Kolmogorov. Această interpretare, 
precum mare parte a logicii intuiționiste, are o abordare constructivistă 
asupra problemelor matematice și logice, privind propozițiile logice ca pe 
probleme ce trebuie rezolvate. Putem formula abordarea lor, anticipînd
dezvoltările istorice, prin legătura cu teoria mulțimilor sau cu teoria
funcțiilor. Astfel, dacă o mulțime (respectiv, o funcție) este privită ca o
propoziție, i.e.\ ca prezentînd o afirmație sau o proprietate, ea este
instanțiată (\emph{demonstrată}) de elementele sale, respectiv de argumentele
și valorile sale, în cazul funcțiilor. Așa prezentată, teoria se aseamănă
cu \emph{teoria realizării}, formulată mai tîrziu de S.\ Kleene, care se
remarcă tot prin abordarea constructivă, însă o prezentare a acesteia
din urmă depășește scopurile lucrării prezente.

Conceptul de \qq{propoziții ca tipuri} (preluînd exprimarea din \cite{wad})
are o profunzime care nu este evidentă de la început. În mare, această
legătură afirmă că există o corespondență bijectivă între propozițiile
unei logici și tipurile dintr-un limbaj de programare.

Dar legătura merge chiar mai departe. Propoziția respectivă trebuie să
aibă o demonstrație, care va corespunde unui program în respectivul limbaj, 
ce face uz de tipul corespunzător propoziției. Deci obținem mai departe
o conexiune de tip \qq{demonstrații ca programe}.

Încă mai departe, atunci cînd vrem să simplificăm o propoziție
(prin rescriere, după cum vom vedea), aceasta revine la rularea programului,
deci avem \qq{simplificarea demonstrațiilor ca evaluarea programelor}.

Istoria modernă a dezvoltării acestei conexiuni se poate trasa de la
sfîrșitul secolului al XIX-lea și începutul secolului XX. Inspirat de
puterea de expresie a logicii formale, precum a fost relevat de Russell
și Whitehead în a lor monumentală \emph{Principia Mathematica}, Hilbert
a pornit un program pentru rezolvarea \emph{problemei deciziei},
adică de a dezvolta o procedură calculabilă eficient care să determine dacă
o afirmație dată este adevărată sau falsă. Programul său conține o presupunere
implicită de \emph{completitudine} a matematicii, anume că, dată o propoziție,
fie ea ori negația ei are o demonstrație.

Dar G\"{o}del a arătat că aritmetica este incompletă și mai mult, programul
lui Hilbert s-a mai lovit de o dificultate, anume aceea de a stabili precis
ce înseamnă o \qq{calculabilitate rezonabilă} a metodei de decizie.

Asemenea definiții au apărut abia în anii '30, cu 3 alternative diferite:
\begin{itemize}
  \item $\lambda$-calculul lui Alonzo Church (1936);
  \item teoria funcțiilor recursive, a lui G\"{o}del și Kleene (1934, 1936);
  \item mașinile Turing, propuse de Alan Turing (1937).
\end{itemize}

Pe scurt, teoria lui Church a pornit din dorința matematicienilor de clarifica
noțiunea de \emph{funcție}. Astfel, Church introduce $\lambda$-abstracția,
în primă fază privită ca un concept tehnic, utilizat la notație și care să
ia din misterul și dificultatea definirii conceptului de funcție.

G\"{o}del nu a fost convins de aborarea lui Church și, atunci cînd a vizitat
Princeton, într-o serie de cursuri, a dezvoltat teoria funcțiilor recursive,
împreună cu Kleene. Church, încrezător, dorea inițial să arate că această
teorie este inclusă în teoria sa de $\lambda$-calcul, dar, pînă la urmă,
cele două s-au dovedit echivalente.

Turing, de cealaltă parte a Oceanului, la Cambridge, a pornit cu o motivație
complet diferită. El și-a propus să înțeleagă conceptul de calculabilitate
cît mai uman. Astfel, el a introdus multe condiții de finitudine sau mărginire
a resurselor folosite, deoarece a modelat teoria pe capacitățile umane.

De asemenea, și teoria lui Turing a fost demonstrat a fi echivalentă cu
celelalte două. Însă Kleene a demonstrat că, în forma originală, sistemul
de $\lambda$-calcul conduce la inconsistență, arătînd cazuri cînd o abordare
de tip paradoxul lui Russell (i.e.\ prin autoaplicare) poate conduce la
calcule infinite. De aceea, Church, ca și Russell, a introdus o ierarhie
de forma teoriei tipurilor, formulînd sistemul $\lambda$ cu tipuri. Totodată,
introducînd tipurile, sistemul lui Church beneficiază și de existența unei
forme normale pentru fiecare termen.


%%%%%%%%%%%%%%%%%%%%%%%%%%%%%%%%%%%%%%%%%%%%%%%%%%%%%%%%%%%%%%%%

\section{Aspecte logice --- Sistemul lui Gentzen}
\indent\indent În 1935, Gerhard Gentzen a introdus 2 formulări ale logicii
care să ajute teoria demonstrației: \emph{deducția naturală} și \emph{calculul %
secvent}. Prin aceste sisteme, el a arătat cum se poate obține o formă
normală pentru demonstrații, care să asigure că acestea nu sînt circulare sau
infinite. El a introdus și cuantificarea universală, ca o paralelă la
cuantificarea existențială introdusă deja de Peano. Simbolurile folosite
de Gentzen sînt $ \supset $ pentru implicație, \& pentru conjuncție și
$ \lor $ pentru disjuncție.

Una dintre ideile originale ale lui Gentzen a fost de a impune ca regulile
de demonstrație să apară în perechi, în cazul deducției naturale fiind
vorba despre \emph{reguli de introducere} și \emph{reguli de eliminare}.
Regula de introducere arată cînd se poate formula o aserțiune care conține
un conector logic, iar regula de eliminare arată cum se poate folosi conectorul
(e.g.\ în \emph{modus ponens}, care arată cum se folosește conectorul $\supset$).

Un aspect foarte util al sistemului lui Gentzen a fost așa-numitul
\qq{principiu al subformulelor}. Conform acestuia, într-o demonstrație a unei
formule nu pot apărea decît subformule ale acesteia sau termeni ai subformulelor.
Aceasta este condiția de normalizare a demonstrației. O consecință imediată a fost
consistența. Pentru a putea demonstra o contradicție, ar trebui să putem demonstra
folosind subformule ale acesteia, dar așa ceva nu există! Rezultatul este formulat
foarte sugestiv de Wadler prin ``\emph{What part of} no \emph{don't you understand?}''.

Calculul secvent (eng.\ \emph{sequent calculus}) a fost introdus și folosit de
Gentzen mai mult ca un instrument tehnic, cu ajutorul lui demonstrînd principiul
subformulelor de mai sus.

Vom prezenta acum pe scurt aspecte din deducția naturală, pentru a evidenția
ulterior legătura cu $ \lambda $-calculul cu tipuri simple și categorii
cartezian închise. Pentru simplitate, vom discuta doar doi conectori,
\& și $ \supset $. Regulile, așa cum au fost introduse de Gentzen, sînt:
\begin{itemize}
  \item introducerea conjuncției:
    \begin{prooftree}
        \hypo{A \quad B}
        \infer1[\&I]{A \& B}
    \end{prooftree};
  \item eliminarea conjuncției, de pe poziția 1 sau 2:
    \begin{prooftree}
      \hypo{A \& B} \infer1[\&E1]{A}
    \end{prooftree}, respectiv
    \begin{prooftree}
      \hypo{A \& B} \infer1[\&E2]{B}
    \end{prooftree};
  \item introducerea implicației:
    \begin{prooftree}
      \hypo{[A]^x} \ellipsis{}{B} \infer1[\supset I^x]{A \supset B}
    \end{prooftree};
  \item eliminarea implicației (sau \emph{modus ponens}):
    \begin{prooftree}
      \hypo{A \supset B \quad A} \infer1[\supset E]{B}
    \end{prooftree}.
\end{itemize}

Regula de introducere a implicației afirmă că dacă din presupunerea că $ A $
are loc putem obține $ B $, atunci putem concluziona că are loc $ A \supset B $
și să neglijăm presupunerea $ A $. Demonstrația se încheie doar cînd fiecare
presupunere din ea a fost eliminată, folosind regula $ \supset I $. Am notat cu
indicele superior $ x $ pentru a arăta care presupunere este eliminată atunci
cînd se aplică introducerea corespunzătoare.

\begin{remark}
  Putem face o remarcă filosofică de felul următor. Într-o regulă de deducție
  care folosește implicația drept premisă, implicația însăși este un fel de
  deducție, la nivel meta (deducția care ne interesează fiind la nivel de obiect).
  Această dificultate aparentă, cum că ar trebui să înțelegem mai întîi implicația
  pentru a înțelege implicația, a fost discutată de Per Martin--L\"{o}f în
  articolul \cite{pmlf}.
\end{remark}

Regulile de rescriere pentru simplificarea demonstrațiilor sînt:
\[
  \begin{prooftree}
    \hypo{\stackrel{\vdots}{A} \quad \stackrel{\vdots}{B}}
    \infer1[\& I]{A \& B}
    \infer2[\&E1]{A}
  \end{prooftree} \Longrightarrow
  \begin{prooftree}
    \ellipsis{}{A}
  \end{prooftree} \qquad \qquad
  \begin{prooftree}
    \hypo{[A]^x} \ellipsis{}{B}
    \infer1[\supset I^x]{A \supset B} \hypo{\stackrel{\vdots}{A}}
    \infer2[\supset E]{B}
  \end{prooftree} \Longrightarrow
  \begin{prooftree}
    \ellipsis{}{A} \ellipsis{}{B}
  \end{prooftree}
\]


Regulile de simplificare ne arată cum să rescriem o demonstrație atunci cînd apar
consecutiv regulile de introducere și de eliminare pentru același conector.
Numim \emph{redex} varianta nesimplificată și \emph{reduct} varianta rescrisă.

Regula de simplificare pentru \&, de exemplu, neglijează orice apariție a lui $ B $,
pentru că, de fapt, vrem să demonstrăm $ A $.

Regula de simplificare pentru $ \supset $ ține seama de faptul că vrem să obținem
$ B $ via $ A $, dar elimină ocolișul nenecesar prin demonstrația lui $ A \supset B $.

Aduse în forma normală, i.e.\ aplicînd toate rescrierile posibile, este clar că
demonstrațiile respectă principiul subformulei.


%%%%%%%%%%%%%%%%%%%%%%%%%%%%%%%%%%%%%%%%%%%%%%%%%%%%%%%%%%%%%%%

\section{$\lambda$-calcul cu tipuri}

\indent\indent Descriem acum $ \lambda $-calculul cu tipuri simple al
lui Church. Pentru simplitate, putem presupune că produsele și funcțiile
sînt obiecte primitive (în articolul original, Church a derivat produsele
din funcții). În loc de formule, judecățile de forma $ M : A $ servesc
drept premise și concluzii și notează că termenul $ M $ are tipul $ A $.

Regulile de tipizare sînt:
\begin{itemize}
  \item introducerea produselor:
    \begin{prooftree}
      \hypo{M : A \quad N : B}
      \infer1[\times I]{\langle M, N \rangle : A \times B}
    \end{prooftree};
  \item introducerea $ \lambda $-abstracției:
    \begin{prooftree}
      \hypo{[x : A]^x} \ellipsis{}{N : B}
      \infer1[\to I^x]{\lambda x . N : A \to B}
    \end{prooftree};
  \item eliminarea produsului, pe prima sau a doua componentă, prin proiecțiile
    canonice, $ \pi_1(L : A \times B) : A $, respectiv $ \pi_2(L : A \times B) : B $;
  \item eliminarea $\lambda$-abstracției, prin aplicare:
    \begin{prooftree}
      \hypo{L : A \to B \quad M : A}
      \infer1[\to E]{LM : B}
    \end{prooftree}.
\end{itemize}

Observăm, de asemenea, că în regula de introducere a $\lambda$-abstracției, variabila
$ x $ apare liberă în $ N $ în premisă și legată în concluzia $ \lambda x . N $.

\begin{remark}
  Distincția între nivelul meta si nivelul obiect în ce privește implicația
  este mai clară în $ \lambda $-calcul. În cazul premiselor, avem o implicație,
  dar la nivel de obiect, în concluzie, avem \emph{funcții}, obiecte diferite.
\end{remark}

Demonstrațiile în $\lambda$-calculul tipizat se numesc \emph{programe}, care
pot fi evaluate prin rescriere, conform regulilor:
\[
  \begin{prooftree}
    \hypo{\stackrel{\vdots}{M : A} \quad \stackrel{\vdots}{N : B}}
    \infer1[\times I]{\langle M, N \rangle : A \times B}
    \infer2[\times E1]{\pi_1\langle M, N \rangle : A}
  \end{prooftree} \Longrightarrow
  \begin{prooftree}
    \ellipsis{}{M : A}
  \end{prooftree}
  \qquad \qquad
  \begin{prooftree}
    \hypo{[x : A]^x} \ellipsis{}{N : B}
    \infer1[\to I^x]{\lambda x . N : A \to B} \hypo{\stackrel{\vdots}{M : A}}
    \infer2[\to E]{(\lambda x.N)M : B}
  \end{prooftree} \Longrightarrow
  \begin{prooftree}
    \ellipsis{}{M : A}
  \ellipsis{}{N[M/x] : B}
  \end{prooftree}
\]

Regulile de rescriere păstrează tipizarea, proprietate numită \emph{corectitudinea %
tipizării} (eng.\ \emph{type soundness}).

Termenul $ N[M/x] $ din a doua regulă de rescriere notează că se înlocuiesc
toate aparițiile libere ale lui $ x $ din $N$ cu $M $. De asemenea, acest
termen substituit poate fi, în general, mai mare decît cel inițial (în funcție de
numărul de apariții ale lui $ x $). Dar, în orice caz, el este mai simplu,
deoarece nu mai are un subtermen de tip $ A \to B $. Astfel, eliminarea asumpțiilor
poate fi înțeleasă ca aplicarea funcțiilor.

\vspace{1cm}

Prezentate astfel, legăturile între deducția naturală și $ \lambda $-calculul
cu tipuri simple sînt evidente. Așa cum am menționat la început, ele merg mult mai
profund decît atît, fiind prezente la mai multe niveluri. Fiecare dintre aceste
niveluri suplimentare cer o aprofundare a subiectelor, ce va fi obiectul cercetărilor
ulterioare.

%%%%%%%%%%%%%%%%%%%%%%%%%%%%%%%%%%%%%%%%%%%%%%%%%%%%%%%%%%%%%%%
\section{Categorii --- Privire de ansamblu}

\indent\indent Trecem acum la un alt nivel care evidențiază legătura prezentată,
anume acela al categoriilor, conexiune introdusă de J.\ Lambek începînd cu
anul 1968. Prezentarea sumară ce urmează se bazează pe articolul \cite{lambek}
și presupune o familiarizare preliminară cu teoria categoriilor.

Conform lui Lambek, există două moduri nestandard de a privi o categorie,
pe care le vom numi \emph{interpretarea logică}, respectiv \emph{interpretarea %
algebrică}. Prima dintre acestea privește o categorie drept un \emph{sistem %
deductiv}, în care obiectele sînt formule logice, iar săgețile, deducții.
Egalitățile care există în axiomele care definesc categoriile sînt caracteristici
bonus, care vor servi atunci cînd avem nevoie de teorii logice cu egalitate.
Săgețile identitate pentru fiecare obiect $ A $ al unei categorii sînt gîndite
ca axiome, iar compunerea săgeților devine tranzitivitatea consecinței logice
(eng.\ \emph{entailment}):
\[
  \begin{prooftree}
    \hypo{f : A \to B \quad g : B \to C}
    \infer1{gf : A \to C}
  \end{prooftree}
\]

Egalitățile suplimentare care există (proprietatea săgeții unitate, respectiv
asociativitatea compunerii) sugerează necesitatea introducerii unei relații
de (tip) egalitate între deducții, discutată de Lambek, Prawitz și alții,
dar care depășește scopurile lucrării prezente.

Interpretarea algebrică privește o catgorie drept o \emph{teorie algebrică %
unară, multisortată}. Obiectele sînt numite sorturi sau tipuri, iar săgețile
sînt operații unare. Aceasta diferă în mod esențial de abordarea mai des
întîlnită, conform căreia teoria este uni-sortată, iar operațiile (funcțiile)
sînt $n$-are. În abordarea curentă, o asemenea interpretare ar implica
$n$-categorii.

În general, teoria categoriilor are meritul de a nu mai lucra cu variabile,
dar aici vom reintroduce variabilele și vom vedea cît ajută la formularea
rezultatelor. De fapt, este cunoscut faptul că se poate da o reformulare
a teoriei categoriilor folosind doar săgeți, cu anumite constrîngeri (care vor
avea loc în cazurile ce ne interesează aici), iar un element $ x $ al unui
obiect $ A $ poate fi gîndit ca o săgeată $ x : 1 \to A $ de la obiectul
inițial al categoriei (în ipoteza că există un asemenea obiect).

Fie, deci, $ x $ o variabilă de sort $ A $ (în terminologia de mai sus,
adică un element $ x $ al obiectului $ A $ din categorie). Fie, de asemenea,
$ f : A \to B $ o operație. Atunci $ fx $ este un termen de sort $ B $
în ceea ce vom numi \emph{limbajul intern} al categoriei, privită ca o
teorie algebrică unară, multisortată.

O asemenea abordare va fi de folos și în interpretarea logică (constructivistă):
variabila $ x $ de sort $ A $ poate fi gîndită ca o presupunere că $ A $
are loc, iar $ fx $, de sort $ B $, să fie demonstrația lui $ B $ din
această presupunere.

Să prespunem, așadar, că lucrăm într-o categorie care are produse finite.
Aceasta implică existența unui produs vid, care este un obiect unic $ 1 $,
numit \emph{obiect terminal} în categorie, împreună cu o săgeată unică,
pentru fiecare obiect $ A $, de forma $ 0_A : A \to 1 $. Pentru obiectele
$ A $ și $ B $, produsul se identifică cu produsul cartezian (în cazul în care
$ A $ și $ B $ sînt mulțimi) și avem săgețile canonice date de proiecții,
care induc o corespondență bijectivă între perechi de săgeți $ (C \to A, C \to B )$
și săgeți $ C \to A \times B $. Astfel definită, extinzînd produsul și la
cazul finit arbitrar, se obține o \emph{categorie carteziană}.

Interpretarea logică a acestei construcții este imediată: obiectul terminal poate
fi gîndit ca $ \top $ (adevărat), iar produsul $ A \times B $, drept
conjuncția $ A \land B $. Atunci avem axiomele:
\[
  0_A : A \to \top, \quad \pi_1 : A \land B \to A, \quad \pi_2 : A \land B \to B
\]
și o regulă de inferență:
\[
  \begin{prooftree}
    \hypo{f : C \to A \quad g : C \to B}
    \infer1{\langle f, g \rangle : C \to A \land B}
  \end{prooftree}
\]

Suplimentar, sînt satisfăcute și:
\begin{itemize}
  \item $ k = 0_A $ pentru orice săgeată $ k : A \to \top $;
  \item $ \pi_1 \langle f, g \rangle = f, \pi_2 \langle f, g \rangle = g $;
  \item $ \langle \pi_1 h, \pi_2 h \rangle = h $ pentru orice săgeată $ h : C \to A \land B $.
\end{itemize}

Un termen $ \varphi(x, y) $ de tip $ C $ din limbajul intern, cu $ x $ de tip $ A $
și $ y $ de tip $ B $ poate fi gîndit ca o demonstrație în deducție naturală
a lui $ C $ din presupunerea $ x $ că $ A $ și $ y $ că $ B $. O proprietate
interesantă (rezultată din proprietatea de universalitate a produsului din categorie)
este că limbajul intern are \emph{completitudine funcțională}, adică orice
demonstrație $ \varphi(x, y) $ are forma $ f(x, y) $ pentru o unică deducție
$ f : A \land B \to C $.

Putem folosi obiectul $ 1 $ al categoriei pentru a nota cu o săgeată nedeterminată
$ x : 1 \to A $ în \emph{categoria polinomială}\footnotemark
a categoriei date, $ \kal{C}[x] $
și atunci putem formula completitudinea funcțională astfel.

\footnotetext{Categoria polinomială este o construcție naturală care indexează
  obiectele unei categorii după un anumit obiect (sau, mai precis, după o anumită
  săgeată $ 1 \to A $ într-un obiect). Din acest punct de vedere, anticipînd un subiect
  pe care nu îl detaliem aici, se poate face o legătură cu \emph{tipuri dependente},
gîndite categorial ca niște \emph{fibrări} (eng.\ \emph{bundles})}

Fie $ p(x) : B \to C $ un polinom în nedeterminata $ x : 1 \to A $. Există o unică
săgeată $ f : A \times B \to C $ care nu depinde de $ x $ astfel încît $ p(x) = f\chi(x) $,
cu $ \chi(x) = \langle x0_B, 1_B \rangle $, conform diagramei comutative:
\[
  \xymatrix{
  A \times B \ar[r]^-f & C \\
  B \ar[u]^-{\chi(x)} \ar[ur]_-{p(x)} &
}
\]

\begin{remark}\label{rk:exists}
  La prima vedere, am putea gîndi $ A \times B $ și ca fiind coprodusul a $ A $
  copii ale lui $ B $ și să scriem:
  \[
    A \times B = \coprod_{x \in A} B
  \]
  Dar, de fapt, nu este neapărat necesar ca $ A $ să fie mulțime, deci structura nu este 
  neapărat produsul cartezian din teoria mulțimilor. Sub interpretarea logică, putem scrie:
  \[
    A \land B = \exists_{x \in A} B,
  \]
  un caz particular al cuantificatorului existențial, cînd $ B $ este o formulă care nu
  depinde de presupunerea $ x $.

  Ne putem pune problema existenței coproduselor $ \ds\coprod_{x \in A} B(x) $, unde
  $ B(x) $ să depindă de $ x $. În categorii carteziene, însă, acest lucru este
  imposibil, deoarece categoria polinomială $ \kal{C}[x] $ are aceleași obiecte cu $ \kal{C} $.
\end{remark}


\begin{remark}\label{rk:slice}
  În locul categoriei polinomiale, se poate face trecerea la \emph{categoria slice},
  conform \cite{joyal} și \cite{macc}, \S II.6 (unde sînt numite \qq{categorii-virgulă},
  eng.\ \emph{comma categories}).
  Astfel, în loc de elementul distins $ x \in A $, cu care trecem la categoria
  polinomială, putem considera obiectul distins $ A $ și trecem la categoria slice
  (sau categoria deasupra lui $ A $), notată $ \kal{C}/A $. Obiectele categoriei
  sînt săgeți $ \ast \to A $, iar săgețile categoriei sînt triunghiuri comutative
  de forma:
  \[
    \xymatrix{
    B \ar[dr] \ar[rr] & & C \ar[dl] \\ & A &
  }
  \]
  
  Acestea au meritul de a evidenția mai clar decît categoriile polinomiale existența sau 
  lipsa structurilor suplimentare (e.g.\ egalizator sau exponențială).
\end{remark}

Vom înzestra ulterior o categorie carteziană cu o exponențială,
conducînd astfel la categorii cartezian închise, ce vor fi legate
strîns de $ \lambda $-calculul cu tipuri simple, dar deocamdată
mai detaliem o construcție elementară.

Fie $ \kal{C} $ o categorie carteziană. Introducem \emph{egalizatorul}
a două săgeți paralele, $ f, g : A \to B $, conform \cite{macc}, \S III.4.
Intuitiv, el este o generalizare a nucleului unui morfism de structuri.
Fie obiectul egalizator $ E(f, g) $, care vine cu un monomorfism canonic
în $ A $, ce poate fi privit ca o incluziune. Putem scrie în limbajul
intern (adică folosind variabile):
\[
  E(f, g) = \{ x \in A \mid fx = gx \},
\]
iar dacă $ A = 1, $ atunci săgeata $ x : 1 \to A $ coincide cu $ 1_A $, 
deci egalizatorul devine $ E(f, g) = \{ x \in 1 \mid f = g \} $. 
Aici $ x $ nu mai joacă niciun rol, deci $ E(f, g) $ este \qq{propoziția} care
afirmă că $ f = g $.

%%%%%%%%%%%%%%%%%%%%%%%%%%%%%%%%%%%%%%%%%%%%%%%%%%%%%%%%%%%%%%%%%%%%%%
\section{Categorii cartezian închise}

\indent\indent Putem înzestra o categorie carteziană cu o structură
suplimentară, numită \emph{exponențială}, care va facilita legătura cu
$\lambda$-calcul și cu programarea funcțională. Prezentarea curentă urmează
\cite{awodey} și \cite{mact}.

Începem cu o motivație și prezentare intuitivă, care va face legătura
și cu programarea funcțională.

O funcție de două variabile $ f : A \times B \to C, f = f(x, y) $ poate
fi privită ca o funcție de o singură variabilă, fixînd pe $ x $,
anume $ f = f(y) \in C^B $. Acum, dacă variem pe $ x $, obținem o 
familie indexată, care ne dă o asociere:
\[
  x \mapsto f_x(y) = f(x, y) : A \to C^B.
\]

Orice asemenea funcție poate fi definită astfel și este complet determinată
de această prezentare. Deci, intuitiv, avem un izomorfism:
\[
  \dr{Hom}(A, C^B) \simeq \dr{Hom}(A \times B, C), \quad f_x(y) = f(x, y).
\]

Putem defini o funcție de tip evaluare, pentru a abstractiza:
\[
  \dr{eval} : C^B \times B \to C, \quad %
  \dr{eval}(g, b) = g(b), \forall g \in C^B, b \in B.
\]

Această funcție are următoarea proprietate de universalitate: dată mulțimea
$ A $ și funcția $ f : A \times B \to C $, există o unică funcție
$ \widetilde{f} : A \to C^B $ astfel încît:
\[
  \dr{eval}(\widetilde{f} \times 1_B) = f \Leftrightarrow %
  \dr{eval}(\widetilde{f}(a), b) = f(a, b),
\]
adică avem diagrama comutativă:
\[
  \xymatrix{
  C^B \times B \ar[r]^-{\dr{eval}} & C \\
  A \times B \ar@{.>}[u]^{\exists! \widetilde{f} \times 1_B} \ar[ur]_-f &
}
\]

Proprietatea poate fi folosită în orice categorie cu produse binare.

\begin{definition}\label{def:exp}
  Fie $ \kal{C} $ o categorie cu produse binare. O \emph{exponențială} a
  obiectelor $ B $ și $ C $ din $ \kal{C} $ este un obiect $ C^B $ și o
  săgeată $\varepsilon : C^B \times B \to C $ cu proprietatea de
  universalitate de mai sus: pentru orice $ Z \in \kal{C} $ și
  $ f : Z \times B \to C $, există o unică săgeată $ \widetilde f : Z \to C^B $
  care să facă diagrama comutativă:
  \[
    \xymatrix{
      C^B \\
      Z \ar[u]^-{\widetilde{f}}
    } \qquad\qquad
    \xymatrix{
    C^B \ar[r]^-\varepsilon & C \\
    Z \times B \ar@{.>}[u]^-{\widetilde{f} \times 1_B} \ar[ur]_-f &
  }
  \]
  adică $ \varepsilon \circ (\widetilde{f} \times 1_B) = f $.

  În acest context, $ \varepsilon $ se numește \emph{evaluare}, iar
  $ \widetilde{f} $ se numește \emph{transpusa} lui $ f $.
\end{definition}

Dată transpusa, putem recupera funcția. Presupunem că pornim cu $ g : Z \to C^B $
și definim:
\[
  \overline{g} = \varepsilon(g \times 1_B) : Z \times B \to C.
\]

Rezultă că avem izomorfismul:
\begin{align*}
  \dr{Hom}(Z \times B, C) &\simeq \dr{Hom}(Z, C^B) \\
  f &\mapsto \widetilde{f} \\
  \overline{g} &\mapsfrom g
\end{align*}

\begin{remark}\label{rk:eval}
  În informatică, transpusa poate fi asociată cu $ \lambda(-) $, adică
  $\widetilde{f} = \lambda f $, iar în limbajele funcționale, săgeata eval
  se asociază, de obicei, cu \texttt{apply}, $ \lambda g $ fiind \texttt{curry(g)}.
\end{remark}


Putem da acum o definiție ecuațională a categoriilor cartezian închise,
care să faciliteze legătura cu $ \lambda $-calculul cu tipuri simple.

\begin{definition}\label{def:ccc}
  O categorie $ \kal{C} $ este \emph{cartezian închisă} dacă are următoarea
  structură:
  \begin{enumerate}[(1)]
    \item Un obiect distins $ 1 $ și pentru orice obiect $ C \in \kal{C} $, o săgeată
      $ !_C : C \to 1 $ astfel încît, pentru orice altă săgeată $ f : C \to 1 $,
      să avem $ f = !_C $;
    \item Pentru orice pereche de obiecte $ A, B $ există un obiect $ A \times B $
      și săgețile $ p_1 : A \times B \to A, p_2 : A \times B \to B $, iar pentru
      orice săgeți $ f : Z \to A $ și $ g : Z \to B $ există o săgeată:
      \[
          \langle f, g \rangle : Z \to A \times B
      \]
      astfel încît:
      \begin{align*}
        p_1 \langle f, g \rangle &= f \\
        p_2 \langle f, g \rangle &= g \\
        \langle p_1 h, p_2 h \rangle &= h, \forall h : Z \to A \times B;
      \end{align*}
    \item Pentru orice pereche de obiecte $ A \times B $, există un obiect
    $ B^A $ și o săgeată $ \varepsilon : B^A \times A \to B $, iar pentru orice
    săgeată $ f : Z \times A \to B $ există o săgeată $ \widetilde{f} : Z \to B^A $
    astfel încît:
    \[
      \varepsilon \circ (\widetilde{f} \times 1_A) = f, \quad %
      \varepsilon \circ (g \times 1_A) = \widetilde{g},
    \]
    pentru orice $ g : Z \to B^A $, unde:
    \[
      g \times 1_A = \langle gp_1, p_2 \rangle : Z \times A \to B^A \times A.
    \]
  \end{enumerate}
\end{definition}


Legătura cu $ \lambda $-calculul cu tipuri poate fi făcută evidentă prin
prezentarea ecuațională a acesteia din urmă, în forma următoare. O
teorie de $\lambda$-calcul cu tipuri se alcătuiește din:
\begin{itemize}
  \item tipuri de bază: $ A, B, \dots, A \times B, A \to B $ etc.;
  \item termeni, care pot fi:
    \begin{itemize}
      \item variabile de tip $ A $;
      \item constante;
      \item perechi $ \langle a, b \rangle : A \times B $, cu $ a : A, b : B $;
      \item proiecții prime: $ \dr{fst}(c) : A$, cu $ c : A \times B $;
      \item proiecții secunde: $ \dr{snd}(c) : B$, cu $ c : A \times B $;
      \item evaluări: $ca : B$, unde $ c : A \times B, a : A $;
      \item abstracții: $ \lambda x . b : A \to B$ ,cu $ x : A, b \in B $
    \end{itemize}
    \item ecuații:
      \begin{align*}
        \dr{fst}(\langle a, b \rangle) &= a \\
        \dr{snd}(\langle a, b \rangle) &= b \\
        \langle \dr{fst}(c), \dr{snd}(c)\rangle &= c \\
        (\lambda x . b) a &= b[a/x] \\
        \lambda x . cx &= c, \text{ cu x liberă în c}
      \end{align*}
\end{itemize}


Astfel prezentată, teoriei de $ \lambda $-calcul i se poate asocia
o \emph{categorie a tipurilor}, unde obiectele sînt tipurile simple
(dintre tipurile de bază, $ A, B, \dots $), săgețile sînt date de
tipurile de forma $ A \to B $, identitățile $ 1_A = \lambda x. x$,
pentru $ x : A $, iar compunerea $ c \circ b = \lambda x.c(bx) $.

Aceasta definește corect o categorie, care are produse binare și
este chiar cartezian închisă. Date două obiecte $ A $ și $ B $
din categorie, definim $ B^A = A \to B $, iar evaluarea:
\[
  \varepsilon = \lambda z . \dr{fst}(z)\dr{snd}(z) : B^A \times A \to B (z : Z).
\]
Atunci, pentru orice săgeată $ f : Z \times A \to B $, putem defini
transpusa prin:
\[
  \widetilde{f} = \lambda z \lambda x . f\langle z, x \rangle : %
  Z \to B^A (z : Z, x : A).
\]

%%%%%%%%%%%%%%%%%%%%%%%%%%%%%%%%%%%%%%%%%%%%%%%%%%%%%%%%%%%%%%%%
\section{Plan}

\indent\indent În ce privește continuarea cercetării, fiecare dintre
componentele corespondenței Curry--Howard--Lambek merită studiată
individual, împreună cu toate variantele sale. 

Totodată, dezvoltarea istorică și filosofică a subiectului o considerăm 
a fi de o importanță deosebită. Așa cum se arată în \cite{collins},
încercările de eliminare a paradoxurilor de tip Russell, prin 
ramificări, ierarhizări și tipizări ale teoriilor s-au întins în multe 
domenii și au primit, la rîndul lor, tratamente diferite. Astfel,
în afară de Russell, Ramsey, Hilbert, Turing, Church, actori principali
ai acestei dezvoltări au fost și Wittgenstein, care a formulat o tipizare
bazată pe teoria funcțiilor, Frege, cu a sa \qq{teorie a treptelor} și
Carnap, care a considerat că locul cel mai potrivit al tipizării este
la nivel de limbaj (formal).

Precum menționează Shulman în \cite{sh}, o altă componentă filosofică
importantă a teoriilor care alcătuiesc corespondența Curry--Howard--Lambek
este aceea că \qq{spațiile} (i.e.\ tipurile) utilizate au o natură sintetică.
Aceastea sînt construite în acord cu intuiția: pentru a le înțelege, trebuie
să știm cum arată obiectele lor și cum ne dăm seama dacă avem de-a face
cu obiecte diferite sau cu același obiect. Această abordare urmează varianta
lui Martin-L\"{o}f pentru regulile de introducere și eliminare ale lui Gentzen.

În partea logică a subiectului, logica intuiționistă și teoria tipurilor
aferentă prezintă un interes deosebit. Ne referim în special la teoria
lui Martin-L\"{o}f (cf.\ \cite{pml}), care introduce tipul egalitate,
nevid atunci cînd se poate demonstra că două tipuri sînt \qq{egale}. Exact
echivocul lăsat de această noțiune de egalitate a permis dezvoltarea
\emph{teoriei de omotopie a tipurilor}, în care relația de egalitate este
relaxată și se preia relația de echivalență omotopică din topologie algebrică
(mai general, algebră omologică). Această noțiune de \qq{egalitate} a deschis
calea către o descriere axiomatică, i.e.\ \emph{axioma de univalență} a lui
Voevodsky și interpretarea ei structuralistă din partea lui Awodey (cf.,
de exemplu, \cite{awos}).

În ce privește aspectele categoriale, sîntem interesați de trecerea către
toposuri, structurile \qq{naturale} pentru a studia fundamente matematice,
în care se poate formula semantica mai multori logici (e.g.\ intuiționistă
și clasică, în special). În aceste structuri, algebrele Heyting joacă un rol
fundamental, deoarece ele ajută la trecerea dinspre logica booleană către
logica intuiționistă. Detalii sînt conținute în \cite{gold}. Totodată, muncă
de pionierat în traducerea logicilor în teoria categoriilor a fost realizată
de Lawvere, începînd cu \cite{law} și continuînd cu Seely în \cite{see}.

Pe lîngă aceste fațete specializate, monografia \cite{ch} reprezintă o lectură
obligatorie pentru aprofundarea le\-gă\-tu\-ri\-lor între conceptele de bază ale
acestei corespondențe de interes. 

\vspace{3cm}

%%%%%%%%%%%%%%%%%%%%%%%%%%%%%%%%%

\bibliography{chl-survey.bib}
\bibliographystyle{apalike}

    




\end{document}
