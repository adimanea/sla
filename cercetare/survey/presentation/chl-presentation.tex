\documentclass[xcolor=dvipsnames]{beamer}
\usepackage{mathpazo}
\usepackage{amsmath}
\usepackage{amsfonts}
\usepackage{amssymb, bbding}
\usepackage{graphicx}
\usepackage[utf8]{inputenc}
\usepackage[T1]{fontenc}
\usepackage{ae,aecompl}
\usepackage[all]{xy} 
\usepackage{lmodern}
\usepackage[romanian]{babel}
\renewcommand{\figurename}{} 
\usepackage{caption}
\captionsetup{compatibility=false}
\usepackage{subcaption}
\usepackage[style=german]{csquotes}
\usepackage{ebproof} % for logic proof trees

\newcommand\ds{\displaystyle}
\newcommand\seq{\subseteq}
\newcommand\speq{\supseteq}
\newcommand\rar{\rightarrow}
\newcommand\Rar{\Rightarrow}

\newtheoremstyle{mystyle}{0mm}{0mm}{}{}{\itshape}{}{ }{}
\theoremstyle{mystyle}
\newtheorem{bl}{}

\newcommand{\bloc}[3]{\begin{bl}<#1->{{\large\color{Gray}{\hrulefill}}\\ \color{firebrick}{\large \emph{#2}}}\\ \vspace*{-2mm}{\color{Gray}{\hrulefill}}\\ #3 \end{bl}} 

\newcommand{\fr}[1]{\frame{#1}}
\newcommand{\ft}[1]{\frametitle{\color{firebrick}{\hfill #1 \hfill}}}
\newcommand{\lin}[3]{\uncover<#1->{\alert<#1>{#2}}{\vspace*{#3 ex}}}
\newcommand{\ite}[2]{\uncover<#1->{\alert<#1>{\item #2}}}
\newcommand{\vs}[1]{\vspace*{#1 ex}}
\definecolor{bleaumarin}{RGB}{30,30,150} 
\definecolor{firebrick}{RGB}{178,34,34}

\useoutertheme{shadow} 
\usetheme{CambridgeUS} 
\usecolortheme[named=firebrick]{structure} 
\useoutertheme[compress]{smoothbars}

\setbeamertemplate{items}[ball]
\setbeamertemplate{blocks}[rounded][shadow=true]
\setbeamertemplate{navigation symbols}{}
% \setbeamercovered{still covered={\opaqueness<1->{15}},again covered={\opaqueness<1->{15}}}
\setbeamertemplate{headline}{}  

\title[Curry-Howard-Lambek]{Corespondența între propoziții, tipuri și categorii}
\author[Adrian Manea]{Adrian Manea \\ \vspace{.3cm} 
                      \small{supervizor: Conf.\ Dr.\ Denisa Diaconescu}}
\institute[410]{SLA, anul 1, grupa 410}

\date{}

\begin{document}

\maketitle

%\fr{%
  %\ft{Motivație}
  %\lin{1}{Teorii computaționale: Turing, categorii, logica}{1}

  %\begin{itemize}
    %\ite{2}{\textbf{Turing}: aspecte filosofice, determinism, finitudine, \enquote{umanizarea}
    %calculatorului;}
    %\ite{3}{\textbf{Categorii}: de la ``abstract nonsense'' la modele computaționale,
    %modelarea proceselor neuronale, lingvistică;}
    %\ite{4}{\textbf{Logica}: teorii cu egalitate, raționamente privite drept \enquote{calcule};}
  %\end{itemize}
  %\vspace{.3cm}
  %\lin{5}{Abordarea multifațetată a acelorași rezultate nu poate fi decît benefică!}{2}
  %\lin{6}{\enquote{propoziții (văzute) ca tipuri}, \enquote{demonstrații (văzute) ca programe},
  %\enquote{simplificarea demonstrațiilor (văzută) ca evaluarea programelor}}{1}
%}

\fr{%
    \ft{Subiectele de interes}

    \lin{1}{\textbf{Informatică teoretică}: programare funcțională, programe scrise
      ca teorii și demonstrații, teoria de omotopie a tipurilor;}{1}

    \lin{2}{\textbf{Logică}: logica intuiționistă, interpretarea BHK,
    teoria intuiționistă a tipurilor (Per Martin-L\"{o}f);}{1}
   
    \lin{3}{\textbf{Categorii}: CCC, $ \infty $-grupoizi, algebre Heyting, 
    fascicule, toposuri;}{2}

    \lin{4}{\textbf{Filosofie}: structuralism (axioma univalenței, Voevodsky-Awodey), 
      \enquote{spații sintetice} (Shulman), ierarhii (tipuri: trepte [Frege], 
    funcții [Wittgenstein], limbaj [Carnap]);}{1}
}

\fr{%
  \ft{Logică: Deducția naturală (Gentzen)}

  \lin{1}{Reguli de deducție în perechi (introducere--eliminare):}{2}

  \lin{2}{%
    \[%
    \begin{prooftree}
        \hypo{A \quad B}
        \infer1[\&I]{A \& B}
    \end{prooftree} \qquad %
    \begin{prooftree}
      \hypo{A \& B} \infer1[\&E1]{A}
    \end{prooftree} \qquad %
    \begin{prooftree}
      \hypo{A \& B} \infer1[\&E2]{B}
    \end{prooftree};
    \]
}{2}

\lin{3}{%
  \[
    \begin{prooftree}
      \hypo{[A]^x} \ellipsis{}{B} \infer1[\supset I^x]{A \supset B}
    \end{prooftree} \qquad %
    \begin{prooftree}
      \hypo{A \supset B \quad A} \infer1[\supset E]{B}
    \end{prooftree}.
\]
}{2}

}

\fr{%
  \ft{Logică: Deducția naturală (Gentzen)}

  \lin{1}{Reguli de rescriere (simplificare):}{2}
  
  \lin{2}{%
    \[
  \begin{prooftree}
    \hypo{\stackrel{\vdots}{A} \quad \stackrel{\vdots}{B}}
    \infer1[\& I]{A \& B}
    \infer2[\&E1]{A}
  \end{prooftree} \Longrightarrow
  \begin{prooftree}
    \ellipsis{}{A}
  \end{prooftree} \qquad \qquad
  \begin{prooftree}
    \hypo{[A]^x} \ellipsis{}{B}
    \infer1[\supset I^x]{A \supset B} \hypo{\stackrel{\vdots}{A}}
    \infer2[\supset E]{B}
  \end{prooftree} \Longrightarrow
  \begin{prooftree}
    \ellipsis{}{A} \ellipsis{}{B}
  \end{prooftree}
\]}{3}
}

\fr{%
  \ft{$\lambda$-calculul cu tipuri simple}

  \lin{1}{Reguli de tipizare:}{0}
  \lin{2}{\begin{itemize}
      \ite{2}{introducere = tipizare/abstracție;}
      \ite{3}{eliminare = aplicare/proiecție.}
  \end{itemize}
}{2}

  \lin{4}{%
    \[
    \begin{prooftree}
      \hypo{M : A \quad N : B}
      \infer1[\times I]{\langle M, N \rangle : A \times B}
    \end{prooftree} \qquad
    \begin{prooftree}
      \hypo{[x : A]^x} \ellipsis{}{N : B}
      \infer1[\to I^x]{\lambda x . N : A \to B}
    \end{prooftree}
  \]
}{2}

\lin{5}{\[\pi_1(L : A \times B) : A, \qquad \pi_2(L : A \times B) : B\]}{1}
\lin{6}{%
  \[
    \begin{prooftree}
      \hypo{L : A \to B \quad M : A}
      \infer1[\to E]{LM : B}
    \end{prooftree}.
  \]
}{2}
}

\fr{%
  \ft{$\lambda$-calculul cu tipuri simple}

    \lin{1}{Reguli de simplificare (evaluare):}{0}
    
    \lin{2}{%
      \[
  \begin{prooftree}
    \hypo{\stackrel{\vdots}{M : A} \quad \stackrel{\vdots}{N : B}}
    \infer1[\times I]{\langle M, N \rangle : A \times B}
    \infer2[\times E1]{\pi_1\langle M, N \rangle : A}
  \end{prooftree} \Longrightarrow
  \begin{prooftree}
    \ellipsis{}{M : A}
  \end{prooftree}
\]}{2}

\lin{3}{%
  \[
  \begin{prooftree}
    \hypo{[x : A]^x} \ellipsis{}{N : B}
    \infer1[\to I^x]{\lambda x . N : A \to B} \hypo{\stackrel{\vdots}{M : A}}
    \infer2[\to E]{(\lambda x.N)M : B}
  \end{prooftree} \Longrightarrow
  \begin{prooftree}
    \ellipsis{}{M : A}
  \ellipsis{}{N[M/x] : B}
  \end{prooftree}
\]}{2}
}

\fr{%
  \ft{Categorii cartezian închise}

  \lin{1}{O categorie $ \mathcal{C} $ se numește \emph{cartezian închisă} dacă are:}{0}
  \lin{2}{\begin{itemize}
      \ite{2}{Un obiect distins $1 \in \mathcal{C} $ și o săgeată $ !_C : C \to 1 $ unică;}
      \ite{3}{Pentru orice $ A, B $, un obiect $ A \times B $ și săgeți $ p_1 : A \times B \to A $
      și $ p_2 : A \times B \to B $ cu \emph{proprietate de universalitate};}
      \ite{4}{Pentru orice $ A \times B $, un obiect $ B^A $ (exponențială, $\lambda(-)$) și
      $ \varepsilon : B^A \times A \to B $ (\texttt{eval(-)});}
      \ite{5}{+ diagrame comutative (compatibilitate și universalitate).}
\end{itemize}}{4}
  }


\fr{%
  \ft{$\lambda$-calculul cu tipuri simple --- Prezentare ecuațională}

  \lin{1}{Tipuri de bază: $ A, B, \dots, A \times B, A \to B$ etc.;}{2}

  \lin{2}{Termeni:}{0}
  
  \lin{3}{\begin{itemize}
      \ite{3}{Variabile de tip $ A $;}
      \ite{4}{Constante;}
      \ite{5}{Perechi $ \langle a, b \rangle : A \times B \quad (a : A, b : B) $;}
      \ite{6}{Proiecții $ \mathrm{fst}(c) : A, \mathrm{snd}(c) : B \quad (c : A \times B) $;}
      \ite{7}{Evaluări $ ca : B \quad (c : A \times B, a : A) $;}
      \ite{8}{Abstracții $ \lambda x . b : A \times B \quad (x : A, b : B) $;}
\end{itemize}}{4}
}

\fr{%
  \ft{$\lambda$-calculul cu tipuri simple --- Prezentare ecuațională}

  \lin{1}{Ecuații:}{0}
  \lin{2}{\begin{itemize}
      \ite{2}{$\mathrm{fst}(\langle a, b \rangle) = a, \quad %
      \mathrm{snd}(\langle a, b \rangle) = b $;}
      \ite{3}{$\langle \mathrm{fst}(c), \mathrm{snd}(c) \rangle = c \quad (c : A \times B) $;}
      \ite{4}{$(\lambda x . b)a = b [a/x] $;}
      \ite{5}{$\lambda x . cx = c $, cu $ x $ liberă în $ c $.}
  \end{itemize}
}{3}

\lin{6}{$\leadsto$ \emph{categoria tipurilor $ \lambda $-calculului}:}{0}
\lin{7}{\begin{itemize}
    \ite{7}{Obiecte: tipurile simple (dintre tipurile de bază, $ A, B, C, \dots $);}
    \ite{8}{Săgeți: tipuri de forma $ A \times B $;}
    \ite{9}{Identități $ 1_A = \lambda x . x \quad (x : A) $;}
    \ite{10}{Compunerea: $ c \circ b = \lambda x . c(bx) $}
\end{itemize}
}{4}

}

\fr{%
  \ft{Plan}
  
  \lin{1}{\textbf{Tipuri (ierarhii)}: introducere logică și filosofică (Frege, Russell, %
  Ramsey, Wittgenstein, Carnap);}{1}

  \lin{2}{\textbf{Logica intuiționistă} (Brouwer, Heyting) și teoria intuiționistă a tipurilor %
    (Martin-L\"{o}f);}{1}

    \lin{3}{Tipuri de egalitate (ML) $\leadsto$ \textbf{teorii de omotopie (HoTT)};}{1}

    \lin{4}{Axioma de \textbf{univalență} (Voevodsky, HoTT): \enquote{echivalența este %
    echivalentă cu egalitatea}}{1}
    
    \lin{5}{\textbf{Toposuri} via fascicule (Mac Lane) sau teorii semantice (Lawvere), 
    categorii ca sisteme deductive (Lambek).}{0}
}


\fr{%
  \ft{Bibliografie orientativă}

  \vspace{1cm}

  \lin{1}{\begin{itemize}
      \ite{1}{Awodey, S.\ --- \emph{Structuralism, Invariance, and Univalence} (2014);}
      \ite{1}{Goldblatt, R.\ --- \emph{Topoi: The Categorial Analysis of Logic} (1984);}
      \ite{1}{Lambek, J., Scott, P.\ J.\ --- \emph{Introduction to Higher Order Categorical %
      Logic} (1988);}
      \ite{1}{Lawvere, F.\ --- \emph{Functorial Semantics of Algebraic Theories} (1963);}
      \ite{1}{Martin-L\"of, P. --- \emph{Intuitionistic Type Theory} (1980);}
      \ite{1}{Shulman, M.\ --- \emph{Homotopy Type Theory: The Logic of Space} (2017);}
      \ite{1}{S{\o}rensen, M.\, Urzyczyn, P.\ --- \emph{Lectures on the Curry-Howard %
      isomorphism} (2006);}
      \ite{1}{Wadler, P.\ --- \emph{Propositions as Types} (2015).}
  \end{itemize}
    }{8}
  }


\end{document}

