% ! TEX root = ../curbe.tex
\chapter{Curbe algebrice}

Printr-o \emph{curbă algebrică} vom înțelege o varietate proiectivă de
dimensiune 1. Vom lucra, în general, cu curbe netede.

Notațiile specifice sînt:
\begin{itemize}
    \item $ C/K $: curba $ C $ este definită peste corpul $ K $;
    \item $ \overline{K}(C) $: corpul de funcții al lui $ C $ peste $ K $;
    \item $ \overline{K}[C]_P $: inelul local al lui $ C $ în punctul $ P $;
    \item $ M_P $: idealul maximal al inelului local $ \overline{K}[C]_P $.
\end{itemize}

\begin{definition}\label{def:valuare-norm}
    \index{valuare!normalizată}
    \index{uniformizator}
    Fie $ C $ o curbă și $ P \in C $ un punct neted.

    \emph{Valuarea normalizată} pe $ \overline{K}[C]_P $ este dată de:
    \begin{align*}
        \dr{ord}_P : \overline{K}[C]_P &\to \NN \cup \{ \infty \} \\
        \dr{ord}_P(f) &= \sup \{ d \in \ZZ \mid f \in M_P^d \}.
    \end{align*}

    Folosind $ \dr{ord}_P(f/g) = \dr{ord}_P(f) - \dr{ord}_P(g) $, putem
    extinde valuarea la $ \ZZ $.

    Un \emph{uniformizator} pentru $ C $ în $ P $ este orice funcție
    $ t \in \overline{K}(C) $, cu $ \dr{ord}_P t = 1 $, adică
    $ t $ este generator pentru dealul $ M_P $.
\end{definition}

Pentru $ C $ și $ P $ ca mai sus, fie $ f \in \overline{K}(C) $.
Se definește \emph{ordinul} lui $ f $ în $ P $ prin $ \dr{ord}_P f $.
Dacă ordinul este pozitiv, spunem că $ f $ are zero în $ P $; altfel,
$ f $ are singularitate (pol) în $ P $. Dacă ordinul este pozitiv,
spunem că $ C $ este \emph{definită} în $ P $ și putem calcula $ f(P) $.
Altfel, $ f(P) = \infty $.

\begin{example}\label{exm:ordin}
    Reluăm unul dintre exemplele anterioare (exemplul \ref{exm:dim2}):
    \[
        C_1: Y^2 = X^3 + X, \quad C_2: Y^2 = X^3 + X^2.
    \]
    Ambele curbe au cîte o singularitate la infinit. Fie $ P = (0, 0) $.
    Atunci $ C_1 $ este netedă în $ P $, dar $ C_2 $ nu este.

    Idealul maximal $ M_P $ al lui $ \overline{K}[C_1]_P $ are proprietatea că
    $ M_P/M_P^2 $ este generat de $ Y $, deci:
    \[
        \dr{ord}_P Y = 1, \quad \dr{ord}_P X = 2, \quad \dr{ord}_P(2Y^2 - X) = 2,
    \]
    ultima egalitate rezultînd din $ 2Y^2 - X = 2X^3 + X $.
\end{example}

%%%%%%%%%%%%%%%%%%%%%%%%%%%%%%%%%%%%%%%%%%%%%%%%%%%%%%%%%%%%%%%%%%%%%%

\section{Divizori}

\begin{definition}\label{def:divizor}
    \index{curbe!divizori}
    Fie $ C $ o curbă algebrică. \emph{Grupul divizorilor} curbei, notat
    $ \dr{Div}C $, este grupul abelian liber ($ \ZZ $-modulul) generat de
    punctele de pe $ C $.

    Deci orice $ D \in \dr{Div}C $ este o sumă formală:
    \[
        D = \sum_{P \in C} n_P(P),
    \]
    unde $ n_P \in \ZZ $ și $ n_P = 0 $ pentru majoritatea $ P \in C $.

    \emph{Gradul divizorului} $ D $ se definește prin:
    \[
        \dr{deg} D = \sum_{P \in C} n_P.
    \]
\end{definition}

Divizorii de grad 0 formează un subgrup al $ \dr{Div}(C) $, pe care
îl notăm și definim astfel:
\[
    \dr{Div}^0(C) = \{ D \in \dr{Div}(C) \mid \dr{deg} D = 0 \}.
\]

Presupunem acum că avem o curbă netedă $ C $ și fie $ f \in \overline{K}(C)^\times $.
Atunci putem asocia un divizor fiecărei funcții $ f $ prin:
\[
    \dr{div}(f) = \sum_{P \in C} \dr{ord}_P(f)(P).
\]
În plus, cum fiecare $ \dr{ord}_P $ definește o valuare, aplicația de mai
sus se poate extinde la:
\[
    \dr{div} : \overline{K}(C)^\times \to \dr{Div}(C),
\]
care devine un morfism de grupuri abeliene. Acțiunea lui este similară aplicației
care trimite un element dintr-un corp de numere în idealul care-l conține.
De aceea, noțiunea se poate generaliza și sîntem conduși la definițiile de mai jos.

\begin{definition}\label{def:divizori-pr}
    \index{divizor!principal}
    \index{divizor!echivalență liniară}
    \index{divizor!grupul Picard}
    Un divizor $ D \in \dr{Div}(C) $ se numește \emph{principal} dacă este de
    forma $ D = \dr{div}f $, pentru un anumit $ f $ ca mai sus.

    Doi divizori se numesc \emph{echivalenți liniar}, notat $ D_1 \sim D_2 $,
    dacă $ D_1 - D_2 $ este un divizor principal.

    \emph{Grupul Picard} (sau grupul claselor de divizori) pentru curba $ C $,
    notat $ \dr{Pic}(C) $, este grupul factor al $ \dr{Div}(C) $ modulo
    subgrupul divizorilor principali.
\end{definition}

Cu aceste noțiuni, se poate demonstra simplu:
\begin{proposition}\label{prop:deg-div}
    Fie $ C $ o curbă netedă și fie $ f \in \overline{K}(C)^\times $.
    \begin{itemize}
        \item $ \dr{div}f = 0 $ dacă și numai dacă $ f \in \overline{K}^\ast $;
        \item $ \dr{deg}(div f) = 0 $.
    \end{itemize}
\end{proposition}

\begin{example}\label{exm:div-princ}
    În $ \PP^1 $, fiecare divizor de grad 0 este principal. Într-adevăr, fie
    $ D = \sum n_P(P) $ un divizor de grad 0. Atunci fie $ P = [\alpha_P, \beta_P] \in \PP^1 $
    și rezultă că $ D $ este divizor pentru funcția:
    \[
        \prod_{P \in \PP^1} (\beta_P X - \alpha_P Y)^{n_P}.
    \]
    Dar $ \sum n_P = 0 $, ceea ce asigură că această funcție aparține $ K(\PP^1) $
    și rezultă că aplicația $ \dr{deg} : \dr{Pic}(\PP^1) \to \ZZ $ este un izomorfism.

    Reciproca este de asemenea adevărată, i.e.\ dacă $ C $ este o curbă netedă
    iar $ \dr{Pic}(C) \simeq \ZZ $, atunci $ C \simeq \PP^1 $.
\end{example}

\begin{example}\label{exm:div-princ2}
    Presupunem că lucrăm într-un corp cu $ \dr{char}K \neq 2 $ și fie 
    $ e_1, e_2, e_3 \in \overline{K} $ trei elemente distincte. Definim curba:
    \[
        C: y^2 = (x - e_1)(x - e_2)(x - e_3).
    \]
    Se poate verifica ușor că această curbă este netedă și că are un singur punct
    la infinit, pe care îl notăm $ P_\infty $. Pentru $ i = 1, 2, 3 $, fie $ P_i = (e_i, 0) \in C $.
    Atunci obținem:
    \begin{align*}
        \dr{div}(x - e_i) &= 2P_i - 2P_\infty \\
        \dr{div}(y) &= P_1 + P_2 + P_3 - 3P_\infty.
    \end{align*}
\end{example}

Rezultă din cele de mai sus că grupul divizorilor principali este un subgrup
al $ \dr{Div}^0 C $. Definim partea de grad 0 a grupului Picard pentru curba $ C $
ca fiind grupul factor al $ \dr{Div}^0 C $ modulo subgrupul divizorilor principali.
Notăm acest grup cu $ \dr{Pic}^0 C $. 

Observațiile de mai sus pot fi puse în această formă: există un șir exact scurt:
\[
    1 \to \overline{K}^\times \to \overline{K}(C)^\times \xrar{\dr{div}}%
    \to \dr{Div}^0 C \to \dr{Pic}^0 C \to 0.
\]

%%%%%%%%%%%%%%%%%%%%%%%%%%%%%%%%%%%%%%%%%%%%%%%%%%%%%%%%%%%%%%%%%%%%%%
\section{Teorema Riemann-Roch}

Fie $ C $ o curbă. Definim o ordine parțială pe grupul divizorilor:
\begin{definition}\label{def:ordine-div}
    \index{divizor!efectiv}
    Un divizor $ D = \sum n_P(P) $ se numește \emph{efectiv} sau pozitiv,
    notat $ D \geq 0 $, dacă $ n_P \geq 0 $ pentru orice punct $ P \in C $.

    Mai departe, pentru orice divizori $ D_1, D_2 \in \dr{Div}(C) $,
    putem scrie $ D_1 \geq D_2 $ dacă diferența $ D_1 - D_2 $ este un divizor efectiv.
\end{definition}

\begin{example}\label{exm:div-efectiv}
    Fie $ f \in \overline{K}(C)^\times $ o funcție care este regulată peste tot,
    mai puțin într-un punct $ P \in C $ și presupunem că acolo are un pol de ordin
    cel mult $ n $ în $ P $. Aceste cerințe pot fi puse mai simplu în forma:
    \[
        \dr{div}f \geq -n(P).
    \]
    Similar, dacă scriem $ \dr{div}f \geq (Q) - n(P) $, afirmăm că $ f $ are un
    zero în $ Q $.
\end{example}

\begin{definition}\label{def:functii-div}
    Fie $ D \in \dr{Div}C $. Asociem divizorului $ D $ o mulțime de funcții:
    \[
        \kal{L}(D) = \{ f \in \overline{K}(C)^\times \mid \dr{div}f \geq -D \} \cup \{0\}.
    \]
    Mulțimea $ \kal{L}(D) $ formează un spațiu vectorial finit dimensional peste
    $ \overline{K} $ și notăm $ \ell(D) = \dim_{\overline{K}}\kal{L}(D) $.
\end{definition}

\begin{theorem}[Riemann-Roch]\label{thm:riemann-roch}
    \index{teorema!Riemann-Roch}
    \index{curbe!gen}
    Fie $ C $ o curbă netedă și fie $ K_C $ un divizor canonic pe $ C $.
    Atunci există un întreg $ g \geq 0 $, numit \emph{genul curbei} $ C $,
    astfel încît pentru orice divizor $ D \in \dr{Div}(C) $, are loc:
    \[
        \ell(D) - \ell(K_C - D) = \dr{deg}D - g + 1.
    \]
\end{theorem}


%%% Local Variables:
%%% mode: latex
%%% TeX-master: "../curbe"
%%% End:
