% ! TEX root = ../curbe.tex
\chapter{Curbe algebrice}

Printr-o \emph{curbă algebrică} vom înțelege o varietate proiectivă de
dimensiune 1. Vom lucra, în general, cu curbe netede.

Notațiile specifice sînt:
\begin{itemize}
\item $ C/K $: curba $ C $ este definită peste corpul $ K $;
\item $ \overline{K}(C) $: corpul de funcții al lui $ C $ peste $ K $;
\item $ \overline{K}[C]_P $: inelul local al lui $ C $ în punctul $ P $;
\item $ M_P $: idealul maximal al inelului local $ \overline{K}[C]_P $.
\end{itemize}

\begin{definition}\label{def:valuare-norm}
  \index{valuare!normalizată}
  \index{uniformizator}
  Fie $ C $ o curbă și $ P \in C $ un punct neted.

  \emph{Valuarea normalizată} pe $ \overline{K}[C]_P $ este dată de:
  \begin{align*}
    \dr{ord}_P : \overline{K}[C]_P &\to \NN \cup \{ \infty \} \\
    \dr{ord}_P(f) &= \sup \{ d \in \ZZ \mid f \in M_P^d \}.
  \end{align*}

  Folosind $ \dr{ord}_P(f/g) = \dr{ord}_P(f) - \dr{ord}_P(g) $, putem
  extinde valuarea la $ \ZZ $.

  Un \emph{uniformizator} pentru $ C $ în $ P $ este orice funcție
  $ t \in \overline{K}(C) $, cu $ \dr{ord}_P t = 1 $, adică
  $ t $ este generator pentru dealul $ M_P $.
\end{definition}

Pentru $ C $ și $ P $ ca mai sus, fie $ f \in \overline{K}(C) $.
Se definește \emph{ordinul} lui $ f $ în $ P $ prin $ \dr{ord}_P f $.
Dacă ordinul este pozitiv, spunem că $ f $ are zero în $ P $; altfel,
$ f $ are singularitate (pol) în $ P $. Dacă ordinul este pozitiv,
spunem că $ C $ este \emph{definită} în $ P $ și putem calcula $ f(P) $.
Altfel, $ f(P) = \infty $.

\begin{example}\label{exm:ordin}
  Reluăm unul dintre exemplele anterioare (exemplul \ref{exm:dim2}):
  \[
    C_1: Y^2 = X^3 + X, \quad C_2: Y^2 = X^3 + X^2.
  \]
  Ambele curbe au cîte o singularitate la infinit. Fie $ P = (0, 0) $.
  Atunci $ C_1 $ este netedă în $ P $, dar $ C_2 $ nu este.

  Idealul maximal $ M_P $ al lui $ \overline{K}[C_1]_P $ are proprietatea că
  $ M_P/M_P^2 $ este generat de $ Y $, deci:
  \[
    \dr{ord}_P Y = 1, \quad \dr{ord}_P X = 2, \quad \dr{ord}_P(2Y^2 - X) = 2,
  \]
  ultima egalitate rezultînd din $ 2Y^2 - X = 2X^3 + X $.
\end{example}

%%%%%%%%%%%%%%%%%%%%%%%%%%%%%%%%%%%%%%%%%%%%%%%%%%%%%%%%%%%%%%%%%%%%%%

\section{Divizori}

\begin{definition}\label{def:divizor}
  \index{curbe!divizori}
  Fie $ C $ o curbă algebrică. \emph{Grupul divizorilor} curbei, notat
  $ \dr{Div}C $, este grupul abelian liber ($ \ZZ $-modulul) generat de
  punctele de pe $ C $.

  Deci orice $ D \in \dr{Div}C $ este o sumă formală:
  \[
    D = \sum_{P \in C} n_P(P),
  \]
  unde $ n_P \in \ZZ $ și $ n_P = 0 $ pentru majoritatea $ P \in C $.

  \emph{Gradul divizorului} $ D $ se definește prin:
  \[
    \dr{deg} D = \sum_{P \in C} n_P.
  \]
\end{definition}

%%% Local Variables:
%%% mode: latex
%%% TeX-master: "../curbe"
%%% End:
