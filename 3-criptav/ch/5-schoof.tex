% ! TEX root = ../curbe.tex

\chapter{Algoritmul lui Schoof}

\todo[inline,noline,backgroundcolor=green!40]{clarifică!}

Există o abordare algoritmică pentru a număra punctele unei curbe
eliptice definită peste un corp finit. Știm din teorema lui Hasse
(teorema \ref{thm:hasse}) că:
\[
    \# E(\FF_q) = q + 1 - a_1, \quad |a_q| \leq 2 \sqrt{q}.
\]
Pentru aplicații criptografice, însă, este util să avem o metodă
eficientă de a calcula numărul de puncte din $ E(\FF_q) $.

Pentru simplitate, vom presupune că lucrăm cu $ q $ impar și
că $ E $ este dată de ecuația Weierstrass de forma:
\[
    E: y^2 = f(x) = 4x^3 + b_2x^2 + 2b_4x + b_6,
\]
pentru care mare parte din rezultatele folosite vor fi valabile
și în caracteristică 2, cu mici modificări.

Există o metodă directă, dar deloc simplă, de a calcula numărul de
puncte, care folosește simboluri Legendre:
\[
    a_q = \sum_{x \in \FF_q} \left( \dfrac{f(x)}{q} \right),
\]
dar fiecare simbol Legendre se calculează folosind reciprocitatea
pătratică în $ O(\log q) $ pași, deci în total avem $ O(q \log q) $
pași, adică un algoritm exponențial.

În continuare, descriem un algoritm care calculează $ \# E(\FF_q) $ în
timp polinomial, i.e.\ $ O(\log^c q) $, cu $ c $ fixat, independent de $ q $.
Ideea acestui algoritm este să se calculeze $ a_q \text{ mod } \ell $
pentru prime mici $ \ell $ și apoi să se folosească lema chineză a resturilor
pentru a recompune $ a_q $.

Fie aplicația:
\[
    \tau : E(\overline{\FF_q}) \to E(\overline{\FF_q}), \quad (x, y) \mapsto (x^q, y^q),
\]
aplicația Frobenius de putere $ q $, deci știm că are loc:
\[
    \tau^2 - a_q \tau + q = 0
\]
în $ \dr{End}(E) $. În particular, pentru $ P \in E(\FF_q)[\ell] $, are loc:
\[
    \tau^2(P) - [a_q]\tau(P) + [q]P = O,
\]
deci dacă punem $ P = (x, y) $ și presupunem $ P \neq O $, avem:
\[
    (x^{q^2}, y^{q^2}) - [a_q](x^q, y^q) + [q](x, y) = O.
\]
Deoarece am presupus că $ P = (x, y) $ are ordinul $ \ell $, rezultă:
\[
    [a_q](x^q, y^q) = [n_\ell](x^q, y^q),
\]
pentru un $ n_\ell \equiv a_q \text{ mod } \ell $ și $ 0 \leq n_\ell < \ell $.

Similar, putem calcula $ [q](x, y) $ prin a reduce $ q $ modulo $ \ell $ mai
întîi.

Nu trebuie să știm exact valoarea lui $ n_\ell $, deci pentru orice întreg
între 0 și $ \ell $ calculăm $ [n](x^q, y^q) $ pentru orice punct
$ (x, y) \in E[\ell] - \{ O \} $ și verificăm dacă satisface:
\[
    [n](x^q, y^q) = (x^{q^2}, y^{q^2}) + [q](x, y).
\]

Problema care apare este că punctele din $ E[\ell] $ sînt definite
peste extinderi destul de mari ale lui $ \FF_q $, deci va trebui
să lucrăm cu toate punctele de $ \ell $-torsiune simultan. Pentru aceasta,
folosim polinomul $ \psi_\ell(x) \in \FF_q[x] $, ale cărui rădăcini
sînt coordonatele $ x $ ale punctelor nenule de $ \ell $-torsiune
din $ E $ (presupunem, pentru simplitate, $ \ell \neq 2 $). Acest
polinom are gradul $ \dfrac{1}{2}(\ell^2 - 1) $ și se poate calcula
simplu ({\color{red}\textbf{v.\ Ex. 3.7, pagina 105}}). Acum putem lucra
în inelul factor:
\[
    R_\ell = \dfrac{\FF_q[x, y]}{\psi_\ell(x), y^2 - f(x)}.
\]

Rezultă că, dacă avem o putere neliniară a lui $ y $, putem înlocui
$ y^2 $ cu $ f(x) $ și dacă avem o putere $ x^d $, mai mare decît
$ \dfrac{1}{2}(\ell^2 - 1) $, putem împărți la $ \psi_\ell(x) $ și luăm
doar restul. Astfel, nu lucrăm niciodată cu polinoame de grad mai mare
decît $ \dfrac{1}{2}(\ell^2 - 3) $.

Scopul va fi să calculăm $ a_q \text{ mod } \ell $ pentru suficiente
prime $ \ell $ și apoi să găsim $ a_q $. Teorema lui Hasse (\ref{thm:hasse})
ne dă $ |a_q| \leq 2 \sqrt{q} $, deci este suficient să luăm primele
$ \ell \leq \ell_{\max} $ astfel încît:
\[
    \prod_{\ell \leq \ell_{\max}} \ell \geq 4 \sqrt{q}.
\]

\begin{theorem}[Algoritmul Schoof]\label{thm:schoof}
    Fie $ E/\FF_q $ o curbă eliptică. Algoritmul descris la \ref{alg:schoof} este
    unul în timp polinomial pentru a calcula $ \# E(\FF_q) $. Mai precis,
    calculează $ \# E(\FF_q) $ în $ O(\log^8 q) $ pași.
\end{theorem}

\begin{algorithm}
  \caption{Algoritmul lui Schoof}
  \begin{algorithmic}[1]
    \Procedure{Schoof}{$q, a$}\Comment{returnează $\# E(\FF_q)$}
      \State $ A \gets 1 $
      \State $ \ell \gets 3 $
      \While{$ A < 4 \sqrt{q} $}
        \While{$n = 0, 1, 2, \dots, \ell - 1 $}
          \If{$(x^{q^2}, y^{q^2}) + [q](x, y) = [n](x^q, y^q)$}
            \texttt{break}
          \EndIf
        \EndWhile
        \State $ A \gets \ell \cdot A $
        \State $ n_\ell = n $
        \State $ \ell \gets $ următorul prim $ \ell $
      \EndWhile
      \State Lema Chineză $ \Rightarrow a \equiv n_\ell \text{ mod } \ell, \forall n_\ell $
      \State \textbf{returnează} $ \# E(\FF_q) = q + 1 - a $
    \EndProcedure
  \end{algorithmic}
  \label{alg:schoof}
\end{algorithm}

\begin{proof}
    Arătăm că timpul de rulare pentru algoritmul Schoof este $ O(\log^8 q) $.

    Mai întîi:
    \begin{enumerate}[(a)]
        \item Cel mai mare număr prim $ \ell $ folosit în algoritm are
            proprietatea $ \ell \leq O(\log q) $:

            Teorema de distribuție a numerelor prime poate fi rescrisă în forma:
            \[
                \lim_{x \to \infty} \dfrac{1}{x} %
                \sum_{\stackrel{\ell \leq x}{l\text{ prim }}} \log \ell = 1.
            \]
            Rezultă $ \ds\prod_{\ell < x} \ell \simeq e^x $, deci pentru a face
            ca produsul să fie mai mare decît $ 4 \sqrt{q} $, este suficient
            să luăm $ x \simeq \dfrac{1}{2} \log(16 q) $.
        \item Înmulțirea în inelul $ R_\ell $ se poate face în $ O(\ell^4 \log^2 q) $ operații
            pe biți:

            Elementele inelului $ R_\ell $ sînt polinoame de grad $ O(\ell^2) $.
            Înmulțirea între două astfel de polinoame și reducerea modulo
            $ \psi_\ell(x) $ consumă $ O(\ell^4) $ operații elementare
            (adunări și înmulțiri) în corpul $ \FF_q $. Similar, înmulțirea
            în $ \FF_q $ consumă $ O(\log^2 q) $ operați pe biți.

            Rezultă că operațiile de bază în $ R_\ell $ consumă $ O(\ell^4 \log^2 q) $
            operații pe biți.

        \item Sînt necesare $ O(\log q) $ operații în inelul $ R_\ell $ pentru
            a reduce $ x^q, y^q, x^{q^2} $ și $ y^{q^2} $ în inelul $ R_\ell $:


            În general, sînt necesare $ O(\log n) $ operații pentru a
            calcula puterile $ x^n $ și $ y^n $ în $ R_\ell $.
            Dar aceste operații sînt făcute o singură dată, iar apoi putem stoca
            punctele de forma:
            \[
                (x^{q^2}, y^{q^2}) + [q \text{ mod } \ell] (x, y) \quad \text{ și } \quad (x^q, y^q)
            \]
            pe care apoi le folosim în pasul 4 al algoritmului Schoof.
    \end{itemize}

    Folosind operațiile elementare de mai sus, putem estima timpul de rulare pentru
    algoritmul Schoof. Din (a), obținem că avem nevoie doar de $ \ell $ prime care sînt
    mai mici decît $ O(log q) $ și cum există $ O\left( \dfrac{\log q}{\log \log q} \right) $
    asemenea prime, rezultă că liniile 4-12 din algoritmul lui Schoof
    se execută de atîtea ori. Apoi, de fiecare dată cînd se intră în bucla
    controlată de $ A $, se execută bucla controlată de $ n $ (liniile
    5-8) de $ \ell = O(\log q) $ ori.

    Mai departe, cum $ \ell = O(\log q) $, din afirmația (b) de mai sus,
    rezultă că operațiile de bază din $ R_\ell $ durează $ O(\log^6 q) $
    operații pe biți. Valoarea $ [n](x^q, y^q) $ din linia 6 a algoritmului
    se poate calcula în $ O(1) $ operații în $ R_\ell $, știind valoarea
    anterioară $ [n-1](x^q, y^q) $.

    Rezultă că numărul total de pași este:
    \[
        \underbrace{O(\log q)}_{\text{bucla A}} \cdot \underbrace{O(\log q)}_{\text{bucla n}} \cdot %
        \underbrace{O(\log^6 q)}_{\text{operații pe biți}} = O(\log^8 q) \text{ operații pe biți}.
    \]
    Am demonstrat, deci, că algoritmul lui Schoof calculează $ \# E(\FF_q) $ în timp polinomial.
\end{proof}

Remarcăm că cele mai costisitoare etape sînt calculele în inelul $ R_\ell $, care este o
extindere a lui $ \FF_q $, de grad $ 2 \ell^2 $. Așadar, deși marginea pentru $ \ell $
este liniară în $ \log q $, pentru valori mari ale lui $ q $, și marginea pentru $ \ell $
și dimensiunea inelului $ R_\ell $ peste $ \FF_q $ sînt mari.

\textbf{Exemplu:} Fie $ q \simeq 2^{256} $, o valoare utilizată în practică în aplicații
criptografice. Rezultă:
\[
    \prod_{\ell \leq 103} \ell \simeq 2^{133} > 4 \sqrt{q} = 2^{130},
\]
deci cel mai mare prim $ \ell $ utilizat de algoritmul lui Schoof este $ \ell = 103 $.

Rezultă că un element din $ V = \FF_q[x]/\psi_\ell(x) $ este reprezentat de un
$ \FF_q $-vector de mărime $ 103^2 \simeq 2^{13} $, iar fiecare element al
$ \FF_q $ este un număr pe 256 biți. Așadar, elementele din $ V $ ocupă
aproximativ $ 2^{22} $ biți, adică mai mult de 16 kB. Mărimea nu este nerezonabilă
pentru computerele moderne, totuși calcule intensive în inele ale căror elemente se
stochează pe 16 kB durează considerabil.

\todo[inline,noline,backgroundcolor=green!40]{exemple + de luat din alte părți explicații}
\todo[inline,noline,backgroundcolor=green!40]{vezi și \href{https://en.wikipedia.org/wiki/Schoof\%27s\_algorithm}{wiki}}




%%% Local Variables:
%%% mode: latex
%%% TeX-master: "../curbe"
%%% End:
