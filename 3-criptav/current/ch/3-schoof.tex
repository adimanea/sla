% ! TEX root = ../pce.tex

\chapter{Algoritmul lui Schoof}

\todo[inline,noline,backgroundcolor=green!40]{de clarificat pașii}
\todo[inline,noline,backgroundcolor=green!40]{de uniformizat descrierea}

Există o abordare algoritmică pentru a număra punctele unei curbe
eliptice definită peste un corp finit. Știm din teorema lui Hasse
(teorema \ref{thm:hasse}) că:
\[
    \# E(\FF_q) = q + 1 - a_q, \quad |a_q| \leq 2 \sqrt{q}.
\]
Pentru aplicații criptografice, însă, este util să avem o metodă
eficientă de a calcula numărul de puncte din $ E(\FF_q) $.

Pentru simplitate, vom presupune că lucrăm cu $ q $ impar și
că $ E $ este dată de o ecuație Weierstrass de forma:
\[
    E: y^2 = f(x) = 4x^3 + b_2x^2 + 2b_4x + b_6,
\]
pentru care mare parte din rezultatele folosite vor fi valabile
și în caracteristică 2, cu mici modificări.

Există o metodă directă, dar deloc simplă, de a calcula numărul de
puncte, care folosește simboluri Legendre:
\[
    a_q = \sum_{x \in \FF_q} \left( \dfrac{f(x)}{q} \right),
\]
dar fiecare simbol Legendre se calculează folosind reciprocitatea
pătratică în $ O(\log q) $ pași, deci în total avem $ O(q \log q) $
pași, adică un algoritm exponențial.

În continuare, descriem un algoritm care calculează $ \# E(\FF_q) $ în
timp polinomial, i.e.\ $ O(\log^c q) $, cu $ c $ fixat, independent de $ q $.
Ideea acestui algoritm este să se calculeze $ a_q \text{ mod } \ell $
pentru prime mici $ \ell $ și apoi să se folosească lema chineză a resturilor
pentru a recompune $ a_q $.

Fie aplicația:
\[
    \tau : E(\overline{\FF}_q) \to E(\overline{\FF}_q), \quad (x, y) \mapsto (x^q, y^q),
\]
aplicația Frobenius de putere $ q $, extinsă la curba definită peste închiderea
algebrică a lui $ \FF_q $, deci știm că are loc:
\[
    \tau^2 - a_q \tau + q = 0
\]
în $ \dr{End}(E) $, conform și demonstrației teoremei \ref{thm:hasse}.
Se definește subgrupul de $ \ell $-torsiune al curbei eliptice, pentru
un prim impar fixat $ \ell $:
\[
  E(\FF_q)[\ell] = \left\{ P \in E(\overline{\FF}_q) \mid lP = O \right\}.
\]
În particular, pentru $ P \in E(\FF_q)[\ell] $, are loc:
\[
    \tau^2(P) - [a_q]\tau(P) + [q]P = O,
\]
deci dacă punem $ P = (x, y) $ și presupunem $ P \neq O $, avem:
\[
    (x^{q^2}, y^{q^2}) - [a_q](x^q, y^q) + [q](x, y) = O.
\]
Deoarece am presupus că $ P = (x, y) $ are ordinul $ \ell $, rezultă:
\[
    [a_q](x^q, y^q) = [n_\ell](x^q, y^q),
\]
pentru un $ n_\ell \equiv a_q \text{ mod } \ell $ și $ 0 \leq n_\ell < \ell $.

Similar, putem calcula $ [q](x, y) $ prin a reduce $ q $ modulo $ \ell $ mai
întîi.

Nu trebuie să știm exact valoarea lui $ n_\ell $, deci pentru orice întreg
între 0 și $ \ell $ calculăm $ [n](x^q, y^q) $ pentru orice punct
$ (x, y) \in E[\ell] - \{ O \} $ și verificăm dacă satisface:
\[
    [n](x^q, y^q) = (x^{q^2}, y^{q^2}) + [q](x, y).
\]

Problema care apare este că punctele din $ E[\ell] $ sînt definite
peste extinderi destul de mari ale lui $ \FF_q $, deci va trebui
să lucrăm cu toate punctele de $ \ell $-torsiune simultan. Pentru aceasta,
folosim polinomul $ \psi_\ell(x) \in \FF_q[x] $, ale cărui rădăcini
sînt coordonatele $ x $ ale punctelor nenule de $ \ell $-torsiune
din $ E $ (presupunem, pentru simplitate, $ \ell \neq 2 $). Acest
polinom are gradul $ \dfrac{1}{2}(\ell^2 - 1) $ și îl descriem
pe scurt mai jos.

\begin{definition}\label{def:polinom-div}
  Mulțimea polinoamelor de diviziune (eng.\ \emph{division polynomials})
  din $ \ZZ[x, y, A, B] $ se definește recursiv prin:
  \begin{align*}
    \psi_0 &= 0 \\
    \psi_1 &= 1 \\
    \psi_2 &= 2y \\
    \psi_3 &= 3x^4 + 6Ax^2 + 12Bx - A^2 \\
    \psi_4 &= 4y\left(x^6 + 5Ax^4 + 20Bx^3 - 5A^2x^2 - 4ABx - 8B^2 - A^3\right) \\
           &\vdots \\
    \psi_{2m + 1} &= \psi_{m + 2} \psi^3 m - \psi_{m-1}\psi_{m+1}^3, m \geq 2 \\
    \psi_{2m} &= \left( \dfrac{\psi_m}{2y} \right) \cdot 5
                \left( \psi_{m+2}\psi_{m-1}^2 - \psi_{m-2} \psi_{m+1}^2 \right), m \geq 3
  \end{align*}
\end{definition}

Legătura cu curbele eliptice este realizată astfel. Presupunem că lucrăm
cu o curbă eliptică de ecuație $ y^2 = x^3 + Ax + B $. Atunci polinoamele
de diviziune definite ca mai sus au proprietatea $ \psi_{2m+1} \in \ZZ[x, A, B] $
și $ \psi_{2m} \in 2y \ZZ[x, A, B] $ și în cazul nostru vom avea,
suplimentar, $ A, B \in \FF_q $, deoarece curba este definită peste acest corp.

În plus, rădăcinile polinoamelor $ \psi_{2n+1} $ sînt coordonatele $ x $
ale punctelor $ E(\FF_q)[2n+1] - \{ O \} $, subgrupul de $ 2n + 1 $-torsiune,
iar similar, rădăcinile polinoamelor $ \dfrac{\psi_{2n}}{y} $ sînt
coordonatele $ x $ ale punctelor din $ E(\FF_q)[2n] - E(\FF_q)[2] $.

Încă mai mult, $ E(\overline{\FF}_q) \simeq \ZZ/\ell \times \ZZ/\ell $
dacă $ \ell \neq p $ și în rest, avem izomorfism cu $ \ZZ/\ell $ sau $ \{ 0 \} $
dacă $ \ell = p $. Rezultă că, după cazuri, gradul polinomului $ \psi_\ell $
este $ \dfrac{1}{2}(\ell^2 - 1) $, $ \dfrac{1}{2} (\ell - 1) $ sau $ 0 $.

Revenind la algoritmul lui Schoof, acum putem lucra în inelul factor:
\[
    R_\ell = \dfrac{\FF_q[x, y]}{\psi_\ell(x), y^2 - f(x)}.
\]

Rezultă că, dacă avem o putere neliniară a lui $ y $, putem înlocui
$ y^2 $ cu $ f(x) $ și dacă avem o putere $ x^d $, mai mare decît
$ \dfrac{1}{2}(\ell^2 - 1) $, putem împărți la $ \psi_\ell(x) $ și luăm
doar restul. Astfel, nu lucrăm niciodată cu polinoame de grad mai mare
decît $ \dfrac{1}{2}(\ell^2 - 3) $.

Scopul va fi să calculăm $ a_q \text{ mod } \ell $ pentru suficiente
prime $ \ell $ și apoi să găsim $ a_q $. Teorema lui Hasse (\ref{thm:hasse})
ne dă $ |a_q| \leq 2 \sqrt{q} $, deci este suficient să luăm primele
$ \ell \leq \ell_{\max} $ astfel încît:
\[
    \prod_{\ell \leq \ell_{\max}} \ell \geq 4 \sqrt{q}.
\]

\begin{theorem}[Algoritmul Schoof]\label{thm:schoof}
  Fie $ E/\FF_q $ o curbă eliptică. Algoritmul descris la figura \ref{alg:schoof} este
  unul în timp polinomial pentru a calcula $ \# E(\FF_q) $. Mai precis,
  calculează $ \# E(\FF_q) $ în $ O(\log^8 q) $ pași.
\end{theorem}

\begin{algorithm}
  \caption{Algoritmul lui Schoof}
  \begin{algorithmic}[1]
    \Procedure{Schoof}{$q, a$}\Comment{returnează $\# E(\FF_q)$}
      \State $ A \gets 1 $
      \State $ \ell \gets 3 $
      \While{$ A < 4 \sqrt{q} $}
        \While{$n = 0, 1, 2, \dots, \ell - 1 $}
          \If{$(x^{q^2}, y^{q^2}) + [q](x, y) = [n](x^q, y^q)$}
            \texttt{ieși}
          \EndIf
        \EndWhile
        \State $ A \gets \ell \cdot A $
        \State $ n_\ell = n $
        \State $ \ell \gets $ următorul prim $ \ell $
      \EndWhile
      \State Lema Chineză $ \Rightarrow a \equiv n_\ell \text{ mod } \ell, \forall n_\ell $
      \State \textbf{returnează} $ \# E(\FF_q) = q + 1 - a $
    \EndProcedure
  \end{algorithmic}
  \label{alg:schoof}
\end{algorithm}

\begin{proof}
    Arătăm că timpul de rulare pentru algoritmul Schoof este $ O(\log^8 q) $.

    Mai întîi:
    \begin{enumerate}[(a)]
        \item Cel mai mare număr prim $ \ell $ folosit în algoritm are
            proprietatea $ \ell \leq O(\log q) $:

            Teorema de distribuție a numerelor prime poate fi rescrisă în forma:
            \[
                \lim_{x \to \infty} \dfrac{1}{x} %
                \sum_{\stackrel{\ell \leq x}{l\text{ prim }}} \log \ell = 1.
            \]
            Rezultă $ \ds\prod_{\ell < x} \ell \simeq e^x $, deci pentru a face
            ca produsul să fie mai mare decît $ 4 \sqrt{q} $, este suficient
            să luăm $ x \simeq \dfrac{1}{2} \log(16 q) $.
        \item Înmulțirea în inelul $ R_\ell $ se poate face în $ O(\ell^4 \log^2 q) $ operații
            pe biți:

            Elementele inelului $ R_\ell $ sînt polinoame de grad $ O(\ell^2) $.
            Înmulțirea între două astfel de polinoame și reducerea modulo
            $ \psi_\ell(x) $ consumă $ O(\ell^4) $ operații elementare
            (adunări și înmulțiri) în corpul $ \FF_q $. Similar, înmulțirea
            în $ \FF_q $ consumă $ O(\log^2 q) $ operați pe biți.

            Rezultă că operațiile de bază în $ R_\ell $ consumă $ O(\ell^4 \log^2 q) $
            operații pe biți.

        \item Sînt necesare $ O(\log q) $ operații în inelul $ R_\ell $ pentru
            a reduce $ x^q, y^q, x^{q^2} $ și $ y^{q^2} $ în inelul $ R_\ell $:


            În general, sînt necesare $ O(\log n) $ operații pentru a
            calcula puterile $ x^n $ și $ y^n $ în $ R_\ell $.
            Dar aceste operații sînt făcute o singură dată, iar apoi putem stoca
            punctele de forma:
            \[
                (x^{q^2}, y^{q^2}) + [q \text{ mod } \ell] (x, y) \quad \text{ și } \quad (x^q, y^q)
            \]
            pe care apoi le folosim în pasul 4 al algoritmului Schoof.
    \end{enumerate}

    Folosind operațiile elementare de mai sus, putem estima timpul de rulare pentru
    algoritmul Schoof. Din (a), obținem că avem nevoie doar de $ \ell $ prime care sînt
    mai mici decît $ O(log q) $ și cum există $ O\left( \dfrac{\log q}{\log \log q} \right) $
    asemenea prime, rezultă că liniile 4-12 din algoritmul lui Schoof
    se execută de atîtea ori. Apoi, de fiecare dată cînd se intră în bucla
    controlată de $ A $, se execută bucla controlată de $ n $ (liniile
    5-8) de $ \ell = O(\log q) $ ori.

    Mai departe, cum $ \ell = O(\log q) $, din afirmația (b) de mai sus,
    rezultă că operațiile de bază din $ R_\ell $ durează $ O(\log^6 q) $
    operații pe biți. Valoarea $ [n](x^q, y^q) $ din linia 6 a algoritmului
    se poate calcula în $ O(1) $ operații în $ R_\ell $, știind valoarea
    anterioară $ [n-1](x^q, y^q) $.

    Rezultă că numărul total de pași este:
    \[
        \underbrace{O(\log q)}_{\text{bucla A}} \cdot \underbrace{O(\log q)}_{\text{bucla n}} \cdot %
        \underbrace{O(\log^6 q)}_{\text{operații pe biți}} = O(\log^8 q) \text{ operații pe biți}.
    \]
    Am demonstrat, deci, că algoritmul lui Schoof calculează $ \# E(\FF_q) $ în timp polinomial.
\end{proof}

Remarcăm că cele mai costisitoare etape sînt calculele în inelul $ R_\ell $, care este o
extindere a lui $ \FF_q $, de grad $ 2 \ell^2 $. Așadar, deși marginea pentru $ \ell $
este liniară în $ \log q $, pentru valori mari ale lui $ q $, și marginea pentru $ \ell $
și dimensiunea inelului $ R_\ell $ peste $ \FF_q $ sînt mari.

\begin{example}\label{exm:schoof}
  Fie $ q \simeq 2^{256} $, o valoare utilizată în practică în aplicații
  criptografice. Rezultă:
  \[
    \prod_{\ell \leq 103} \ell \simeq 2^{133} > 4 \sqrt{q} = 2^{130},
  \]
  deci cel mai mare prim $ \ell $ utilizat de algoritmul lui Schoof este $ \ell = 103 $.

  Rezultă că un element din $ V = \FF_q[x]/\psi_\ell(x) $ este reprezentat de un
  $ \FF_q $-vector de mărime $ 103^2 \simeq 2^{13} $, iar fiecare element al
  $ \FF_q $ este un număr pe 256 biți. Așadar, elementele din $ V $ ocupă
  aproximativ $ 2^{22} $ biți, adică mai mult de 16 kB. Mărimea nu este nerezonabilă
  pentru computerele moderne, totuși calcule intensive în inele ale căror elemente se
  stochează pe 16 kB durează considerabil.
\end{example}

Rezultatul lui Schoof a fost formulat în 1985 și este unul foarte important din punct
de vedere teoretic, deoarece toți algoritmii inițiali erau exponențiali, iar
cel mai bun de pînă atunci era varianta Baby Step, Giant Step, care are un timp
de rulare de asemenea exponențial.

Ulterior, în anii `90, s-au adus îmbunătățiri algoritmului lui Schoof, de
către Noam Elkies și A.\ O.\ Atkin, prin considerarea unor prime $ \ell $
de o anumită formă și folosind rezultate teoretice mult mai puternice,
precum formele modulare.

Varianta îmbunătățită rulează în aproximativ $ O(\log^6 q) $, cu presupunerea
că aproximativ jumătate dintre primele mai mici decît $ O(\log q) $ au
proprietatea dorită. Însă această presupunere este una euristică și nu
a fost demonstrată, nici măcar în cazul în care ipoteza lui Riemann ar fi
adevărată.


%%% Local Variables:
%%% mode: latex
%%% TeX-master: "../pce"
%%% End:
