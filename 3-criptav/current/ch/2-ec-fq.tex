% ! TEX root = ../pce.tex

\chapter{Numărul de puncte peste $ \FF_{\text{q}} $}

\section{Problema și abordarea naivă}

Fie acum $ E/\FF_q $ o curbă eliptică definită peste un corp finit.
Vrem să estimăm numărul punctelor din $ E(\FF_q) $, notat $ \# E(\FF_q) $,
adică una sau mai multe soluții ale ecuației Weierstrass scrisă în forma:
\[
  E : y^2 + a_1 xy + a_3 y = x^3 + a_2 x^2 + a_4 x + a_6, \quad %
  (x, y) \in \FF_q^2.
\]

Evident că valoarea lui $ x $ conduce la cel mult 2 valori pentru $ y $,
deci vom avea o margine superioară:
\[
  \# E(\FF_q) \leq 2q + 1.
\]
Dar o ecuație pătratică aleatorie are mici șanse să fie rezolvabilă
în $ \FF_q $, deci ne așteptăm ca marginea superioară să conțină mai curînd
$ q $, nu $ 2q $.

\begin{example}\label{exm:puncte}
  Să luăm un exemplu simplu mai întîi, pe care să-l rezolvăm manual.

  Fie $ E $ curba $ y^2 = x^3 + x + 1 $, definită peste $ \FF_5 $.

  Putem calcula direct astfel: luăm resturile lui $ x $ modulo 5, calculăm
  pătratele lor, precum și expresia $ x^3 + x + 1 $. Urmărind apoi primele
  două coloane ale tabelului de mai jos, putem găsit dacă există $ y $
  care să fie pătrat modulo 5, egal cu expresia corespunzătoare.

  \begin{center}
    \begin{tabular}{|c|c|c|c|c|}
      \hline
      $ x $ & $ x^2 $ & $ x^3 + x + 1 $ & $ y $ & Puncte \\
      \hline
      0 & 0 & 1 & 1, 4 & $ (0, 1), (0, 4) $ \\
      1 & 1 & 3 & $ \nexists $ & $ \nexists $ \\
      2 & 4 & 1 & 1, 4 & $ (2, 1), (2, 4) $ \\
      3 & 4 & 1 & 1, 4 & $ (3, 1), (3, 4) $ \\
      4 & 1 & 4 & 2, 3 & $ (4, 2), (4, 3) $ \\
                         \hline
    \end{tabular}
  \end{center}

  Rezultă $ \# E(\FF_5) = 9 $, adică cele 8 puncte din tabel și punctul de la
  infinit.

  În acest caz particular, am avut nevoie de $ O(5) $ pași, pentru că a
  trebuit să calculăm toți termenii din tabel, pentru orice $ x \in \FF_5 $.
\end{example}


%%%%%%%%%%%%%%%%%%%%%%%%%%%%%%%%%%%%%%%%%%%%%%%%%%%%%%%%%%%%%%%%%%%%%%

\section{Optimizarea 1: Teorema Hasse}

Calculul manual de mai sus nu este practic pentru valori mari ale lui $ q $.
Avem un rezultat teoretic care ne arată, însă, că nu este necesar să luăm
chiar toate valorile $ x \in \FF_q $, lucru care conduce la faptul că
marginea superioară a pașilor de calculat este ceva mai blîndă.

Vom avea nevoie de:
\begin{lemma}\label{le:deg-hasse}
  Fie $ A $ un grup abelian și $ d : A \to \ZZ $ o formă pătratică
  pozitiv definită. Atunci:
  \[
    | d(\psi - \phi) - d(\phi) - d(\psi)| \leq 2 \sqrt{d(\phi)d(\psi)}, %
    \quad \forall \psi, \phi \in A.
  \]
\end{lemma}

\begin{proof}
  Fie $ \psi, \phi \in A $. Folosim forma biliniară asociată formei
  pătratice $ d $:
  \[
    L(\psi, \phi) = d(\psi - \phi) - d(\phi) - d(\psi).
  \]

  Cum $ d $ este pozitiv definită, pentru orice $ m, n \in \ZZ $:
  \[
    0 \leq d(m\psi - n\phi) = m^2 d(\psi) + mnL(\psi, phi) + n^2 d(\phi).
  \]

  În particular, luăm
  \[
    m = -L(\psi, \phi), \quad n = 2 d(\psi)
  \]
  și se obține:
  \[
    0 \leq d(\psi)\left(4d(\psi)d(\phi) - L(\psi, \phi)^2\right).
  \]

  Pentru $ \psi \neq 0 $, rezultă inegalitatea noastră,
  iar pentru $ \psi = 0 $, inegalitatea este trivială.
\end{proof}

Rezultatul important de mai jos a fost formulat ca o conjectură de E.\ Artin
în 1924 și demonstrat de H.\ Hasse în 1933:
\begin{theorem}[Hasse]\label{thm:hasse}
  Fie $ E/\FF_q $ o curbă eliptică definită peste corpul finit $ \FF_q $.

  Atunci:
  \[
    \left| \# E(\FF_q) - q - 1 \right| \leq 2 \sqrt{q}.
  \]
\end{theorem}


%%% Local Variables:
%%% mode: latex
%%% TeX-master: "../pce"
%%% End:
