% ! TEX root = ../pce.tex

\chapter{Curbe eliptice peste corpuri finite}

Lucrăm acum într-un caz particular, acela al curbelor eliptice definite
peste corpuri finite $ \FF_q $. Notațiile pe care le fixăm sînt:
\begin{itemize}
\item $ q $, o putere a unui prim $ p $;
\item $ \FF_q $, un corp finit cu $ q $ elemente;
\item $ \overline{\FF}_q $, o închidere algebrică a lui $ \FF_q $.
\end{itemize}

Fie acum $ E/\FF_q $ o curbă eliptică definită peste un corp finit.
Vrem să estimăm numărul punctelor din $ E(\FF_q) $, notat $ \# E(\FF_q) $,
adică una sau mai multe soluții ale ecuației Weierstrass scrisă în forma:
\[
  E : y^2 + a_1 xy + a_3 y = x^3 + a_2 x^2 + a_4 x + a_6, \quad %
  (x, y) \in \FF_q^2.
\]

Evident că valoarea lui $ x $ conduce la cel mult 2 valori pentru $ y $,
deci vom avea o margine superioară:
\[
  \# E(\FF_q) \leq 2q + 1.
\]
Dar o ecuație pătratică aleatorie are mici șanse să fie rezolvabilă
în $ \FF_q $, deci ne așteptăm ca marginea superioară să conțină mai curînd
$ q $, nu $ 2q $.

Rezultatul important de mai jos a fost formulat ca o conjectură de E.\ Artin
în 1924 și demonstrat de H.\ Hasse în 1933:
\begin{theorem}[Hasse]\label{thm:hasse}
  Fie $ E/\FF_q $ o curbă eliptică definită peste corpul finit $ \FF_q $.

  Atunci:
  \[
    \left| \# E(\FF_q) - q - 1 \right| \leq 2 \sqrt{q}.
  \]
\end{theorem}


%%% Local Variables:
%%% mode: latex
%%% TeX-master: "../pce"
%%% End:
