% ! TEX root = ../pce.tex

\chapter*{Introducere}

\indent\indent Lucrarea de față urmărește să prezinte o introducere într-un subiect
foarte important în domeniul criptării pe curbe eliptice. Este vorba
despre un algoritm de numărarea punctelor de pe curbe eliptice,
algoritm care poartă numele lui Ren\'e Schoof, cel care l-a introdus
în anul 1985 în \cite{Schoof1985EllipticCO} și publicat în forma 
rafinată în articolul \cite{JTNB_1995__7_1_219_0}.

Aparatul matematic utilizat se bazează pe o teoremă care mărginește
superior numărul posibil de puncte, rezultat care a apărut mai întîi
în lucrările lui Emil Artin în 1924 și demonstrat de Helmut Hasse
în 1933 (teorema \ref{thm:hasse}).

Lucrarea conține cîteva generalități pentru introducere în studiul
curbelor eliptice și familiarizarea cu noțiunile fundamentale.
Aceste generalități sînt conținute în secțiunea \ref{ch:curbe}
și folosesc totodată și drept preliminarii pentru abordarea
matematică ce urmează.

În partea a doua, conținînd a doua secțiune, introducem problema
numărului de puncte ale unei curbe eliptice peste corpul $ \FF_q $
și prezentăm două rezultate care ajută atît la facilitarea punerii
problemei, cît și la rezolvarea acesteia. Este vorba despre
teorema lui Hasse (teorema \ref{thm:hasse}) și algoritmul
Baby Step, Giant Step pentru calculul logaritmului discret
(Algoritmul \ref{alg:bsgs} pe cazul general, respectiv Algoritmul
\ref{alg:bsgs-fq} pentru cazul particular peste $ \FF_q $).

În fine, ultima parte a lucrării conține algoritmul lui Schoof
(Algoritmul \ref{alg:schoof}), împreună cu introducerea matematică
ce îl motivează și o demonstrație a convergenței sale, împreună
cu un calcul grosier al complexității-timp.

Lucrarea se încheie cu un exemplu de aplicare și cîteva observații
privitoare la optimizările ulterioare ale algoritmului Schoof.

%%% Local Variables:
%%% mode: latex
%%% TeX-master: "../pce"
%%% End:
