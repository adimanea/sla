\chapter{Alte aspecte și idei}

\section{Auditarea modificărilor cu \texttt{pgaudit}}
\index{audit (\texttt{pgaudit})}

În continuare, putem să activăm funcția de auditare din extensia
\texttt{pgaudit}, care va afișa într-un log modificările făcute la baza de
date, filtrate după tipul modificării. De exemplu, se pot înregistra în
log operații de tipul:
\begin{itemize}
\item \texttt{READ}, adică atunci cînd se fac operații de citire de tipul
  \texttt{SELECT, COPY} sau un query;
\item \texttt{WRITE}, cînd se introduc, se modifică sau se șterg valori ori
  alte operații care schimbă conținutul înregistrărilor;
\item \texttt{FUNCTION}, atunci cînd se execută funcții sau blocuri \texttt{DO};
\item \texttt{ROLE}, pentru cînd se atribuie, șterg sau modifică roluri și
  utilizatori noi;
\item \texttt{MISC}, pentru alte tipuri de operații
  (\texttt{DISCARD, VACUUM, FETCH} etc.).
\end{itemize}

Să creăm un tabel care să înregistreze datele pentru audit cînd se citesc valori:
{
  \small
\begin{verbatim}
# create table pgAuditX( id serial, continut text );
# insert into pgAuditX(id, continut) values(DEFAULT, 'test audit');
\end{verbatim}
}

Acum trebuie să activăm extensia de audit care să urmărească operațiile de tip
\texttt{READ}:
{
  \small
\begin{verbatim}
# alter system set pgaudit.log to 'read';
# select pg_reload_conf();
# -- test:
# select continut from pgAuditX;
\end{verbatim}
}

Acum verificăm în terminal, în afara bazei de date:
{
  \small
\begin{verbatim}
$ grep AUDIT postgresql-Sat.log | grep READ

2020-01-18 postgres postgres LOG: 
AUDIT: SESSION,1,1,READ,SELECT,,,SELECT name FROM pgAuditX;,<none>
\end{verbatim}
}

%%%%%%%%%%%%%%%%%%%%%%%%%%%%%%%%%%%%%%%%%%%%%%%%%%%%%%%%%%%%%%%%%%%%%%
\section{Mascarea informațiilor cu \texttt{postgresql\_anonymizer}}
\index{securitate!mascare (anonimizare)}

Extensia \texttt{postgresql\_anonymizer} permite mascarea informațiilor
sensibile și înlocuirea lor cu unele aleatorii, astfel încît informațiile
inițiale să poată fi recuperate. Anonimizarea se face pe baza etichetelor
de securitate din PostgreSQL (\cite{psqlsecl}). Un exemplu de bază este:
{
  \small
\begin{verbatim}
# create extension if not exists anon cascade;
# select anon.load();
# create table player(id serial, name text, points int);
# security label for anon on column player.name 
# is 'masked with function anon.fake_last_name()';
# security label for anon on column player.id
# is 'masked with value null';
\end{verbatim}
}

Anonimizarea poate fi făcută \texttt{in-place}, fie folosind funcția
\texttt{anon.anymize\_database()}, care este \emph{destructivă} (i.e.\ înlocuiește
datele cu cele anonimizate):
{
  \small
\begin{verbatim}
# select * from customer;

 id  |   full_name      |   birth    |    employer   | zipcode | fk_shop
-----+------------------+------------+---------------+---------+---------
 911 | Chuck Norris     | 1940-03-10 | Texas Rangers | 75001   | 12
 112 | David Hasselhoff | 1952-07-17 | Baywatch      | 90001   | 423

# create extension if not exists anon cascade;
# select anon.load();
# security label for anon on column customer.full_name 
# is 'masked with function anon.fake_first_name() || '' '' || anon.fake_last_name()';
# security label for anon on column customer.birth   
# is 'masked with function anon.random_date_between(''01/01/1920''::DATE,now())';
# security label for anon on column customer.employer
# is 'masked with function anon.fake_company()';
# security label for anon on column customer.zipcode
# is 'masked with function anon.random_zip()';
# select anon.anonymize_database();
# select * FROM customer;

 id  |     full_name     |   birth    |     employer     | zipcode | fk_shop
-----+-------------------+------------+------------------+---------+---------
 911 | michel Duffus     | 1970-03-24 | Body Expressions | 63824   | 12
 112 | andromache Tulip  | 1921-03-24 | Dot Darcy        | 38199   | 423
\end{verbatim}
}

Sau putem doar să ascundem temporar informațiile, declarîndu-le cu
eticheta \texttt{MASKED}. Informațiile inițiale sînt:
{
  \small
\begin{verbatim}
# select * from people;
 id | fistname | lastname |   phone    
----+----------+----------+------------
 T1 | Sarah    | Conor    | 0609110911
(1 row)
\end{verbatim}
}

Pentru scopul nostru, mai întîi activăm extensia:
{
  \small
\begin{verbatim}
# create extension if not exists anon CASCADE;
# select anon.start_dynamic_masking();
\end{verbatim}
}

Declarăm un utilizator ce va fi mascat:
{
  \small
\begin{verbatim}
# create role skynet login;
# security label for anon on role skynet is 'masked';
\end{verbatim}
}

Apoi regulile de mascare:
{
  \small
\begin{verbatim}
# security label for anon on column people.lastname 
# is 'masked with function anon.fake_last_name()';
# security label for anon on column people.phone 
# is 'masked with function anon.partial(phone,2,$$******$$,2)';
\end{verbatim}
}

În fine, ne conectăm cu utilizatorul mascat și găsim informațiile ascunse:
{
  \small
\begin{verbatim}
# \! psql peopledb -U skynet -c 'select * from people;'
 id | fistname | lastname  |   phone    
----+----------+-----------+------------
 T1 | Sarah    | Stranahan | 06******11
(1 row)
\end{verbatim}
}


%%% Local Variables:
%%% mode: latex
%%% TeX-master: "../story"
%%% End:
