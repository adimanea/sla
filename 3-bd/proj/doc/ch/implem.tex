% !TEX root = ../story.tex

\chapter{Implementare în PostgreSQL}

\section{Instalare și configurare inițială}

Pentru implementare, vom folosi sistemul de baze de date
\href{https://www.postgresql.org/}{PostgreSQL}, disponibil gratuit și
open source.

Implementarea se va face pe laptopul personal, folosind:
\begin{itemize}
\item OS: Manjaro Linux i3;
\item Emacs 26 pentru editare text și comenzi shell;
\item \texttt{st} pentru comenzi avansate de terminal, dacă este necesar.
\end{itemize}

Programul se instalează folosind managerul de pachete din Manjaro,
cu comanda:
{
  \small
\begin{verbatim}
$ sudo pacman -S postgresql
\end{verbatim}
}

După instalare, putem consulta consulta detaliile de pe pagina
\href{https://wiki.archlinux.org/index.php/PostgreSQL\#Create\_your\_first\_database/user}{ArchWiki}.

Pe scurt, PostgreSQL creează automat un utilizator cu numele \texttt{postgres},
care este proprietarul implicit al bazelor de date. Astfel, pentru a lansa
aplicația și a face modificări, trebuie să ne identificăm ca utilizatorul \texttt{postgres},
cu una dintre comenzile:

{
  \small
\begin{verbatim}
$ sudo -iu postgres
$ su -l postgres
\end{verbatim}
}

Inițializarea cluster-ului de baze de date, unde se vor crea bazele și tabelele
se face în calitatea de utilizator \texttt{postgres}, de exemplu, în locația
implicită, cu comanda:

{
  \small
\begin{verbatim}
[postgres]$ initdb -D /var/lib/postgres/data

# pentru a forța utilizarea limbii engleze și a codării UTF-8, folosim comanda
[postgres]$ initdb --locale=en_US.UTF-8 -E UTF8 -D /var/lib/postgres/data

# ... output ...
# finalizat cu ... ok
\end{verbatim}
}

Folosind \texttt{systemd}, trebuie să activăm și să pornim daemon-ul \texttt{postgresql},
cu comenzile:
{
  \small
\begin{verbatim}
$ sudo systemctl enable postgresql
$ sudo systemctl start postgresql
\end{verbatim}
}

Pentru început, creăm un utilizator nou, căruia îi putem da ce atribuții dorim,
precum și o bază de date pe care utilizatorul respectiv să o poată accesa (sau
administra, în funcție de rolul dat):
{
  \small
\begin{verbatim}
# devenim utilizatorul postgres mai întîi
$ sudo -iu postgres

# creăm utilizatorul (e.g. theUser) cu dialog pas cu pas, pentru a alege rolul
[postgres]$ createuser --interactive

# creăm o bază de date pentru el (e.g. theDatabase)
[postgres]$ createdb -O theUser theDatabase

# dacă theUser nu are rol de creare, putem crea cu postgres pentru el
[postgres]$ createdb -U postgres -O theUser theDatabase
\end{verbatim}
}

Acum putem porni subshell-ul \texttt{psql} și să ne conectăm la baza de date
creată mai sus:
{
  \small
\begin{verbatim}
$ sudo -iu postgres
[postgres]$ psql -d theDatabase

# cîteva dintre meta-comenzile pentru subshell-ul psql:
=> \help                    # accesează help
=> \c <database>            # conectează-te la baza de date <database>
=> \du                      # afișează utilizatorii și permisiunile
=> \dt                      # afișează tabelele și permisiunile
=> \q                       # închide subshell-ul
=> \?                       # afișează toate meta-comenzile
\end{verbatim}
}


%%% Local Variables:
%%% mode: latex
%%% TeX-master: "../story"
%%% End:
