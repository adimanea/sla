% ! TEX root = ../story.tex

\chapter{Descrierea aplicației}

Aplicația servește la gestionarea unei biblioteci, în relația cu cititorii,
oferind roluri speciale pentru clienți din mediul academic.

%%%%%%%%%%%%%%%%%%%%%%%%%%%%%%%%%%%%%%%%%%%%%%%%%%%%%%%%%%%%%%%%%%%%%% 
\section{Utilizatori și atribute}
\label{sec:util-atr}

Utilizatorii aplicației, identificați prin IP-uri specifice sînt:
\begin{itemize}
\item \textbf{Administratori} ai bazei de date;
\item \textbf{Personal educațional (Edu)};
\item \textbf{Cititori} din afara mediului educațional;
\item \textbf{Bibliotecari}.
\end{itemize}

Considerăm că studenții și profesorii fac parte din personalul educațional
(Edu), statut pe care trebuie să-l verifice periodic. De asemenea, un
cititor oarecare poate deveni Edu în timp, dacă adaugă verificarea.

%%%%%%%%%%%%%%%%%%%%%%%%%%%%%%%%%%%%%%%%%%%%%%%%%%%%%%%%%%%%%%%%%%%%%% 
\section{Procesele din aplicație}
\label{sec:procese}

\begin{enumerate}[(P1)]
\item Vizualizarea cărților în stoc;
\item Vizualizarea bibliotecarilor, cu specializările lor;
\item Adăugarea unei cărți;
\item Adăugarea unui cititor;
\item Adăugarea unui abonat Edu;
\item Înregistrarea unui cont public;
\item Administrarea utilizatorilor;
\item Actualizarea stocului unor cărți;
\item Verificarea statutului Edu;
\item Administrarea abonamentului;
\item Vizualizarea cărților de o anumită specializare;
\item Vizualizarea revistelor de o anumită specializare;
\item Adăugare reviste;
\item Administrare abonament revistă;
\item Administrare abonament la newsletter;
\end{enumerate}

\todo[inline,noline,backgroundcolor=green!40]{Descompunerea funcțională pe 2 nivele}

%%%%%%%%%%%%%%%%%%%%%%%%%%%%%%%%%%%%%%%%%%%%%%%%%%%%%%%%%%%%%%%%%%%%%% 

\section{Matricea proces-utilizator}
\label{sec:matrice-pu}

\begin{center}
  \small
  \begin{tabular}{|l|c|c|c|c|c|c|c|c|c|c|c|c|c|c|c|}
    \hline
    & P1 & P2 & P3 & P4 & P5 & P6 & P7 & P8 & P9 & P10 & P11 & P12 & P13 & P14 & P15\\
    \hline \hline
    Administrator & X & X & X & X & X & X & X & X & X & X & X & X & X & X & X \\
    \hline
    Edu & X & X & & & & & & & X & X & X & & & X & X \\
    \hline
    Cititor & X & X & & & & & & & X & X & X & & & & X\\
    \hline
    Bibliotecar & X & X & X & X & X & & & X & X & X & X & & X & X & \\
    \hline
    Public & X & & & & & & & & & & X & & & & X \\
    \hline
  \end{tabular}
\end{center}

%%%%%%%%%%%%%%%%%%%%%%%%%%%%%%%%%%%%%%%%%%%%%%%%%%%%%%%%%%%%%%%%%%%%%% 

\section{Entitățile și modelarea datelor}
\label{sec:ent-model}

Biblioteca este un centru de împrumut care permite și abonamentul la reviste
pentru cei din mediul academic.

În bibliotecă se pot adăuga \emph{cărți} și \emph{reviste} de diverse
specializări, în funcție de contractele cu furnizorii. Fiecare
carte și revistă aparțin unei singure specializări.

Personalul Edu se poate abona și la reviste de specialitate, în baza
unei verificări actualizate. Ceilalți cititori cu abonament pot deveni
Edu printr-o verificare. Publicul larg poate doar să vadă stocul
de cărți, general sau pe specializări. Toți utilizatorii se pot abona
la newsletter pentru a afla cînd se actualizează stocul.

De asemenea, bibliotecarii sînt asociați specializărilor, fiind
responsabil de publicațiile dintr-o anumită specializare.

\todo[inline,noline,backgroundcolor=green!40]{diagrama E/R}
\todo[inline,noline,backgroundcolor=green!40]{diagrama conceptuală}

%%%%%%%%%%%%%%%%%%%%%%%%%%%%%%%%%%%%%%%%%%%%%%%%%%%%%%%%%%%%%%%%%%%%%%

\section{Schemele relaționale}
\label{sec:scheme-rel}

\begin{verbatim}
BIBLIOTECAR     (id_bib#, nume, prenume, id_specializare);
CITITOR         (id_cit#, nume, prenume, email, newsletter, abonament_activ,
                 id_carte, id_spec);
EDU             (id_edu#, nume, prenume, email, newsletter, abonament_activ,
                 id_carte, id_rev, id_spec);
PUBLIC          (id_pub#, email, pw);
CARTE           (id_carte#, autor_n, autor_p, titlu, an, id_spec, stoc);
REVISTA         (id_rev#, autor_n, autor_p, titlu, numar, id_spec, stoc);
SPECIALIZARE    (id_spec#, descriere);
\end{verbatim}

\todo[inline,noline,backgroundcolor=green!40]{cum țin număr de împrumuturi \& termen?}

%%%%%%%%%%%%%%%%%%%%%%%%%%%%%%%%%%%%%%%%%%%%%%%%%%%%%%%%%%%%%%%%%%%%%%

\section{Matricea entitate-proces}
\label{sec:matr-ep}

\begin{center}
  \footnotesize
  \begin{tabular}{|l|c|c|c|c|c|c|c|c|c|c|c|c|c|c|c|}
    \hline
    & P1 & P2 & P3 & P4 & P5 & P6 & P7 & P8 & P9 & P10 & P11 & P12 & P13 & P14 & P15\\
    \hline \hline
    \texttt{BIBLIOTECAR} & S & S & I,U,D & I,U,D & I,U,D & & & I,U,D & U & S & S & I,U,D & I,U,D & U & \\
    \hline
    \texttt{CITITOR} & S & S & & & & & & & I,U,D & I,U,D & S & & & & U \\
    \hline
    \texttt{EDU} & S & S & & & & & & & U & I,U,D & S & S & & U & U \\
    \hline
    \texttt{PUBLIC} & S & & & & & I,U,D & & & & & & S & & & I,U,D \\
    \hline
    \texttt{CARTE} & S & & I,U,D & & & & & & & & S & & & & \\
    \hline
    \texttt{REVISTA} & & & & & & & & & & & & S & I & & \\
    \hline
    \texttt{SPECIALIZARE} & & & & & & & & & & & S & S & & & \\
    \hline
  \end{tabular}
\end{center}

\emph{Legenda:} I = Insert, U = update, D = delete, S = select.

%%%%%%%%%%%%%%%%%%%%%%%%%%%%%%%%%%%%%%%%%%%%%%%%%%%%%%%%%%%%%%%%%%%%%%

\section{Matricea entitate-utilizator}
\label{sec:matr-eu}

\begin{center}
  \small
  \begin{tabular}{|l|c|c|c|c|c|}
    \hline
    & Edu & Cititor & Bibliotecar & Admin & Public \\
    \hline\hline
    \texttt{BIBLIOTECAR} & S & S & S & S, I, U, D & \\
    \hline
    \texttt{CITITOR} & & S, I, U, D & S, I, U, D & & \\
    \hline
    \texttt{EDU} & & S, I, U, D & S, I, U, D & & \\
    \hline
    \texttt{PUBLIC} & & & & & \\
    \hline
    \texttt{CARTE} & S & S & S, I, U, D & S, I, U, D & S \\
    \hline
    \texttt{REVISTA} & S & & S, I, U, D & S, I, U, D & S \\
    \hline
    \texttt{SPECIALIZARE} & S & S & S & S, I, U, D & S \\
    \hline
  \end{tabular}
\end{center}

\emph{Legenda:} I = Insert, U = update, D = delete, S = select.

%%%%%%%%%%%%%%%%%%%%%%%%%%%%%%%%%%%%%%%%%%%%%%%%%%%%%%%%%%%%%%%%%%%%%%

\section{Utilizatori și conturi}
\label{sec:ut-cont}

Conturile utilizatorilor vor fi definite individual pentru utilizatorii
identificați de bibliotecă.

În această etapă a dezvoltării aplicației, vor fi disponibile maximum
10000 de conturi Edu, 10000 de conturi de cititori (înregistrați),
10 conturi de bibliotecar, 1000 conturi de utilizator neabonat
(\texttt{public}).



\todo[inline,noline,backgroundcolor=green!40]{clarificare tabele/scheme}

%%% Local Variables:
%%% mode: latex
%%% TeX-master: "../story"
%%% End:
