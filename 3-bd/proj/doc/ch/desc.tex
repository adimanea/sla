% ! TEX root = ../story.tex

\chapter{Descrierea aplicației}

Aplicația urmărește să servească pentru gestionarea unei biblioteci,
atît în relația cu cititorii (clienții), cît și cu furnizorii (edituri,
persoane fizice sau persoane juridice). De asemenea, biblioteca are
șoferi angajați, care livrează periodice sau cărți cititorilor.

%%%%%%%%%%%%%%%%%%%%%%%%%%%%%%%%%%%%%%%%%%%%%%%%%%%%%%%%%%%%%%%%%%%%%% 
\section{Utilizatori și atribute}
\label{sec:util-atr}

Utilizatorii aplicației și atributele lor sînt:
\begin{itemize}
\item \textbf{Studenți}, identificați prin afiliere și CNP;
\item \textbf{Profesori}, identificați prin afiliere și CNP;
\item \textbf{Cititori înregistrați}, identificați prin CNP;
\item \textbf{Administratorii} aplicației, identificați prin CNP;
\item \textbf{Bibliotecarii}, identificați prin CNP;
\item \textbf{Publicul} (cititori neînregistrați) identificat prin CNP;
\item \textbf{Șoferii}, identificați prin CNP.
\end{itemize}

La nivelul tuturor acestor utilizatori, CNP-ul servește drept atribut
de identificare. De asemenea, putem utiliza și o adresă de e-mail confirmată,
care poate servi drept contact pentru diverse anunțuri.

%%%%%%%%%%%%%%%%%%%%%%%%%%%%%%%%%%%%%%%%%%%%%%%%%%%%%%%%%%%%%%%%%%%%%% 
\section{Procesele din aplicație}
\label{sec:procese}

\begin{enumerate}[(P1)]
\item Administrare utilizatori;
\item Adăugarea unei cărți;
\item Adăugarea unei reviste;
\item Adăugarea unui cititor;
\item Adăugarea unui șofer;
\item Adăugarea unui bibliotecar;
\item Abonare la periodice;
\item Actualizarea dovezii de plată;
\item Actualizarea dovezii de statut academic;
\item Vizualizarea cărților;
\item Vizualizarea revistelor;
\item Vizualizarea cititorilor;
\item Vizualizarea furnizorilor;
\end{enumerate}

\todo[inline,noline,backgroundcolor=green!40]{Descompunerea funcțională pe 2 nivele}

%%%%%%%%%%%%%%%%%%%%%%%%%%%%%%%%%%%%%%%%%%%%%%%%%%%%%%%%%%%%%%%%%%%%%% 

\section{Matricea proces-utilizator}
\label{sec:matrice-pu}

\begin{center}
  \small
  \begin{tabular}{|l|c|c|c|c|c|c|c|c|c|c|c|c|c|}
    \hline
    & P1 & P2 & P3 & P4 & P5 & P6 & P7 & P8 & P9 & P10 & P11 & P12 & P13\\
    \hline \hline
    Bibliotecari & & X & X & X & & & & X & X & X & X & X & X \\
    \hline
    Studenți & & & & & & & & X & X & X & X & & \\
    \hline
    Profesori & & & & & & & X & X & X & X & X & & \\
    \hline
    Cititori & & & & & & & & X & X & X & X & X & \\
    \hline
    Public & & & & & & & X & X & X & X & X & & X \\
    \hline
    Administratori & X & X & X & X & X & X & X & X & X & X & X & X & X\\
    \hline
    Șofer & & & & & & & & & & & & X & \\
    \hline
  \end{tabular}
\end{center}

%%%%%%%%%%%%%%%%%%%%%%%%%%%%%%%%%%%%%%%%%%%%%%%%%%%%%%%%%%%%%%%%%%%%%% 

\section{Entitățile și modelarea datelor}
\label{sec:ent-model}

Aplicația servește unei biblioteci moderne, care permite împrumutul
cărților sau periodicelor, cu oferte speciale pentru profesori,
cercetători și studenți. Mai mult, ea oferă și legături directe cu
furnizori, care pot fi edituri sau persoane fizice, constituindu-se
și într-un fel de anticariat cu închiriere. Toate cărțile și
periodicele pot fi împrumutate de la sediul bibliotecii sau livrate cu
ajutorul șoferilor bibliotecii. Șoferii sînt asociați cu anumite
regiuni în care se pot deplasa.

În baza de date, bibliotecarul poate adăuga \emph{cărțile} și \emph{periodicele}
disponibile. Fiecare dintre acestea pot fi adăugate într-o cantitate arbitrară,
care constituie \emph{stocul}. O carte poate proveni de la mai mulți
\emph{furnizori}, deoarece se permit multiple exemplare, care să mărească
stocul.

Totodată, bibliotecarul poate adăuga cititori, care pot fi \emph{profesori, %
  studenți} sau din \emph{publicul larg}. Un student nu se poate abona
la periodice și poate împrumuta maximum 10 cărți, iar un profesor se poate
abona la maximum 10 periodice și poate împrumuta maximum 10 cărți.
Publicul larg nu se poate abona la periodice și poate împrumuta maximum
5 cărți.

Utilizatorii își pot administra datele de plată, pentru a putea să-și
înnoiască abonamentul automat.

Șoferii pot vizualiza numai adresele cititorilor care au ales să facă acest
cîmp vizibil, pentru a putea livra cărțile sau periodicele corespunzătoare.
Aceasta se va face numai dacă metoda de plată este verificată, iar abonamentul,
activ.

\todo[inline,noline,backgroundcolor=green!40]{diagrama E/R}
\todo[inline,noline,backgroundcolor=green!40]{diagrama conceptuală}

%%%%%%%%%%%%%%%%%%%%%%%%%%%%%%%%%%%%%%%%%%%%%%%%%%%%%%%%%%%%%%%%%%%%%%

\section{Schemele relaționale}
\label{sec:scheme-rel}

\begin{verbatim}
CARTE       (cod_carte#, titlu_carte, autor_carte, 
             editura, furnizor, stoc)
REVISTA     (cod_revista#, titlu_revista, numar_revista, 
             an_revista, furnizor, stoc)
STUDENT     (cod_student#, nume_student, prenume_student, adresa_student,
             adresa_publica, modalitate_plata, statut_valid, email, afiliere)
PROFESOR    (cod_prof#, nume_prof, prenume_prof, adresa_prof,
             adresa_publica, modalitate_plata, statut_valid, email, afiliere)
CITITOR     (cod_cititor#, nume_cititor, prenume_cititor, adresa_cititor,
             adresa_publica, modalitate_plata, statut_valid, email)
PUBLIC      (cod_public#)
BIBLIOTECAR (cod_bib#, nume_bib, prenume_bib)
SOFER       (cod_sofer#, nume_sofer, prenume_sofer, regiune)
ADMIN       (cod_admin#, nume_admin, prenume_admin)
FURNIZOR    (cod_furnizor#, tip_furnizor, nume_furnizor, prenume_furnizor)
\end{verbatim}

\todo[inline,noline,backgroundcolor=green!40]{cum țin număr de împrumuturi \& termen?}

%%%%%%%%%%%%%%%%%%%%%%%%%%%%%%%%%%%%%%%%%%%%%%%%%%%%%%%%%%%%%%%%%%%%%%

\section{Matricea entitate-proces}
\label{sec:matr-ep}

\begin{center}
  \footnotesize
  \begin{tabular}{|l|c|c|c|c|c|c|c|c|c|c|c|c|c|}
    \hline
    & P1 & P2 & P3 & P4 & P5 & P6 & P7 & P8 & P9 & P10 & P11 & P12 & P13 \\
    \hline \hline
    \texttt{CARTE} & & I,U,D & & & & & & & & S & & & \\
    \hline
    \texttt{REVISTA} & & & I,U,D & & & & & & & & S & & \\
    \hline
    \texttt{STUDENT} & I,U,D & & I,U,D & & & & & I,U,D & I,U,D & & & S & \\
    \hline
    \texttt{PROFESOR} & I,U,D & & I,U,D & & & & I,U,D & I,U,D & I,U,D & & & & \\
    \hline
    \texttt{CITITOR} & I,U,D & & I,U,D & & & & & I,U,D & I,U,D & & & & \\
    \hline
    \texttt{PUBLIC} & & & I,U,D & & & & & I,U,D & I,U,D & & & & \\
    \hline
    \texttt{BIBLIOTECAR} & I,U,D & & & & & I,U,D & & & & & & & \\
    \hline
    \texttt{SOFER} & I,U,D & & & & I,U,D & & & & & & & & \\
    \hline
    \texttt{ADMIN} & I,U,D & & & & & & & & & & & & \\
    \hline
    \texttt{FURNIZOR} & I,U,D& & & & & & & & & & & & S \\
    \hline
  \end{tabular}
\end{center}

\emph{Legenda:} I = Insert, U = update, D = delete, S = select.

%%%%%%%%%%%%%%%%%%%%%%%%%%%%%%%%%%%%%%%%%%%%%%%%%%%%%%%%%%%%%%%%%%%%%%

\section{Matricea entitate-utilizator}
\label{sec:matr-eu}

\begin{center}
  \small
  \begin{tabular}{|l|c|c|c|c|c|c|c|}
    \hline
    & Profesor & Student & Cititor & Bibliotecar & Șofer & Admin & Public \\
    \hline\hline
    \texttt{CARTE} & S & S & S & S, I, U, D & & S, I, U, D & S \\
    \hline
    \texttt{REVISTA} & S & S & S & S, I, U, D & & S, I, U, D & S \\
    \hline
    \texttt{STUDENT} & & & & S, I, U, D & S & S, I, U, D & \\
    \hline
    \texttt{PROFESOR} & & & & S, I, U, D & S & S, I, U, D & \\
    \hline
    \texttt{CITITOR} & & & & S, I, U, D & S & S, I, U, D & \\
    \hline
    \texttt{BIBLIOTECAR} & & & & & & S, I, U, D & \\
    \hline
    \texttt{SOFER} & & & & S & & S, I, U, D & \\
    \hline
    \texttt{ADMIN} & & & & & & S, I, U, D & \\
    \hline
    \texttt{FURNIZOR} & S & S & S & S & S & S, I, U, D & S \\
    \hline
  \end{tabular}
\end{center}

\emph{Legenda:} I = Insert, U = update, D = delete, S = select.

%%%%%%%%%%%%%%%%%%%%%%%%%%%%%%%%%%%%%%%%%%%%%%%%%%%%%%%%%%%%%%%%%%%%%%

\section{Utilizatori și conturi}
\label{sec:ut-cont}

Conturile utilizatorilor vor fi definite individual pentru utilizatorii
identificați de bibliotecă.

În această etapă a dezvoltării aplicației, vor fi disponibile maximum
10000 de conturi de studenți, 10000 de conturi de profesori, 10000 de
conturi de cititori (înregistrați), 10 conturi de bibliotecar,
10 conturi de șofer.

De asemenea, se va mai crea un cont de vizitator (\texttt{PUBLIC}).

\textbf{Proprietarul unui obiect al bazei de date} este utilizatorul
care are drepturi nelimitate de utilizare și administrare a obiectului
respectiv. Aceste drepturi nu îi pot fi revocate nici chiar de administratorul
bazei de date.

Un utilizator care este proprietarul cel puțin al unui obiect din baza
de date nu poate fi șters din sistem.

\textbf{Schema unui utilizator} este totalitatea obiectelor (tabele,
view-uri, indecși, funcții și proceduri PL/SQL, pachete PL/SQL), care au
ca proprietar un anume utilizator al bazei de date.

\todo[inline,noline,backgroundcolor=green!40]{clarificare tabele/scheme}

%%% Local Variables:
%%% mode: latex
%%% TeX-master: "../story"
%%% End:
