\chapter{Aspecte de securitate}

\textbf{Observație:} Pentru simplitate, în cele ce urmează, vom omite
prompt-ul de la subshell-ul \texttt{psql}, precum și indicațiile de
autentificare ca utilizatorul \texttt{postgres}. Așadar, vom presupune
că utilizatorul s-a identificat deja și s-a conectat la baza de date
\texttt{biblioteca}, conform indicațiilor din capitolul anterior,
pe care le reluăm pentru ultima dată:

{
  \small
\begin{verbatim}
$ sudo -iu postgres
[postgres@~/]$ psql -d biblioteca
biblioteca# -- comenzile se vor face de la acest nivel
\end{verbatim}
}

Astfel, vom scrie doar \texttt{\#} ca prefix pentru comenzile
introduse de programator, iar răspunsul programului se va nota fără
prefix, adică, în general:
{
  \small
\begin{verbatim}
# comanda psql
# continuare comanda;
RĂSPUNS
continuare răspuns
\end{verbatim}
}

%%%%%%%%%%%%%%%%%%%%%%%%%%%%%%%%%%%%%%%%%%%%%%%%%%%%%%%%%%%%%%%%%%%%%%
\section{Criptarea parolelor cu \texttt{pgcrypto}}
\index{securitate!criptarea \texttt{pgcrypto}}

Utilizatorii care au abonament (de tip \texttt{EDU} și \texttt{NEDU}) își
administrează datele și împrumuturile prin in\-ter\-ac\-ți\-u\-ne directă cu
bibliotecarii. Dacă vor să lase un feedback, se adresează bibliotecarului,
care verifică situația și dă un token unic cu care se autentifică pentru
a lăsa feedback.

În schimb, utilizatorii externi (\texttt{RESTUL}) își pot crea un cont
și se identifică cu email-ul și parola pentru a consulta baza de cărți
și stocul. Această parolă va fi stocată în baza de date prin calculul
hash-ului, folosind extensia \texttt{pgcrypto}. De exemplu, pentru a crea
un hash salted al parolei \texttt{mypass}, folosind algoritmul blowfish
cu 4 runde de criptare, se rulează comenzile (în subshell-ul \texttt{psql}):

{
  \small
\begin{verbatim}
# create extension pgcrypto;
# select crypt('mypass', gen_salt('bf', 4));
CRYPT
----------------------------------------------------------
 $2a$04$1bfMQDOR6aLyD4q3KVb8/ujG7ZAkyie4d/s3ABwuZNbzkFFgXtC76
\end{verbatim}
}

\index{securitate!hash al parolei}
Acum putem introduce în baza de date acest hash pentru cîmpul de parolă.
Mai întîi, adăugăm coloana corespunzătoare parolei criptate:

{
  \small
\begin{verbatim}
# alter table restul
# add column pw_enc varchar(100);
ALTER TABLE
\end{verbatim}
}

Apoi, introducem parolele criptate pentru utilizatorii respectivi:

{
  \small
\begin{verbatim}
# update restul
# set pw_enc = '$2a$04$1bfMQDOR6aLyD4q3KVb8/ujG7ZAkyie4d/s3ABwuZNbzkFFgXtC76' 
# where id_pub = 3;
\end{verbatim}
}

Cum criptarea se face cu o cheie simetrică, putem verifica apoi:

{
  \small
\begin{verbatim}
# select * from restul where id_pub = 5 
# and pw_enc = crypt('mypass', pw_enc);

id_pub  | pw_enc
--------+--------------------------------------------------------------
5       | $2a$04$1bfMQDOR6aLyD4q3KVb8/ujG7ZAkyie4d/s3ABwuZNbzkFFgXtC76
(1 row)
\end{verbatim}
}

%%%%%%%%%%%%%%%%%%%%%%%%%%%%%%%%%%%%%%%%%%%%%%%%%%%%%%%%%%%%%%%%%%%%%%
\section{Trigger pentru actualizare}
\index{securitate!trigger actualizare}

Presupunem că bibliotecarul vrea să urmărească actualizarea stocului
unor cărți și, de asemenea, că administratorul aplicației vrea să facă
o statistică despre utilizatorii neabonați care urmăresc biblioteca
(pentru o perioadă fixată de timp, e.g.\ o lună).

Pentru aceste două scopuri, vom crea cîte un trigger. Primul va urmări
cînd se modifică înregistrările din coloana \texttt{stoc} a tabelei
\texttt{carte}, iar al doilea va urmări cînd apar înregistrări noi în
tabela \texttt{restul}.

Înregistrăm diferențele în tabele separate, \texttt{stoc\_mod} și,
respectiv, \texttt{restul\_mod}, pe care le creăm corespunzător:

{
  \small
\begin{verbatim}
# create table stoc_mod (
# id_stoc_mod serial primary key,
# carte_id int not null,
# autor_n varchar(20) not null,
# autor_p varchar(20) not null,
# titlu varchar(30) not null,
# schimbare_stoc smallint not null,
# schimbat timestamp(6) not null );
\end{verbatim}
}

Apoi se definește funcția corespunzătoare, care urmărește și înregistrează
schimbările:

{
  \small
\begin{verbatim}
# create or replace function log_stoc()
# returns trigger as
# $body$
# begin
# if new.stoc <> old.stoc then
# insert into stoc_mod(id_stoc_mod, carte_id, autor_n,
#                      autor_p, titlu, schimbare_stoc, schimbat)
# values(DEFAULT, old.id_carte, old.autor_n, old.autor_p, old.titlu, 
#        new.stoc - old.stoc, now());
# end if;
# return new;
# end;
# $body$ language plpgsql;
CREATE FUNCTION
\end{verbatim}
}

Apoi legăm această funcție de un trigger, care să urmărească tabelul
și coloana de \texttt{stoc}:
{
  \small
\begin{verbatim}
# create trigger stoc_modificare
# before update
# on carte
# for each row
# execute procedure log_stoc();
CREATE TRIGGER
\end{verbatim}
}

Modificările pot fi văzute atunci cînd se schimbă o valoare de \texttt{stoc}
din tabela carte, lucru care poate fi testat astfel:
{
  \small
\begin{verbatim}
# update carte
# set stoc=5 where id_carte=7;
UPDATE 1
# select * from stoc_mod;
 id_stoc_mod | carte_id | autor_n | autor_p |  titlu  | schimbare_stoc | schimbat          
-------------+----------+---------+---------+---------+----------------+-----------------
           1 |        7 | Allah   | Akbar   | Coranul |              9 | 2020-01-18 10:49
           2 |        7 | Allah   | Akbar   | Coranul |             -8 | 2020-01-18 10:52
           3 |        7 | Allah   | Akbar   | Coranul |              4 | 2020-01-18 10:54
(3 rows)
\end{verbatim}
}

Similar, pentru cealaltă situație, creăm un tabel care adaugă noii utilizatori
neabonați:
{
  \small
\begin{verbatim}
# create table restul_mod (
# id_restul_mod serial primary key,
# email_nab varchar(30) not null,
# adaugat timestamp(6) not null );
\end{verbatim}
}

Apoi funcția care urmărește schimbările:
{
  \small
\begin{verbatim}
# create or replace function nab_nou()
# returns trigger as
# $body$
# begin
# insert into restul_mod(id_restul_mod, email_nab, adaugat)
# values (DEFAULT, new.email, now());
# end if;
# return new;
# end;
# $body$ language plpgsql;
CREATE FUNCTION
\end{verbatim}
}
\index{securitate!funcție de actualizare}

În fine, trigger-ul, care de data aceasta se execută atunci cînd
se adaugă un rînd nou în tabel:
{
  \small
\begin{verbatim}
# create trigger nab_modificare
# after insert                  -- declanșează doar după INSERT
# on restul
# for each row
# execute procedure nab_nou();
CREATE TRIGGER
\end{verbatim}
}

Și testăm cu:
{
  \small
\begin{verbatim}
# insert into restul(id_pub, email, pw, newsletter) 
# values(DEFAULT, 'newone@y.c', 'IamNewHere', true);
INSERT 1
# select * from restul_mod;

 id_restul_mod | email_nab  |          adaugat           
---------------+------------+----------------------------
             3 | newone@y.c | 2020-01-18 11:23:41.204905
(1 row)
\end{verbatim}
}

%%%%%%%%%%%%%%%%%%%%%%%%%%%%%%%%%%%%%%%%%%%%%%%%%%%%%%%%%%%%%%%%%%%%%%
\section{Utilizatori și roluri}
\index{securitate!roluri}
\index{securitate!permisiuni}
\index{securitate!atribute (\texttt{GRANT})}

Utilizatorii pe care îi creăm pot avea mai multe tipuri de privilegii. În ordinea
descrescătoare a permisiunilor, acestea sînt:
\begin{itemize}
\item \texttt{SUPERUSER}: un astfel de rol are drepturi depline și trece peste
  orice verificare de securitate. Mai mult decît atît, un utilizator care are
  acest rol are inclusiv posibilitatea de a mai crea încă unul cu aceleași
  drepturi. De aceea, rolul \texttt{SUPERUSER} trebuie folosit cu prudență;
\item \texttt{CREATEDB}: rolul permite creare bazelor de date;
\item \texttt{CREATEROLE}: putem crea încă un alt rol, mai puțin cel de tip
  \texttt{SUPERUSER};
\item \texttt{LOGIN}: permite utilizatorului să se logheze drept client
  într-o conexiune cu baza de date.
\end{itemize}

Sintaxa pentru crearea rolurilor este simplă:
{
  \small
\begin{verbatim}
# -- crearea rolului
# create role unRol;
CREATE ROLE
# -- ștergerea rolului creat;
# drop role unRol;
DROP ROLE
\end{verbatim}
}

Implicit, dacă nu se specifică exact rolul pe care să îl aibă utilizatorul,
el nu va avea niciun drept. Apoi, dacă vrem să creăm utilizatori cu roluri
anume, se pot atribui explicit sau se pot modifica rolurile existente:
{
  \small
\begin{verbatim}
# -- utilizatorul are parolă, dar nu poate face nimic cu ea
# create role nolog_user with password 'pass1';
CREATE ROLE
# -- utilizator cu parolă, cu drepturi de LOGIN
# create role log_user with login password 'pass2';
CREATE ROLE
# -- modificăm rolul și mai adăugăm privilegii
# alter role log_user createrole createdb;
ALTER ROLE
\end{verbatim}
}

Verificarea se poate face în tabela \texttt{pg\_role}, cu sau fără filtrare:
{
  \small
\begin{verbatim}
# select rolcreaterole, rolcreatedb from pg_roles where rolname='log_user';

 rolcreaterole | rolcreatedb 
---------------+-------------
 t             | t
(1 row)
\end{verbatim}
}

Alternativ, putem verifica toate rolurile folosind meta-comanda \texttt{\textbackslash du},
care afișează toate rolurile și dacă sînt incluse în grupuri.

%%%%%%%%%%%%%%%%%%%%%%%%%%%%%%%%%%%%%%%%%%%%%%%%%%%%%%%%%%%%%%%%%%%%%%
\subsection{Fișierul de configurare \texttt{pg\_hba.conf}}
\index{securitate!\texttt{pg\_hba.conf}}

Setările privitoare la roluri și modurile de autentificare sînt salvate în fișierul
\texttt{pg\_hba.conf}, pe care îl putem găsi cu o comandă de forma:
{
  \small
\begin{verbatim}
# select name, setting from pg_settings where name like '%hba';

 name     | setting 
----------+-------------------------------------
 hba_file | /etc/postgresql/10/main/pg_hba.conf
(1 row)
\end{verbatim}
}

Acest fișier specifică rolurile, drepturile și modurile de conectare la bazele
de date (prin arhitectura client-server sau local). Mai multe detalii pot fi
găsite în documentația oficială (\cite{pgoff}). Deoarece vom folosi doar o conexiune
locală, în continuare ne vom concentra pe această modalitate. Pentru aceasta,
avem sintaxa generală:
{
  \small
\begin{verbatim}
local      database  user  auth-method  [auth-options]
\end{verbatim}
}

Metodele de autentificare (\texttt{auth-method}) sînt multiple, dintre ele menționăm:
\begin{itemize}
\item \texttt{trust}: conexiunea se face automat, fără verificare;
\item \texttt{reject}: conexiunea se respinge automat;
\item \texttt{password}: se cere clientului o parolă, care se verifică în text clar;
\item \texttt{md5, scram-sha-256}: se cere clientului o parolă, care se verifică
  prin metodele de hashing din nume.
\end{itemize}

Pentru cele două roluri din exemplul de mai sus, în fișierul \texttt{pg\_hba.conf}
avem intrările:
{
  \small
\begin{verbatim}
# TYPE  DATABASE        USER            ADDRESS                 METHOD
local   all             nolog_user                              password
local   all             log_user                                password
\end{verbatim}
}

Putem verifica acest lucru dintr-un rol de \texttt{SUPERUSER} cu următorul
query:
{
  \small
\begin{verbatim}
# SELECT database, user_name, auth_method
# FROM pg_hba_file_rules
# WHERE CAST(user_name AS TEXT) LIKE '%log_user%';

 database | user_name    | auth_method 
----------+--------------+-------------
    {all} | {nolog_user} | password
    {all} | {log_user}   | password
\end{verbatim}
}

Apoi, deoarece utilizatorul \texttt{nolog\_user} nu se poate autentifica,
primim erori, chiar dacă acestuia i-a fost atribuită o parolă:
{
  \small
\begin{verbatim}
[postgres]$ psql -U nolog_user -W postgres
Password for user nolog_user: 
psql: FATAL: role "nolog_user" is not permitted to log in
[postgres]$ psql -U log_user -W postgres
Password for user log_user: 
psql (12.1)
Type "help" for help.
postgres=>
\end{verbatim}
}

%%%%%%%%%%%%%%%%%%%%%%%%%%%%%%%%%%%%%%%%%%%%%%%%%%%%%%%%%%%%%%%%%%%%%%
\subsection{Atribuirea (\texttt{GRANT}) permisiunilor}
\index{securitate!atribute (\texttt{GRANT})}

Pentru ca rolurile create să poată folosi obiectele din bazele de date
(tabele, vizualizări, coloane, funcții etc.), trebuie să le fie acordate
privilegii în acest sens. Acest lucru se face cu comanda \texttt{GRANT}.

De exemplu, putem crea un rol care poate doar să vizualizeze unele
coloane dintr-un anumit tabel. În cazul nostru, putem permite unui
bibliotecar să vadă doar numele, prenumele, email-ul și statutul abonamentului
cititorilor \texttt{EDU} sau \texttt{NEDU}, fără alte informații. De asemenea,
bibliotecarul respectiv poate modifica doar cîmpul \texttt{ab\_activ} pentru
fiecare cititor.

Pentru aceasta, creăm utilizatorul corespunzător, cu credențiale și
drepturi de login:
{
  \small
\begin{verbatim}
# create role bib1 with login password 'ilovebooks';
CREATE ROLE
# grant select (nume, prenume, email, ab_activ) on table edu to bib1;
GRANT
# grant select (nume, prenume, email, ab_activ) on table nedu to bib1;
GRANT
# grant update (ab_activ) on table edu to bib1;
GRANT
# grant update (ab_activ) on table nedu to bib1;
GRANT
\end{verbatim}
}

Ne conectăm apoi cu utilizatorul \texttt{bib1} la baza de date \texttt{biblioteca}
și testăm:
{
  \small
\begin{verbatim}
[postgres]$ psql -U bib1 -W biblioteca
Password:
psql (12.1)
# select * from edu;
ERROR: permission denied for table edu
# select select nume from edu;
    nume    
------------
 Ionescu
 Marcelescu
 Georgescu
(3 rows)
# update edu
# set ab_activ='f' where nume='Ionescu';
UPDATE 1
\end{verbatim}
}

%%% Local Variables:
%%% mode: latex
%%% TeX-master: "../story"
%%% End:
