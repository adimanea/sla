\chapter{Aspecte de securitate}

\textbf{Observație:} Pentru simplitate, în cele ce urmează, vom omite
prompt-ul de la subshell-ul \texttt{psql}, precum și indicațiile de
autentificare ca utilizatorul \texttt{postgres}. Așadar, vom presupune
că utilizatorul s-a identificat deja și s-a conectat la baza de date
\texttt{biblioteca}, conform indicațiilor din capitolul anterior,
pe care le reluăm pentru ultima dată:

{
  \small
\begin{verbatim}
$ sudo -iu postgres
[postgres@~/]$ psql -d biblioteca
biblioteca# -- comenzile se vor face de la acest nivel
\end{verbatim}
}

Astfel, vom scrie doar \texttt{\#} ca prefix pentru comenzile
introduse de programator, iar răspunsul programului se va nota fără
prefix, adică, în general:
{
  \small
\begin{verbatim}
# comanda psql
# continuare comanda;
RĂSPUNS
continuare răspuns
\end{verbatim}
}

\section{Criptarea parolelor cu \texttt{pgcrypto}}

Utilizatorii care au abonament (de tip \texttt{EDU} și \texttt{NEDU}) își
administrează datele și împrumuturile prin in\-ter\-ac\-ți\-u\-ne directă cu
bibliotecarii. Dacă vor să lase un feedback, se adresează bibliotecarului,
care verifică situația și dă un token unic cu care se autentifică pentru
a lăsa feedback.

În schimb, utilizatorii externi (\texttt{RESTUL}) își pot crea un cont
și se identifică cu email-ul și parola pentru a consulta baza de cărți
și stocul. Această parolă va fi stocată în baza de date prin calculul
hash-ului, folosind extensia \texttt{pgcrypto}. De exemplu, pentru a crea
un hash salted al parolei \texttt{mypass}, folosind algoritmul blowfish
cu 4 runde de criptare, se rulează comenzile (în subshell-ul \texttt{psql}):

{
  \small
\begin{verbatim}
# create extension pgcrypto;
# select crypt('mypass', gen_salt('bf', 4));
CRYPT
----------------------------------------------------------
 $2a$04$1bfMQDOR6aLyD4q3KVb8/ujG7ZAkyie4d/s3ABwuZNbzkFFgXtC76
\end{verbatim}
}

Acum putem introduce în baza de date acest hash pentru cîmpul de parolă.
Mai întîi, adăugăm coloana corespunzătoare parolei criptate:

{
  \small
\begin{verbatim}
# alter table restul
# add column pw_enc varchar(100);
ALTER TABLE
\end{verbatim}
}

Apoi, introducem parolele criptate pentru utilizatorii respectivi:

{
  \small
\begin{verbatim}
# update restul
# set pw_enc = '$2a$04$1bfMQDOR6aLyD4q3KVb8/ujG7ZAkyie4d/s3ABwuZNbzkFFgXtC76' 
# where id_pub = 3;
\end{verbatim}
}

Cum criptarea se face cu o cheie simetrică, putem verifica apoi:

{
  \small
\begin{verbatim}
# select * from restul where id_pub = 5 
# and pw_enc = crypt('mypass', pw_enc);

id_pub  | pw_enc
--------+--------------------------------------------------------------
5       | $2a$04$1bfMQDOR6aLyD4q3KVb8/ujG7ZAkyie4d/s3ABwuZNbzkFFgXtC76
(1 row)
\end{verbatim}
}

%%%%%%%%%%%%%%%%%%%%%%%%%%%%%%%%%%%%%%%%%%%%%%%%%%%%%%%%%%%%%%%%%%%%%%

\section{Trigger pentru actualizare}

Presupunem că bibliotecarul vrea să urmărească actualizarea stocului
unor cărți și, de asemenea, că administratorul aplicației vrea să facă
o statistică despre utilizatorii neabonați care urmăresc biblioteca
(pentru o perioadă fixată de timp, e.g.\ o lună).

Pentru aceste două scopuri, vom crea cîte un trigger. Primul va urmări
cînd se modifică înregistrările din coloana \texttt{stoc} a tabelei
\texttt{carte}, iar al doilea va urmări cînd apar înregistrări noi în
tabela \texttt{restul}.

Înregistrăm diferențele în tabele separate, \texttt{stoc\_mod} și,
respectiv, \texttt{restul\_mod}, pe care le creăm corespunzător:

{
  \small
\begin{verbatim}
# create table stoc_mod (
# id_stoc_mod serial primary key,
# carte_id int not null,
# autor_n varchar(20) not null,
# autor_p varchar(20) not null,
# titlu varchar(30) not null,
# schimbare_stoc smallint not null,
# schimbat timestamp(6) not null );
\end{verbatim}
}

Apoi se definește funcția corespunzătoare, care urmărește și înregistrează
schimbările:

{
  \small
\begin{verbatim}
# create or replace function log_stoc()
# returns trigger as
# $body$
# begin
# if new.stoc <> old.stoc then
# insert into stoc_mod(id_stoc_mod, carte_id, autor_n,
#                      autor_p, titlu, schimbare_stoc, schimbat)
# values(DEFAULT, old.id_carte, old.autor_n, old.autor_p, old.titlu, 
#        new.stoc - old.stoc, now());
# end if;
# return new;
# end;
# $body$ language plpgsql;
CREATE FUNCTION
\end{verbatim}
}

Apoi legăm această funcție de un trigger, care să urmărească tabelul
și coloana de \texttt{stoc}:
{
  \small
\begin{verbatim}
# create trigger stoc_modificare
# before update
# on carte
# for each row
# execute procedure log_stoc();
CREATE TRIGGER
\end{verbatim}
}

Modificările pot fi văzute atunci cînd se schimbă o valoare de \texttt{stoc}
din tabela carte, lucru care poate fi testat astfel:
{
  \small
\begin{verbatim}
# update carte
# set stoc=5 where id_carte=7;
UPDATE 1
# select * from stoc_mod;
 id_stoc_mod | carte_id | autor_n | autor_p |  titlu  | schimbare_stoc | schimbat          
-------------+----------+---------+---------+---------+----------------+-----------------
           1 |        7 | Allah   | Akbar   | Coranul |              9 | 2020-01-18 10:49
           2 |        7 | Allah   | Akbar   | Coranul |             -8 | 2020-01-18 10:52
           3 |        7 | Allah   | Akbar   | Coranul |              4 | 2020-01-18 10:54
(3 rows)
\end{verbatim}
}

Similar, pentru cealaltă situație, creăm un tabel care adaugă noii utilizatori
neabonați:
{
  \small
\begin{verbatim}
# create table restul_mod (
# id_restul_mod serial primary key,
# email_nab varchar(30) not null,
# adaugat timestamp(6) not null );
\end{verbatim}
}

Apoi funcția care urmărește schimbările:
{
  \small
\begin{verbatim}
# create or replace function nab_nou()
# returns trigger as
# $body$
# begin
# insert into restul_mod(id_restul_mod, email_nab, adaugat)
# values (DEFAULT, new.email, now());
# end if;
# return new;
# end;
# $body$ language plpgsql;
CREATE FUNCTION
\end{verbatim}
}

În fine, trigger-ul, care de data aceasta se execută atunci cînd
se adaugă un rînd nou în tabel:
{
  \small
\begin{verbatim}
# create trigger nab_modificare
# after insert                  -- declanșează doar după INSERT
# on restul
# for each row
# execute procedure nab_nou();
CREATE TRIGGER
\end{verbatim}
}

Și testăm cu:
{
  \small
\begin{verbatim}
# insert into restul(id_pub, email, pw, newsletter) 
# values(DEFAULT, 'newone@y.c', 'IamNewHere', true);
INSERT 1
# select * from restul_mod;

 id_restul_mod | email_nab  |          adaugat           
---------------+------------+----------------------------
             3 | newone@y.c | 2020-01-18 11:23:41.204905
(1 row)
\end{verbatim}
}

%%%%%%%%%%%%%%%%%%%%%%%%%%%%%%%%%%%%%%%%%%%%%%%%%%%%%%%%%%%%%%%%%%%%%%

\section{Auditarea modificărilor cu \texttt{pgaudit} -- NU MERGE}

În continuare, putem să activăm funcția de auditare din extensia
\texttt{pgaudit}, care va afișa într-un log modificările făcute la baza de
date, filtrate după tipul modificării. De exemplu, se pot înregistra în
log operații de tipul:
\begin{itemize}
\item \texttt{READ}, adică atunci cînd se fac operații de citire de tipul
  \texttt{SELECT, COPY} sau un query;
\item \texttt{WRITE}, cînd se introduc, se modifică sau se șterg valori ori
  alte operații care schimbă conținutul înregistrărilor;
\item \texttt{FUNCTION}, atunci cînd se execută funcții sau blocuri \texttt{DO};
\item \texttt{ROLE}, pentru cînd se atribuie, șterg sau modifică roluri și
  utilizatori noi;
\item \texttt{MISC}, pentru alte tipuri de operații
  (\texttt{DISCARD, VACUUM, FETCH} etc.).
\end{itemize}

Să creăm un tabel care să înregistreze datele pentru audit cînd se citesc valori:
{
  \small
\begin{verbatim}
# create table pgAuditX( id serial, continut text );
# insert into pgAuditX(id, continut) values(DEFAULT, 'test audit');
\end{verbatim}
}

Acum trebuie să activăm extensia de audit care să urmărească operațiile de tip
\texttt{READ}:
{
  \small
\begin{verbatim}
# alter system set pgaudit.log to 'read';
# select pg_reload_conf();
# -- test:
# select continut from pgAuditX;
\end{verbatim}
}

Acum verificăm în terminal, în afara bazei de date:
{
  \small
\begin{verbatim}
$ grep AUDIT postgresql-Sat.log | grep READ

2020-01-18 postgres postgres LOG: 
AUDIT: SESSION,1,1,READ,SELECT,,,SELECT name FROM pgAuditX;,<none>
\end{verbatim}
}


%%% Local Variables:
%%% mode: latex
%%% TeX-master: "../story"
%%% End:
