% ! TEX root = ../semdis.tex
\chapter{Aspecte teoretice}

\section{Premisele semanticii distribuționale}

Așa cum se explică în \cite{boleda} și \cite{boledaherbelot}, semantica
distribuțională (SD) s-a dovedit a fi foarte utilă în științele cognitive,
în cercetări privitoare la asimilarea și atribuirea sensurilor unor cuvinte
din limbajul comun, însă aplicațiile la lingvistica teoretică și computațională
au fost, comparativ, mai rare.

Această ramură a lingvisticii se bazează pe \emph{ipoteza distribuțională},
care afirmă că similaritatea semantică rezultă în similarități ale unor
distribuții statistice. Spus altfel, cuvintele relaționate semantic apar
în contexte similare. Însă SD procedează invers: pornește de la corpusuri de
text și induce reprezentări semantice pe baza distribuțiilor observate.

O remarcă istorică relevantă este că această metodă a apărut în anii 1950 în
Statele Unite, încadrată într-un curent mai general al lingvisticii,
\emph{distribuționalismul}, care urmărea să folosească metode de teoria
distribuțiilor pentru diverse studii cantitative și calitative ale limbajului.
Părinții curentului au fost Leonard Bloomfield și Zellig S.\ Harris, iar teoria
gramaticilor generative formulată de Noam Chomsky este considerat a fi puternic
influențată de aceasta. Totodată, metodele distribuționaliste specifice și-au
adus contribuția inclusiv în didactica limbilor străine.

Intuitiv, ideile SD sînt foarte simple: în multe situații, putem ghici sensul
unui cuvînt din contextele în care acesta este folosit. Astfel, este ca
și cum reprezentăm si interpretăm sensul ca pe o distribuție în fundalul
contextelor lingvistice în care se observă utilizarea cuvintelor-cheie.
Ideea poate fi trasată încă de la filosoful Ludwig Wittgenstein, care
afirma în \emph{Tractatus Logico-Philosophicus} (1922) că \qq{sensul %
  unui cuvînt este utilizarea lui în limbaj}.

Partea dificilă în acest studiu este, însă, să se explice precis ce se
înțelege prin \qq{context}. În varianta naivă (și chiar utilizată în multe
studii), contextul înseamnă pur și simplu vecinătatea cuvîntului-țintă,
implementată cu ajutorul unei ferestre de conținut, i.e.\ un interval de
lungime fixată, centrat în cuvîntul căruia dorim să îi analizăm semantica.

De obicei, metodele matematice cu ajutorul cărora se implementează modelele
dis\-tri\-bu\-ți\-o\-na\-le sînt, oarecum surprinzător, \emph{spațiile vectoriale.}
Dimensiunea spațiilor este legată de cuvintele din context, într-un mod pe
care îl detaliem mai jos, iar astfel, coordonatele depind de cuvintele
co-ocurente. Rezultă că, prin această abordare, vecinătatea în context devine
vecinătate în spațiu, într-un mod cît se poate de concret.

Din punctul de vedere al implementării informatice, modelele distribuționale
pot fi învățate dintr-un corpus într-o manieră nesupervizată.

%%%%%%%%%%%%%%%%%%%%%%%%%%%%%%%%%%%%%%%%%%%%%%%%%%%%%%%%%%%%%%%%%%%%%%
\section{Construcția spațiului vectorial semantic}

Vom prezenta foarte pe scurt modul în care se construiește spațiul vectorial
cu ajutorul căruia se poate face studiul semantic. Mai multe detalii se pot
găsi într-un articol mult mai tehnic pe care îl recomandăm, \cite{turney}.

Concret, se dă un corpus de cuvinte, de exemplu cîteva cărți sau pagini
de Internet din care se poate extrage informație text suficient de multă.
Se dă, de asemenea, un cuvînt-țintă căruia se dorește să i se stabilească
sensul, în cazul acestei abordări. Cuvîntul-țintă se poate schimba pe parcursul
analizei, bineînțeles, însă acum presupunem că avem un cuvînt-țintă fixat la
care ne referim. Se înregistrează co-ocurențele relativ la acest cuvînt-țintă,
adică ce alte cuvinte apar în vecinătatea lui, de exemplu, într-o fereastră
de conținut de 10 cuvinte. Mai precis, pentru fiecare apariție a
cuvîntului-țintă în corpusul folosit, se iau în considerare 5 cuvinte care
urmează după el și 5 cuvinte care-l preced. În majoritatea situațiilor, în
funcție de partea de vorbire care este cuvîntul-țintă, se pot lua în considerare
doar părți de vorbire relevante. De exemplu, dacă vrem să studiem un
cuvînt-țintă care este un substantiv, am putea ignora prepozițiile, punctuația,
pronumele și numele proprii și vom păstra verbele, adjectivele, adverbele.

Odată stabilite și filtrate aceste co-ocurențe, se construiește un vector
asociat cuvîntului-țintă, sortat descrescător după numărul de apariții.
De exemplu, dacă corpusul este, să zicem, Enciclopedia Britannica, iar
cuvîntul-țintă este \qq{coleopter}, vectorul său asociat poate fi de forma:
\[
  v = \begin{pmatrix}
    244 & 111 & 35 & 21 & 5 & 2 & 2 & 1 & 1 & 1 & 1
  \end{pmatrix}^t,
\]
corespunzător co-ocurențelor \emph{aripă, larvă, copac, floare, cîmpie, %
  cald, roșu, frumos, reprezentant, trifoi}. Deducem că cel mai adesea,
cuvîntul \qq{coleopter} a apărut în vecinătatea (contextul) cuvintelor
\emph{aripă, larvă, copac, floare}, celelalte co-ocurențe fiind semnificativ
mai rare.

Atunci, spațiul vectorial (numit și \emph{spațiu semantic}) asociat acestui
cuvînt are dimensiune 10 (dată de fereastra de conținut, de fapt). Evident,
se pot face unele optimizări, precum ignorarea frecvențelor sub un anumit
prag (în cazul nostru, frecvențele $ \leq 5 $ am putea spune că sînt irelevante,
dată fiind mărimea corpusului). O altă optimizare este să se ignore cuvintele
prea generale. De exemplu, dacă întîlnim în preajma cuvîntului-cheie și cuvinte
precum \emph{foarte, mare, mic}, acestea pot fi ignorate, pentru că se potrivesc
în prea multe contexte pentru a fi specifice cuvîntului-cheie.

Odată construit, spațiul semantic poate fi folosit în mai multe moduri. În cele
mai multe cazuri, se asociază spațiului un cvadruplu $ \langle A, S, B, M \rangle $,
unde:
\begin{itemize}
\item $ A $ este o funcție care asociază ponderi relevante co-ocurențelor
  (depinde de la caz la caz);
\item $ S $ este un coeficient de similaritate, de obicei calculat drept
  cosinusul unghiului dintre vectorii asociați cuvintelor, folosind
  produsul scalar euclidian;
\item $ B $ sînt elementele din bază (sau dimensiunea);
\item $ M $ este o transformare a spațiului, de obicei una care poate
  reduce dimensiunea, pentru a face calculele mai relevante.
\end{itemize}

Un exemplu des întîlnit este \emph{analiza semantică latentă} (LSA) , introdusă
în 1997, care folosește:
\begin{itemize}
\item $ A $ --- un indicator numit tf-idf, acronim provenit de la denumirea
  în engleză, \emph{term freq\-uency-inverse distribution frequency}, care
  produce rezultate mai relevante decît numărul de co-ocurențe;
\item $ S $ este cosinusul dintre vectori;
\item $ B $ este o bază spațiului semantic;
\item $ M $ este o descompunere folosind valori singulare, notată folosind
  acronimul englezesc SVD (\emph{singular value decomposition}).
\end{itemize}

Nu insistăm, însă, pe această metodă, deoarece vom întîlni o îmbunătățire
a acesteia într-o secțiune ulterioară, sub forma \emph{analizei semantice %
  explicite}, ESA. Indicăm doar ca referință pentru studiu suplimentar
articolul relevant din
\href{http://www.scholarpedia.org/article/Latent_semantic_analysis}{Scholarpedia}.

Alte abordări mai sofisticate înlocuiesc vectorii cu matrice multidimensionale
(numite și \emph{tensori}), precum și metode complexe de analiza bayesiană,
care calculează funcții de distribuție și pot extrage subiectul despre care
se vorbește în text (eng.\ \emph{topic}).

%%%%%%%%%%%%%%%%%%%%%%%%%%%%%%%%%%%%%%%%%%%%%%%%%%%%%%%%%%%%%%%%%%%%%%
\section{Problema semanticii}

Înainte de a merge mai departe cu metodele distribuționale specifice, împreună
cu studiile de caz, ne oprim pentru o vreme asupra problemei semanticii în sine.
Prezentăm cîteva aspecte istorice și filosofice, dar și literare care arată
de ce semantica limbajului natural este o problemă, care au fost abordările
cele mai cunoscute pentru rezolvarea acestei probleme și de ce, în particular,
cazul poeziilor necesită atenție deosebită.\footnote{Aplicațiile pe care le vom
  prezenta au în vedere poezii scrise în limba engleză. De aceea, vom da
  titlurile și citatele în original, fără a încerca o traducere.}

Începînd cu lucrările lui Gottlob Frege (\emph{Sens și referință}, 1892) și
Alfred Tarski (\emph{Conceptul semantic de adevăr și fundamentele semanticii}, 1944),
semantica formală s-a plasat în contextul teoriei mulțimilor. În acest sens,
cuvintele sînt gîndite precum \emph{concepte}, care au \emph{extensiune}.
De pildă, cuvîntul \qq{pisică} este un concept, în a cărui extensiune
intră toate pisicile din lume, adică identificăm extensiunea conceptului
cu mulțimea tuturor obiectelor care îl instanțiază.

Prin această abordare, se poate vorbi ușor despre adevăr și falsitate, în sensul
că o propoziție care enunță o proprietate a unui obiect (concept) este luată
ca adevărată numai atunci cînd extensiunea conceptului respectiv conține
extensiunea proprietății \emph{la nivel de incluziune între mulțimi}.
De exemplu, propoziția \qq{Toți caii sînt albi} este adevărată dacă întreaga
mulțime a cailor (i.e.\ extensiunea cuvîntului [conceptului] \qq{cal}) este
inclusă în mulțimea obiectelor albe (i.e.\ extensiunea cuvîntului [conceptului]
\qq{alb}).

Mai mult, această abordare ne permite să facem distincția semantică între
construcții precum \qq{soarele amiezii} și \qq{soarele apusului}, chiar dacă,
de fapt, este vorba despre același obiect fizic.

Semantica poeziilor și, de fapt, întreaga întrebare \emph{dacă} poeziile
au semantică sau sens, în general, a fost una îndelung dezbătută. Poeziile
sînt greu de studiat folosind tehnici de lingvistică, deoarece o bună parte
a poeziilor, în special cele moderne și contemporane sau cele foarte vechi,
nu respectă regulile stricte de sintaxă și semantică.

Totuși, semantica distribuțională se poate dovedi o unealtă foarte puternică în
acest sens, deoarece tiparele de distribuție care se remarcă pot fi relevante
și în cazul textelor poetice. Ideea din spatele acestei încrederi este că,
deși în multe situații, limbajul poeziilor nu este reprezentativ pentru
limbajul comun, anumite tipare pe care le distingem în poezii sînt, într-un fel,
latente în limbajul nostru fundamental. De exemplu, sînt multe cuvinte al
căror sens este neschimbat în poezii: cuvîntul \qq{copac}, cel mai probabil
(deși nu cert) apare în poezii cu același sens cu care îl găsim și în
limbajul comun. Aceasta ne arată că, deși limbajul poetic poate fi considerat
unul special, el cu siguranță \emph{nu este complet disociat} de acesta, în
sensul că orice text poetic, oricît de dificil, poate fi recunoscut ca fiind
alcătuit din elemente de limbaj natural și mai mult, în majoritatea cazurilor,
se poate identifica faptul că este alcătuit din construcții cu sens.

Un exemplu (preluat din \cite{herbelot}) de schimb de idei privitor la semantica
poeziilor, prin comparație cu cea a limbajului natural este corespondența
între filosoful Philip Wheelwright (1901-1970) și poeta și criticul literar
Josephine Miles (1911-1985), din anii 1940. Conform lui Wheelwright, în lucrarea
\emph{On the Semantics of Poetry}, limbajul poeziilor este complet diferit de
cel al științei. El consideră că sensul cuvintelor folosite în teorii științifice
se bazează pe concepte, în vreme ce cuvintele din poezii au ceea ce el a numit
\emph{înțeles metalogic}, adică dat de o semantică nebazată pe logică.
Miles a replicat afirmînd că ambiguitate există și în limbajul comun și că nu
este specific poeziilor ca apariția unui cuvînt să depindă de context (spre
deosebire de știință, unde același cuvînt apare de fiecare dată cu același
înțeles, independent de context).

În favoarea lui Wheelwright este, într-adevăr, greu de susținut că expresii
poetice precum:
\begin{itemize}
\item ``{\emph{Music is the exquisite knocking of the blood}}'' (Rupert Brooke);
\item ``\emph{Your huge mortgage of hope}'' (Ted Hughes);
\item ``\emph{Skeleton bells of trees}'' (Avery Slater)
\end{itemize}
au o interpretare folosind teoria mulțimilor.

Dar în același timp, este greu și să combatem teza lui Miles, conform căreia
semantica poeziilor își are rădăcinile în semantica limbajului comun. Într-adevăr,
fără cunoașterea acesteia din urmă, construcțiile poetice sînt lipsite de
orice fel de semantică, iar orice încercare de descifrare este sortită eșecului.

O poziție care iese din această dezbatere este aceea a lui Gerald Bruns,
de la Universitatea Notre Dame din Indiana, SUA. În cartea \cite{bruns},
el afirmă că \emph{poezia este alcătuită din limbaj, dar nu este o utilizare %
  a acestuia}, în sensul că acele cuvinte care apar în poezii nu ar trebui să
fie privite ca fiind definite de contextul poetic. Bruns continuă prin a afirma
că, asemenea lui Wittgenstein (\emph{înțelesul este utilizarea}, eng.\
\emph{meaning is use}), el consideră că extensiunea unui concept nu poate fi
închisă de nicio frontieră, lăsînd, astfel, loc pentru utilizări atipice ale
limbajului, precum poezia.

Aceasta sugerează că semantica, cel puțin în cazul poeziilor, ar trebui ancorată
în context. Acest lucru relaxează frontierele, prin comparație cu teoria
mulțimilor și afirmă că, de exemplu, sensul cuvîntului \qq{pisică} nu mai este
legat de pisicile din lume, ci de modul în care oamenii vorbesc despre pisici.
Regăsim aici primele indicații în direcția semanticii distribuționale:
\qq{responsabilitatea} sensurilor pică pe context și, indirect, pe vorbitorii care
plasează termenul în context.

Teoria distribuțională, însă, nu a prins roade pînă în a doua parte a anilor
1950, cînd dezvoltarea puterii de calcul a contat foarte mult pentru analiza
corpusurilor de text.

Legătura istorică este una foarte puternică: o studentă a lui Wittgenstein,
Margaret Masterman, a fondat Cambridge Language Research Unit (CLRU), una dintre
instituțiile care s-au implicat timpuriu în lingvistica computațională în Marea
Britanie. Tot ea a fost interesată si de aspectele de generare a limbajului, fiind
una dintre cei care au lucrat la o primă variantă software de generare a poeziilor.
În fapt, programul dezvoltat de ei nu producea poezie propriu-zisă, ci oferea
sugestii contextuale pentru a umple spațiile dintr-un haiku (lucrare publicată,
de exemplu, sub forma \cite{masterman}).

Cum puterea de calcul era încă limitată, comparativ cu zilele noastre, echipa
lui Masterman nu a putut analiza corpusuri mari de text în sine. Ei au trebuit
să dezvolte rețele semantice, care să le permită \qq{comunicarea} dintre
diverse seturi, pe baza cărora au extras apoi informații statistice.
Aceasta a fost una dintre mișcările de bază în direcția teoriei distribuționale.


%%% Local Variables:
%%% mode: latex
%%% TeX-master: "../semdis"
%%% End:
