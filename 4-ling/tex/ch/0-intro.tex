% ! TEX root = ../semdis.tex
\chapter*{Introducere}

\indent\indent Proiectul de față prezintă aspecte introductive privitoare la \emph{semantica %
  distribuțională}, îm\-pre\-u\-nă cu două aplicații relevante care se concentrează
pe studiul poeziilor scrise în limba engleză.

Astfel, materialul urmărește în primul rînd prezentarea elementelor fundamentale
de semantică distribuțională, într-o manieră potrivită pentru cei
nefamiliarziați cu domeniul (cum a fost și cazul autorului la începutul
documentării pentru această lucrare). În prima parte a lucrării, așadar, se
vor introduce principalele noțiuni teoretice privitoare la această metodă de
studiu empiric al semanticii, folosind obiecte matematice surprinzătoare,
anume spații vectoriale. Ne vom limita, însă, la noțiunile ce vor fi de
folos efectiv în restul lucrării, iar pentru mai multe detalii, vom indica
referințe bibliografice relevante.

Metodele distribuționale de atribuire a semanticii se vor folosi apoi pentru
expunerea a două exemple concrete, ambele legate de texte poetice. Primul
exemplu studiază cîteva poezii moderne și contemporane din literatura engleză,
pentru a justifica faptul că, deși semantica poetică nu este mereu la îndemîna
tuturor celor familiarizați doar cu limbajul comun, se pot identifica suficiente
indicii că ea este bazată pe acest limbaj. Se urmărește, în particular,
\emph{coerența subiectelor} din textele poetice, adică, odată identificate
tipare privitoare la subiectul poeziilor, sîntem interesați dacă vocabularul
din contextul subiectului respectiv este relevant pentru subiectul în sine.

A doua aplicație pe care o prezentăm încearcă să răspundă la o controversă
privitoare la opera poetică a lui Lord Byron și cea a lui Thomas Moore. Cum
ambii au fost poeți reprezentativi pentru un anume curent literar ---
orientalismul romantic --- se pot remarca suficiente conexiuni între poeziile
celor doi. Folosind metode de semantică distribuțională, se încearcă răspunsul
la întrebarea dacă aceste conexiuni sînt inerente, cumva impuse și induse de
încadrarea în (sub)genul literar respectiv sau într-adevăr se poate vorbi despre
imitație.

\vspace{0.3cm}

Înainte de a începe efectiv prezentarea, mai menționăm încă o dată că lucrarea
se dorește a fi una introductivă și de ansamblu, astfel că majoritatea
subiectelor incluse sînt tratate la un nivel elementar. Detalii suplimentare
pot fi găsite atît în referințele indicate explicit, cît și în bibliografiile
acestora.


%%% Local Variables:
%%% mode: latex
%%% TeX-master: "../semdis"
%%% End:
