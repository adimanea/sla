% ! TEX root = ../semdis.tex

\chapter{Aplicație: Legăturile între \\ Lord Byron și Thomas Moore}

\section{Introducere și motivație}

Ideea articolului \cite{agg}, din care preluăm aplicația de față, este să
utilizeze un model de semantică distribuțională pentru a verifica dacă
poetul Lord Byron a fost imitat de Thomas Moore, așa cum consideră unii
critici sau, de fapt, este vorba despre o caracteristică generală a
vocabularului poeziilor din secolul al XIX-lea (sau, mai precis, vocabularul
specific sub-genului literar în care s-au încadrat cei doi),
caracteristică pe care nu puteau să nu o aibă și poeziile lui Byron și Moore.

În perioada 1813-1817, poeții prieteni Lord Byron și Thomas Moore au
scris o serie de poezii care sînt privite astăzi drept reprezentative
pentru \emph{orientalismul romantic}, o subcategorie a literaturii romantice,
caracterizată prin temele și plasarea în contexte orientale și din
Orientul Mijlociu. Printre poeziile publicate de cei doi, s-au descoperit
foarte multe coincidențe, precum desfășurări similare de acțiuni, decoruri și
nume de personaje similare.

Articolul \cite{agg} își propune să folosească o metodă empirică nouă
pentru a încerca să clarifice aceste coincidențe. De exemplu, ar putea
răspunde la întrebarea dacă cumva orientalismul romantic este, în sine,
un gen de poezie care folosește un vocabular limitat și atunci, similaritățile
sînt inevitabile. Totodată, întrebarea mai generală poate fi și dacă se poate
caracteriza un gen literar prin tocmai vocabularul său.

%%%%%%%%%%%%%%%%%%%%%%%%%%%%%%%%%%%%%%%%%%%%%%%%%%%%%%%%%%%%%%%%%%%%%%
\section{Metodologia și experimentul}

Autorii au utilizat o metodă de semantică distribuțională prin \emph{analiză %
  semantică explicită} (ESA), ca alternativă la analiza latentă, prezentată
în prima parte a lucrării. ESA a fost introdusă în \cite{esa}. În varianta
originală, metoda folosește texte din Wikipedia pentru a stabili unele sensuri,
iar apoi, textul dat este interpretat prin intermediul conexiunilor deja
stabilite cu ajutorul Wikipedia. Informal, este ca și cum clasificatorul
semantic se \qq{antrenează} cu Wikipedia, pentru ca apoi să rezolve problema
textului dat.

ESA permite reprezentarea unor vectori de sens într-un spațiu vectorial cu
un număr foarte mare de dimensiuni, dată fiind complexitatea și diversitatea
articolelor din Wikipedia, iar coeficientul de relaționare semantică se
calculează din nou folosind cosinusul unghiului celor doi vectori care
reprezintă cuvintele analizate.

Experimentul a constat prin alegerea a 4 poezii lungi (poeme narative, în fapt,
de cîteva mii de versuri) publicate de Byron între 1813 și 1814 și 4 poezii lungi
publicate de Moore în 1817. Aceste poezii au fost împărțite în grupuri de cîte
227 versuri în cazul lui Byron și 246 de versuri în cazul lui Moore. După aceea,
s-a calculat scorul ESA pentru grupurile din poeziile lui Byron și același
lucru pentru operele lui Moore.

Însă abordarea a fost ceva mai complexă decît în modelul original ESA. S-au
folosit, de fapt, 2 modele: unul care folosește Wikipedia, ca în articolul
original, iar altul a folosit un corpus dat de 892 de poezii lungi (poeme
narative) din secolele XVIII și XIX. Pentru ambele corpusuri s-a folosit
ponderea tf-idf, pe care am introdus-o în primul capitol.

%%%%%%%%%%%%%%%%%%%%%%%%%%%%%%%%%%%%%%%%%%%%%%%%%%%%%%%%%%%%%%%%%%%%%%
\section{Rezultate si concluzii}

Rezultatele, sortate după coeficienții de relaționare, au fost clasificate
în \qq{foarte legate}, \qq{posibil legate} și \qq{nelegate}. În prima
categorie au intrat aproximativ 1000 de perechi de versuri. S-au analizat
(manual, uman) 15 perechi obținute cu metoda Wikipedia și 15 perechi
obținute cu cealaltă metodă.

Drept concluzii, autorii remarcă faptul că, pe de o parte, este surprinzător
că această metodă de analiză automată a găsit similarități exact în ce
privește conceptele așteptate. Este vorba exact despre elementele remarcate
de critici, specifice curentului literar: personaje și decoruri. Astfel,
s-a constatat că perechile de versuri care au intrat în categoria
\qq{foarte legate}, atît în clasificarea folosind Wikipedia, cît și cu
cealaltă metodă, sînt legate de sentimente și de cadre naturale.
Ambele concepte caracterizează într-o măsură destul de mare curentul
orientalismului romantic și se poate deduce de aici că, într-adevăr,
inspirația dintre cei doi, dacă a existat, s-a manifestat exact unde se
bănuiește.

Pe de altă parte, atît corpusul de texte analizate, cît și rezultatele sînt,
din punct de vedere cantitativ, nu foarte reprezentative. Pentru o concluzie
mai precisă, idei de îmbunătățire pot privi atît rafinarea modelelor ESA,
cît și îmbogățirea corpusului de texte studiate.


%%% Local Variables:
%%% mode: latex
%%% TeX-master: "../semdis"
%%% End:
