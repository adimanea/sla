\documentclass[xcolor=dvipsnames]{beamer}

% font setup
\usepackage{libertine}
\renewcommand*\familydefault{\sfdefault}    % Linux Libertine = default sans serif
\usepackage{inconsolata}                    % Inconsolata = monospaced
\usepackage[utf8]{inputenc}
\usepackage[T1]{fontenc}
\usepackage{tikz}
\usetikzlibrary{positioning}
\usepackage[style=german]{csquotes}
\newcommand\qq{\enquote}

\usepackage{algorithm}
\usepackage{algpseudocode}
\makeatletter
\renewcommand{\ALG@name}{Algoritm}
\makeatother
\algrenewcommand\algorithmicprocedure{\textbf{procedură}}
\algrenewcommand\algorithmicend{\textbf{final}}

% Graphics and other packages
\usepackage[romanian]{babel}
\usepackage{graphicx}
\addto\captionsromanian{\renewcommand{\figurename}{Ilustrație}}
\usepackage{caption}
\usepackage{subcaption}
\usepackage[style=german]{csquotes}

% Custom macros
\newcommand{\bloc}[3]{\begin{bl}<#1->{{\large\color{Gray}{\hrulefill}}\\ \color{bleumarin}{\large \emph{#2}}}\\ \vspace*{-2mm}{\color{Gray}{\hrulefill}}\\ #3 \end{bl}} 
\newcommand{\fr}[1]{\frame{#1}}
\newcommand{\ft}[1]{\frametitle{\color{bleumarin}{\hfill #1 \hfill}}}
\newcommand{\lin}[3]{\uncover<#1->{\alert<#1>{#2}}{\vspace*{#3 ex}}}
\newcommand{\ite}[2]{\uncover<#1->{\alert<#1>{\item #2}}}
\newcommand{\vs}[1]{\vspace*{#1 ex}}
\definecolor{bleumarin}{RGB}{30,30,150} 
\definecolor{firebrick}{RGB}{178,34,34}

% Theme setup
\useoutertheme{shadow} 
\usetheme{CambridgeUS} 
\usecolortheme[named=bleumarin]{structure} 
\useoutertheme[compress]{smoothbars}

% Theme finetuning
\setbeamertemplate{items}[ball]
\setbeamertemplate{blocks}[rounded][shadow=true]
\setbeamertemplate{navigation symbols}{}
\setbeamertemplate{headline}{}  


%%%%%%%%%%%%%%%%%%%%%%%%%%%%%%%%%%%%%%%%%%%%%%%%%%%%%%%%%%%%%%%%%%%%%%
% TITLE PAGE
\title[ML4SymbMath]{Învățare automată \\ pentru matematică simbolică}
\author{Adrian Manea}
\institute{510, SLA}

\date{}

\begin{document}

\maketitle

% SLIDES START HERE
%%%%%%%%%%%%%%%%%%%%%%%%%%%%%%%%%%%%%%%%%%%%%%%%%%%%%%%%%%%%%%%%%%%%%%
\fr{
    \ft{Scopul și metoda}

    \lin{1}{Generarea expresiilor matematice folosind tehnici de NLP.}{4}

    \lin{2}{Odată obținut un corpus de expresii matematice, pot fi
    testate dacă satisfac ecuații complicate (ODE, PDE, int).}{4}

    \lin{3}{Metoda: traducere automată (seq2seq) + beam search (\cite{chollet}).}{4}

    \lin{4}{Expresii în forma prefixată (poloneză):
        \[
            2 + 3 \cdot (5 + 2) \mapsto \texttt{[+ 2 * 3 + 5 2]}.
    \]}{4}
}

\fr{
    \ft{Scopul și metoda}
\begin{figure}[!htbp]
    \centering
    \begin{tikzpicture}[level distance=0.5cm,
        level 1/.style={sibling distance=2cm},
        level 2/.style={sibling distance=2cm},
        level 3/.style={sibling distance=2cm},
        level 4/.style={sibling distance=2cm}]
        \node {\texttt{+}}
            child {node {\texttt{2}}}
            child {node {\texttt{*}}
                child {node {\texttt{3}}}
                child {node {\texttt{+}}
                    child {node {\texttt{5}}}
            child {node {\texttt{2}}}}};
    \end{tikzpicture}
    \caption{Arbore binar pentru expresia \texttt{[+ 2 * 3 + 5 2]}}
    \label{fig:mat-polish1}
\end{figure}
}

\fr{
    \ft{Scopul și metoda}
\begin{figure}[!htbp]
    \centering
    \small
    \begin{tikzpicture}[level distance=1cm,
        level 1/.style={sibling distance=5cm},
        level 2/.style={sibling distance=2cm},
        level 3/.style={sibling distance=1cm},
        level 4/.style={sibling distance=0.5cm}]
        \node {$-$}
            child {node {$\partial$}
                child {node {$\partial$}
                    child {node {$u$}}
                    child {node {$x$}}
                }
            child {node {$x$}}}
            child {node {$\times$}
                child {node {$/$}
                    child {node {$1$}}
                    child {node {\texttt{pow}}
                        child {node {$c$}}
                child {node {$2$}}}}
                child {node {$\partial$}
                    child {node {$\partial$}
                        child {node {$u$}}
                    child {node {$t$}}}
            child {node {$t$}}}};
    \end{tikzpicture}
    \caption{Arbore binar pentru expresia %
    $ \dfrac{\partial^2 u}{\partial x^2} - \dfrac{1}{c^2} \dfrac{\partial^2 u}{\partial t^2}$}
    \label{fig:arbore-pder}
\end{figure}
}

\fr{
    \ft{Probleme și soluții: EDO \& EI}
    
    \lin{1}{\textbf{Generare directă} (\texttt{FWD}):
        \[
            LC \to f(x) \xrightarrow{CAS} \int f(x) dx.
    \]}{1}

    \lin{2}{\textbf{Generare inversă} (\texttt{BWD}):
        \[
            LC \to f(x) \xrightarrow{CAS} f'(x).
    \]}{1}
    
    \lin{3}{\textbf{Generare inversă și integrare prin părți} (\texttt{IBP}):
        \[
            \int Fg = FG - \int fG.
    \]}{1}

    \lin{4}{
        \[
            F, G \xrightarrow{\texttt{BWD}} f, g \rightarrow fG \xrightarrow{\texttt{FWD}} Fg.
    \]}{1} 
}

\fr{
    \ft{Rezultate, concluzii și critici \cite{davis2019use}}

    \lin{1}{Rezultatele sînt comparabile cu Mathematica, Matlab, Maple ($\pm 10\%$).}{4}

    \lin{2}{Nu \qq{știe matematică}: simplificări, expresii echivalente,
    \underline{expresii fără sens}.}{4}

    \lin{3}{Modelul nu este verificat formal (se speră la dezvoltarea SymPy).}{4}

    \lin{4}{Algoritmii se bazează pe CAS $ \Rightarrow $ comparația nu are sens și
    elementul de noutate este minimizat.}{0}
}
    
%%%%%%%%%%%%%%%%%%%%%%%%%%%%%%%%%%%%%%%%%%%%%%%%%%%%%%%%%%%%%%%%%%%%%%

% Bibliography
\fr{
    \ft{Bibliografie}
    \bibliography{symbolic-prez.bib}
    \bibliographystyle{apalike}
    \nocite{*}
}

\end{document}

