%! TEX root = ../symbolic.tex

\chapter{Probleme și soluții}

Autorii și-au propus să ajute la rezolvarea ecuațiilor diferențiale
(eventual cu derivate parțiale) și a ecuațiilor integrale. Ideea de
bază este că astfel de probleme sînt foarte greu de rezolvat folosind
instrumente matematice standard, însă operațiile în sine, de derivare
și integrare, sînt relativ simple, chiar și într-o formă algoritmică.

De aceea, abordarea propusă este să se genereze seturi foarte mari
de funcții și expresii matematice, cu zeci de milioane de membri, iar
fiecare element să fie apoi testat dacă satisface ecuația dată. Din nou,
ideea de bază este că, în locul rezolvării ecuației propriu-zise, se poate
doar deriva sau integra, ambele proceduri fiind relativ ieftine din
punct de vedere computațional.

Pe lîngă dificultatea eliminării expresiilor fără sens, apare, totodată,
și problema funcțiilor ale căror primitive sînt foarte greu sau imposibil
de calculat, precum $ \exp(x^2) $ sau $ \dfrac{\sin x}{x} $. Aceste
neajunsuri sînt asumate de autori.

%%%%%%%%%%%%%%%%%%%%%%%%%%%%%%%%%%%%%%%%%%%%%%%%%%%%%%%%%%%%%%%%%%%%%%
\section{Generarea funcțiilor și a primitivelor}

Autorii propun trei metode de abordare:
\begin{enumerate}[(1)]
    \item \textbf{Generarea directă} (\texttt{FWD}): se generează funcții
        aleatorii cu pînă la $ n $ operatori și se calculează primitivele
        lor folosind software specializat (Computer Algebra System, CAS).
        Dacă programul CAS nu poate calcula primitiva, se elimină funcția
        generată din mulțime;
    \item \textbf{Generarea inversă} (\texttt{BWD}): se generează
        funcții care se derivează folosind un CAS sau altă metodă, care
        este mult mai ieftină computațional și se poate calcula mereu.
        Astfel, se înregistrează rezultatul invers: funcția generată
        de program va fi primitiva celei care se calculează prin derivare;
    \item \textbf{Generare inversă și integrare prin părți} (\texttt{IBP}):
        folosind formula de integrare prin părți:
        \[
              \int Fg = FG - \int fG,
        \]
        cu notațiile standard $ F' = f $ și $ G' = g $, putem face o combinație
        a metodelor de mai sus: se dau două funcții generate $ F $ și $ G $.
        Se calculează derivatele, folosind \texttt{BWD} și se verifică dacă
        $ fG $ se găsește deja în mulțime. Dacă da, știm deja integrala sa
        (din metoda \texttt{FWD}) și atunci putem obține integrala $ Fg $ din
        formula de integrare prin părți. Similar dacă $ Fg $ a fost deja
        generată, îi știm integrala din \texttt{FWD} și putem obține integrala $ fG $.
\end{enumerate}

%%%%%%%%%%%%%%%%%%%%%%%%%%%%%%%%%%%%%%%%%%%%%%%%%%%%%%%%%%%%%%%%%%%%%%
\section{Generarea soluțiilor EDO I}

Se pot genera ecuații diferențiale ordinare de ordinul I în felul următor.
Se pornește cu o funcție $ F(x, y) $ astfel încît ecuația (problema Cauchy) 
$ F(x, y) = c $ să poată fi rezolvată pentru $ y $, dată o constantă $ c $.
Cu alte cuvinte, există o funcție $ f $ astfel încît pentru orice $ (x, c)$,
are loc $ F(x, f(x, c)) = c $.

Calculăm derivata în raport cu $ x $ a acestei expresii și se obține:
\[
    \frac{\partial F(x, f_c(x))}{\partial x} + f'_c(x) \cdot %
    \frac{\partial F(x, f_c(x))}{\partial y} = 0,
\]
unde am notat prin $ f_c $ funcția $ x \mapsto f(x, c) $. Rezultă că,
pentru orice constantă $ c $, funcția $ f_c $ este o soluție pentru
ecuația diferențială de ordinul întîi:
\[
      F_x + y' F_y = 0,
\]
unde $ F_x = \dfrac{\partial F(x, y)}{\partial x} $ și similar pentru $ F_y $.

De exemplu:
\begin{enumerate}[(1)]
    \item Se generează o funcție aleatorie: $ f(x) = x \log(c/x) $;
    \item Se rezolvă ecuația $ c = x\exp\left( \dfrac{f(x)}{x} \right) = F(x, f(x)) $;
    \item Se calculează derivata în raport cu $ x $:
        \[
            \exp\left( \frac{f(x)}{x} \right) \left(1 + f'(x) - \frac{f(x)}{x} \right) = 0;
        \]
    \item Prin simplificare, ajungem la $ xy' - y + x = 0 $.
\end{enumerate}

\begin{remark}\label{rk:ode2}
    Se poate proceda similar pentru ecuații diferențiale de ordinul al doilea,
    rezultat pe care îl omitem, indicînd referința \cite[\S 3.3]{lample2019deep}.
\end{remark}

