%! TEX root = ../symbolic.tex

\chapter{Preliminarii}

\section{Introducere și motivație}

Articolul își propune să introducă o metodă prin care, folosind
învățarea automată, să se genereze expresii matematice de diverse
complexități și forme. Utilizarea acestor expresii este, de exemplu,
pentru rezolvarea problemelor matematice care necesită, din motive
teoretice, căutarea unor soluții într-un spațiu foarte mare.

Mai mult decît atît, însăși problema generării expresiilor simbolice
este una complicată pentru rețele neuronale, iar abordarea din articolul
\cite{lample2019deep} pe care îl prezentăm este una specifică traducerilor
automate sau a procesării limbajului natural (NLP).

Autorii remarcă faptul că tehnicile folosite în NLP pot fi utile și
în cazul expresiilor matematice simbolice. Dacă sistemele de tip
\emph{computer algebra} de obicei rezolvă probleme matematice prin algoritmi
foarte sofisticați, oamenii lucrează prin identificarea tiparelor în
expresiile matematice. Nicio rețea neuronală existentă în prezent nu a
fost folosită pentru identificarea tiparelor în expresii matematice,
însă.

%%%%%%%%%%%%%%%%%%%%%%%%%%%%%%%%%%%%%%%%%%%%%%%%%%%%%%%%%%%%%%%%%%%%%%

\section{Expresii matematice prin arbori sintactici}

Autorii propun utilizarea unei metode de traducere de tip \emph{sequence to %
sequence} (seq2seq, pe scurt), descrisă cu tot cu implementare în Python
în \cite{chollet}. Pentru aceasta, expresiile matematice sînt descompuse
folosind arbori de sintaxă, iar reprezentarea lor se face folosind
așa-numita \emph{formă poloneză} cu prefix. Astfel, expresia scrisă în 
mod clasic prin:
\[
    2 + 3 \cdot (5 + 2)
\]
va fi descrisă în notația poloneză prin \texttt{[+ 2 * 3 + 5 2]}, iar
prin arbore binar în forma din figura \ref{fig:mat-polish1}.

\begin{figure}[!htbp]
    \centering
    \begin{tikzpicture}[level distance=0.5cm,
        level 1/.style={sibling distance=2cm},
        level 2/.style={sibling distance=2cm},
        level 3/.style={sibling distance=2cm},
        level 4/.style={sibling distance=2cm}]
        \node {\texttt{+}}
            child {node {\texttt{2}}}
            child {node {\texttt{*}}
                child {node {\texttt{3}}}
                child {node {\texttt{+}}
                    child {node {\texttt{5}}}
            child {node {\texttt{2}}}}};
    \end{tikzpicture}
    \caption{Arbore binar pentru expresia \texttt{[+ 2 * 3 + 5 2]}}
    \label{fig:mat-polish1}
\end{figure}

Similar se pot coda și expresii mai complicate, precum, de exemplu (v.\ fig. \ref{fig:arbore-pder}):
\[
    \frac{\partial^2 u}{\partial x^2} - \dfrac{1}{c^2} \frac{\partial^2 u}{\partial t^2}.
\]


\begin{figure}[!htbp]
    \centering
    \begin{tikzpicture}[level distance=1cm,
        level 1/.style={sibling distance=7cm},
        level 2/.style={sibling distance=5cm},
        level 3/.style={sibling distance=3cm},
        level 4/.style={sibling distance=1.5cm}]
        \node {$-$}
            child {node {$\partial$}
                child {node {$\partial$}
                    child {node {$f$}}
                    child {node {$x$}}
                }
            child {node {$x$}}}
            child {node {$\times$}
                child {node {$/$}
                    child {node {$1$}}
                    child {node {\texttt{pow}}
                        child {node {$u$}}
                child {node {$2$}}}}
                child {node {$\partial$}
                    child {node {$\partial$}
                        child {node {$f$}}
                    child {node {$t$}}}
            child {node {$t$}}}};
    \end{tikzpicture}
    \caption{Arbore binar pentru expresia %
    $ \dfrac{\partial^2 u}{\partial x^2} - \dfrac{1}{c^2} \dfrac{\partial^2 u}{\partial t^2}$}
    \label{fig:arbore-pder}
\end{figure}

Cu ajutorul arborilor, înțelesurile expresiilor sînt imediate și în plus,
folosind notația poloneză, avem o legătură directă între expresia scrisă
în limbaj matematic și arborele său sintactic.

Evident, o primă problemă remarcată este că un arbore sintactic nu poate
identifica expresiile matematice fără sens, precum împărțirile la zero
sau expresii de forma $ \log 0 $ sau $ \sqrt{-\ln (-3)} $.

Însă scopul principal al articolului este să genereze un corpus de funcții
și expresii matematice complexe, care apoi pot fi testate dacă sînt soluții
pentru probleme complicate, precum ecuații cu derivate parțiale sau ecuații
integrale.

Inferența formulelor generate din arbori se face folosind o tehnică
specifică, des utilizată împreună cu metoda seq2seq, numită
\emph{beam search}, cu mărimi de 1, 10 și 50.

Reluăm pe scurt prezentarea metodei, din \cite{chollet}. Cazul general
este acela al traducerilor automate, deci cînd se pornește cu o expresie
de o anumită lungime și se dorește obținerea unei expresii care poate avea
o altă lungime.

Pentru aceasta, avem nevoie de:
\begin{itemize}
    \item O rețea neuronală recursivă (RNN), care \qq{codează} secvența de
        intrare și returnează o stare internă, specifică. Nu vom fi
        interesați de codarea propriu-zisă, ci de starea specifică returnată,
        folosită mai departe ca un context;
    \item O rețea neuronală recursivă de \qq{decodare}, care va fi
        antrenată să prezică următorul caracter din secvență, dacă i se
        dau caracterele anterioare.
\end{itemize}

Pentru \emph{inferență}, adică atunci cînd se dorește să se decodeze
secvențe de intrare necunoscute, procedura este cumva diferită:
\begin{enumerate}[(1)]
    \item Se codează secvența de intrare folosind vectori de stare;
    \item Se pornește cu o secvența de lungime 1, alcătuită doar
        din primul caracter al secvenței de intrare;
    \item Se transmit vectorii de stare și primul caracter către
        decodor, care va produce predicții pentru caracterul ce
        urmează;
    \item Se estimează următorul caracter folosind metoda
        argmax și se adaugă rezultatul la șir;
    \item Se repetă pînă se ajunge fie la ultimul caracter din
        șir, fie la o lungime prestabilită.
\end{enumerate}

Astfel că, în cazul seq2seq aplicat expresiilor matematice reprezentate
ca arbori binari, se generează un arbore de căutare de tip \emph{breadth-first}
și se caută succesorii, sortîndu-i după cost, dar păstrînd doar un
număr de succesori egal cu mărimea razei de căutare (1, 10, 50 în
cazul curent).

De îndată ce s-a stabilit metoda de generare și de înțelegere a expresiilor
generate (i.e.\ folosind arbori binari și \emph{beam search}), se poate analiza exact
dimensiunea spațiului soluțiilor, care va conține:
\begin{itemize}
    \item arbori cu cel mult $ n $ noduri, pentru un anume $ n $;
    \item o multime de operatori unari (e.g.\ $ \sin, \cos, \exp, \log $);
    \item o mulțime de operatori binari (e.g.\ $ +, -, \times $, \texttt{pow});
    \item o multime de frunze care să conțină variabile, constante și numere
        reale (sau întregi, pentru simplitate).
\end{itemize}

Mulțimile operatorilor unari, respectiv binari au cardinalele bine stabilite,
determinate de numerele Catalan, respectiv Schr\"{o}der, care se pot găsi,
de exemplu, și în \cite{sloane}.

În continuare, prezentăm problemele abordate și soluțiile propuse, folosind
instrumentele teoretice de mai sus.
