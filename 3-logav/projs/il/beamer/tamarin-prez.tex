\documentclass[xcolor=dvipsnames,handout]{beamer}

% font setup
\usepackage{libertine}
\renewcommand*\familydefault{\sfdefault}    % Linux Libertine = default sans serif
\usepackage{inconsolata}                    % Inconsolata = monospaced
\usepackage[utf8]{inputenc}
\usepackage[T1]{fontenc}
\usepackage{tikz}
\usetikzlibrary{positioning}

\usepackage{algorithm}
\usepackage{algpseudocode}
\makeatletter
\renewcommand{\ALG@name}{Algoritm}
\makeatother
\algrenewcommand\algorithmicprocedure{\textbf{procedură}}
\algrenewcommand\algorithmicend{\textbf{final}}

% Graphics and other packages
\usepackage[romanian]{babel}
\usepackage{graphicx}
\addto\captionsromanian{\renewcommand{\figurename}{Ilustrație}}
\usepackage{caption}
\usepackage{subcaption}
\usepackage[style=german]{csquotes}

% Custom macros
\newcommand{\bloc}[3]{\begin{bl}<#1->{{\large\color{Gray}{\hrulefill}}\\ \color{bleumarin}{\large \emph{#2}}}\\ \vspace*{-2mm}{\color{Gray}{\hrulefill}}\\ #3 \end{bl}} 
\newcommand{\fr}[1]{\frame{#1}}
\newcommand{\ft}[1]{\frametitle{\color{bleumarin}{\hfill #1 \hfill}}}
\newcommand{\lin}[3]{\uncover<#1->{\alert<#1>{#2}}{\vspace*{#3 ex}}}
\newcommand{\ite}[2]{\uncover<#1->{\alert<#1>{\item #2}}}
\newcommand{\vs}[1]{\vspace*{#1 ex}}
\definecolor{bleumarin}{RGB}{30,30,150} 
\definecolor{firebrick}{RGB}{178,34,34}

% Theme setup
\useoutertheme{shadow} 
\usetheme{CambridgeUS} 
\usecolortheme[named=bleumarin]{structure} 
\useoutertheme[compress]{smoothbars}

% Theme finetuning
\setbeamertemplate{items}[ball]
\setbeamertemplate{blocks}[rounded][shadow=true]
\setbeamertemplate{navigation symbols}{}
\setbeamertemplate{headline}{}  


%%%%%%%%%%%%%%%%%%%%%%%%%%%%%%%%%%%%%%%%%%%%%%%%%%%%%%%%%%%%%%%%%%%%%% 
% TITLE PAGE
\title{\textsc{Tamarin}}
\author{Adrian Manea}
\institute{510, SLA}

\date{}

\begin{document}

\maketitle

% SLIDES START HERE
%%%%%%%%%%%%%%%%%%%%%%%%%%%%%%%%%%%%%%%%%%%%%%%%%%%%%%%%%%%%%%%%%%%%%% 
\fr{
  \ft{Teorie}

  \lin{1}{Analizăm formal protocoalele de securitate, pe baza specificării:}{0}
  \begin{itemize}
    \ite{2}{acțiunilor protocolului;}
    \ite{3}{adversarului (tip Dolev--Yao);}
    \ite{4}{proprietăților dorite (leme).}
  \end{itemize}
}

\begin{frame}[fragile]
  \ft{Sintaxă (stări inițiale)}

  \lin{1}{Declararea funcțiilor definite sau importate:}{0}
  {\footnotesize
\begin{verbatim}
builtins: hashing, asymmetric-encryption
functions: mac/2, g/0, shk/0 [private]
\end{verbatim}
  }

  \lin{2}{Regula de înregistrare a cheii publice:}{0}

  {\footnotesize
\begin{verbatim}
rule Register_pk: [ Fr(~ltk) ] --> [ ~Ltk($A, ~ltk), ~Pk($A, pk(~ltk)) ]
\end{verbatim}
  }

  \lin{3}{Tipuri (sorturi):}{0}

  \begin{itemize}
    \ite{4}{\texttt{\~{}x} = variabilă nouă = \texttt{x:fresh};}
    \ite{5}{\texttt{\$x} = variabilă publică = \texttt{x:pub};}
    \ite{6}{\texttt{\#i} = variabilă temporală = \texttt{i:temporal};}
    \ite{7}{\texttt{m} = mesaj = \texttt{m:msg}.}
  \end{itemize}

\end{frame}

\begin{frame}[fragile]
  \ft{Sintaxă (stări inițiale)}
  \lin{1}{Propagarea cheii în rețea:}{0}
  {\footnotesize
\begin{verbatim}
rule Get_pk: [ !Pk(A, pubkey) ] --> [ Out(pubkey) ]
/* !Pk = acțiune persistentă, se poate repeta oricînd e nevoie */
\end{verbatim}
  }

  \lin{2}{Compromiterea cheii (dezvăluirea de către adversar) este reprezentată
    ca un fapt în acțiune = urmă $ \neq $ stare:}{0}

  {\footnotesize
\begin{verbatim}
rule Reveal_ltk: [ !Ltk(A, ltk) ] --[ LtkReveal(A) ]-> [ Out(ltk) ]
\end{verbatim}
  }
\end{frame}

\begin{frame}[fragile]
  \ft{Sintaxă (funcționare)}

  \lin{1}{Creează fir pentru client, alege server non-determinist:}{0}

  {\footnotesize
\begin{verbatim}
rule Client_1: [ Fr(~k), ~Pk($S, pkS) ] --> [ Client_1($S, ~k), Out(aenc(~k, pkS)) ]
\end{verbatim}
  }

  \lin{2}{Folosind cheia, se menționează pe urmă că s-a stabilit o legătură cu
    serverul \texttt{S}, pe baza cheii \texttt{k}:}{0}

  {\footnotesize
\begin{verbatim}
rule Client_2: [ Client_1(S, k), In(h(k)) ] --[ SessKeyC(S,k) ]-> []
\end{verbatim}
  }

  \lin{3}{Răspunsul serverului (menționat tot pe urmă):}{0}

  {\footnotesize
\begin{verbatim}
rule Serv_1: [ ~Ltk($S, ~ltkS), In(request) ]
     --[ AnswerRequest($S, adec(request, ~ltkS)) ]->
    [ Out(h(adec(request, ~ltkS))) ]
\end{verbatim}
  }
\end{frame}

\begin{frame}[fragile]
  \ft{Proprietăți (leme)}

  {\footnotesize
\begin{verbatim}
lemma Client_session_key_secrecy:
  " not( /* nu este cazul ca */
        Ex S k #i #j.
          /* clientul a stabilit o cheie de sesiune k cu serverul S */
          SessKeyC(S, k) @ #i
          /* iar adversarul să cunoască cheie */
          & K(k) @ #j
          /* fără a fi aplicat o dezvăluire pe S */
          & not(Ex #r. LtkReveal(S) @ r)
        )
  "
\end{verbatim}
  }
\end{frame}

\begin{frame}[fragile]
  \ft{Proprietăți (leme)}

  {\footnotesize
\begin{verbatim}
lemma Client_auth:
  " /* Pentru toate cheile de sesiune 'k'
       stabilite de clienți cu serverul 'S' */
    ( All S k #i. SessKeyC(S, k) @ #i
        ==>
      /* există un server care a răspuns la cerere */
        ( (Ex #a. AnswerRequest(S, k) @ a)
      /* sau adversarul a aflat cheia înainte de a fi stabilită */
          | (Ex #r. LtkReveal(S) @ r & r < i)
        )
    )
  "
\end{verbatim}
  }

  {\footnotesize
\begin{verbatim}
/* Execuție nevidă */
lemma Client_session_key_honest_setup:
  exists-trace
    " Ex S k #i.
        SessKeyC(S, k) @ #i
        & not(Ex #r. LtkReveal(S) @ r)
    "
\end{verbatim}
}
\end{frame}

\begin{frame}[fragile]
  \ft{Proprietate suplimentară: autorizare injectivă}

  {\footnotesize
\begin{verbatim}
lemma Client_auth_injective:
  " /* Pentru toate cheile de sesiune 'k' stabilite cu serverul 'S' */
    ( All S k #i. SessKeyC(S, k) @ #i
        ==>
        /* există un server care a răspuns */
      ( (Ex #a. AnswerRequest(S, k) @ a
        /* și niciun alt client n-a avut exact aceeași cerere */
        & (All #j. SessKeyC(S, k) @ #j ==> #i = #j)
        ) /* sau adversarul a aflat cheia înainte de a fi stabilită */
        | (Ex #r. LtkReveal(S) @ r & r < i)
      )
    )
  "
\end{verbatim}
  }
\end{frame}

%%%%%%%%%%%%%%%%%%%%%%%%%%%%%%%%%%%%%%%%%%%%%%%%%%%%%%%%%%%%%%%%%%%%%% 

% Bibliography
\fr{
  \ft{Bibliografie}
  \bibliography{tamarin-prez.bib}
  \bibliographystyle{apalike}
  \nocite{*}
}
\end{document}

