%! TEX root = ../tamarin.tex

\chapter{Exemple}

\section{Sintaxă și elemente specifice}

Preluăm din articolul \cite{DBLP:journals/siglog/BasinCDS17} cîteva
elemente introductive privitoare la funcționarea Tamarin.

Studiul de caz prezentat este acela al schimbului de chei Diffie-Hellman (DH).

\begin{remark}\label{rk:sintaxa}
  Sintaxa folosită în cele de mai jos este imprecisă, deoarece am
  preferat lizibilitatea și claritatea. Sintaxa exactă este dată în
  fișierele-sursă conținute în directorul proiectului, în interiorul
  directorului \texttt{src}.
\end{remark}

Sursa de cod va fi un fișier cu extensia \texttt{spthy}, care reprezintă
o teorie pe care compilatorul o va testa. Fișierul de teorie conține
următoarele elemente:

\textbf{Date de intrare: Teoria ecuațională:} Specificăm mesajele pentru
protocol sub forma:

\begin{verbatim}
builtins: diffie-hellman
functions: mac/2, g/0, shk/0 [private]
\end{verbatim}

Astfel, elementele specifice protocolului DH (care este inclus în biblioteca
standard, lucru menționat prin sintaxa \texttt{builtins}) sînt încărcate.
În particular, avem definit operatorul \texttt{\^}, folosit pentru exponențiere.

Se mai definește o funcție de 2 variabile \texttt{mac}, care modelează un
cod de autentificare a mesajelor (eng.\ \emph{Message Authentication Code})
și două constante, \texttt{g} pentru a modela generatorul grupului în care
are loc problema, respectiv \texttt{shk} pentru o cheie secretă partajată
(eng.\ \emph{shared secret key}). Acestea sînt declarate \texttt{private},
deci nu pot fi obținute direct de către adversar.

De asemenea, cum Tamarin este implementat în Haskell, avem suport inclus
pentru func\-ți\-i\-le specifice perechilor, \texttt{<\_,\_>, fst, snd}.

\textbf{Date de intrare: Protocolul:} Pentru modelarea protocolului vom
utiliza 3 reguli de rescriere într-un sistem bazat pe un multiset.
Fiecare regulă este un triplet, alcătuit din șiruri de fapte în membrul
stîng, etichete și membrul drept. Faptele sînt de forma $ F(t_1, \dots, t_k) $,
pentru un simbol specific $ F $ și termenii $ t_i $. Regulile protocolului
folosesc simboluri unare de fapte \texttt{Fr} și \texttt{In} pentru a
obține constante noi (eng.\ \emph{fresh names}) și mesajele de pe
rețea.

Un mesaj este trimis pe rețea folosind un simbol unar de fapte \texttt{Out}
în membrul drept.

Prima regulă creează un fir de execuție pentru protocol \texttt{tid}
care alege un exponent nou \texttt{x} și trimite participanților \texttt{g\^{}x},
concatenat cu \texttt{mac} aplicat acestei valori:

\begin{verbatim}
rule Step1: [ Fr(tid:fresh), Fr(x:fresh) ] -[ ]->
            [ Out(<g^(x:fresh), mac(shk, <g^(x:fresh), A:pub, B:pub>)>)
            , Step1(tid:fresh, A:pub, B:pub, x:fresh) ]
\end{verbatim}

În aceste linii de sintaxă se specifică explicit ca variabilele utilizate
să fie instanțiate doar cu sorturile \texttt{fresh}, respectiv \texttt{pub},
acolo unde este cazul.

Spunem că o instanță a regulii \texttt{Step1} \emph{consumă} două fapte
\texttt{Fr} pentru a obține constantele noi \texttt{tid} și \texttt{x}
și \emph{generează} un fapt \texttt{Out} care conține mesajul trimis
și un fapt \texttt{Step1} care arată că firul de execuție a finalizat
primul pas cu parametrii dați.

Remarcăm că regula nu are etichete, deci ea nu va afecta urma sistemului
de rescriere.

A doua regulă este pasul al doilea din protocol:

\begin{verbatim}
rule Step2: [ Step1(tid, A, B, x:fresh), In(<Y, mac(Y, B, A>)>) ]
              -[ Accept(tid, Y^(x:fresh)) ]-> []
\end{verbatim}

Aici se folosește faptul din pasul 1 și se verifică MAC-ul.
Eticheta din final, \texttt{Accept(tid, Y\^{}(x:fresh))}, arată
că firul de execuție \texttt{tid} a acceptat cheia de sesiune
\texttt{Y\^{}(x:fresh)}.

Regula a treia arată cheia secretă adversarului:
\begin{verbatim}
rule RevealKey: [] -[ Reveal() ]-> [ Out(shk) ]
\end{verbatim}

Eticheta \texttt{Reveal()} asigură că se va vedea pe urma sistemului
dacă și cînd a avut loc dezvăluirea.

\textbf{Date de intrare: Proprietăți:} În continuare, trebuie
specificate proprietățile pe care să le satisfacă sistemul, sub
forma unor leme. Acestea vor fi verificate și admise sau respinse
de către Tamarin. Un exemplu este:

\begin{verbatim}
lemma Accept_Secret:
    ∀ i j tid key. Accept(tid,key)@i & K(key)@j =>
        ∃ n. Reveal()@n & n < i
\end{verbatim}

Această lemă cuantifică peste momentele de timp \texttt{i, j, n} și mesajele
\texttt{tid, key}. Predicatele folosite sînt de forma \texttt{F \@ i}, care
arată că faptul \texttt{F} are loc la momentul \texttt{i}. În total, lema
arată că dacă un fir de execuție \texttt{tid} a acceptat o cheie \texttt{key}
la momentul \texttt{i}, iar cheia este cunoscută și adversarului, atunci
trebuie să existe un moment de timp \texttt{n} anterior lui \texttt{i} cînd
a avut loc dezvăluirea (\texttt{Reveal()}).

\textbf{Date de ieșire} În fine, dacă rulăm Tamarin pe informațiile
de mai sus, salvate în fișierul-sursă \texttt{example.spthy}, invocat în
linia de comandă cu:
\begin{verbatim}
$ tamarin-prover example.spthy
\end{verbatim}
rezultă următorul mesaj:

\begin{verbatim}
analyzed example.spthy: Accept_Secret (all-traces) verified (9 steps)
\end{verbatim}

Acest lucru arată că Tamarin a analizat cu succes că toate urmele protocolului
satisfac proprietatea din lema \texttt{Accept\_Secret}.

%%%%%%%%%%%%%%%%%%%%%%%%%%%%%%%%%%%%%%%%%%%%%%%%%%%%%%%%%%%%%%%%%%%%%%
\section{\texttt{FirstExample.spthy}}

%%% Local Variables:
%%% mode: latex
%%% TeX-master: "../tamarin"
%%% End:
