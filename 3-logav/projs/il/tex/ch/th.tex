%! TEX root = ../tamarin.tex

\chapter{Fundamente teoretice}

\section{Teoria urmelor}

Vom prelua din prezentarea \cite{mazu}, autorul fiind fondatorul teoriei.
Carl Petri este citat ca avînd o influență foarte importantă în dezvoltarea
teoriei urmelor, prin introducerea (la vîrsta de 13 ani!) a
\emph{rețelelor Petri}, utilizate în primă fază pentru reacții chimice,
dar dezvoltate apoi pentru modelarea sistemelor computaționale concurente.

Teoria urmelor a venit ca o încercare de a înțelege sistemele concurente
folosind limbaje formale. În primă fază, la apariția teoriei în 1970, ea a
fost folosită pentru a muta accentul de pe concurența în execuție pe
non-determinism. Principalele dificultăți de depășit se legau de:
\begin{itemize}
    \item \emph{intercalarea proceselor} (eng.\ \emph{interleaving});
    \item \emph{rafinarea} comportamentului unui sistem, prin rafinarea
        proceselor concurente ce se execută;
    \item \emph{inevitabilitatea} unui anume comportament --- aspect
        care intra în contradicție cu non-determinismul;
    \item \emph{serializarea impusă} a tranzacțiilor.
\end{itemize}

%%%%%%%%%%%%%%%%%%%%%%%%%%%%%%%%%%%%%%%%%%%%%%%%%%%%%%%%%%%%%%%%%%%%%%
\subsection{Obiecte de lucru și notații}

Prezentăm acum cîteva elemente de bază din teoria urmelor. Pornim cu
o mulțime $ X $, o relație binară $ R $ definită pe $ X $ care va avea
anumite proprietăți (e.g.\ ordine totală). Vom nota mulțimea numerelor
naturale $ \{ 0, 1, 2, \dots, n, \dots, \} $ cu $ \omega $. Printr-un
\emph{alfabet} vom înțelege o colecție finită de simboluri, iar
alfabetele se vor nota cu majuscule grecești.

Dacă $ \Sigma $ este un alfabet, șirurile de simboluri cu număr finit
de elemente se vor nota cu $ \Sigma^\ast $, iar șirul de lungime
zero se va nota $ \epsilon $.

Mulțimea șirurilor peste un alfabet, împreună cu operația de concatenare
formează un monoid, care se va numi \emph{monoidul liber} generat de
(alfabetul) $ \Sigma $.

Dacă $ w $ este un șir, vom nota cu $ \dr{Alph}(w) $ mulțimea
simbolurilor din $ w $ (\emph{alfabetul} lui $ w $), iar prin
$ w(a) $ vom înțelege numărul de apariții ale simbolului $ a $ în
șirul $ w $.

Fie $ \Sigma $ un alfabet și $ w $ un șir, posibil definit peste
un alt alfabet. Se definește \emph{proiecția} lui $ w $ peste $ \Sigma $
funcția:
\[
    \pi_\Sigma(w) = %
    \begin{cases}
        \epsilon, & w = \epsilon \\
        \pi_\Sigma(u), & w = ua, a \notin \Sigma \\
        \pi_\Sigma(u)a, & w = ua, a \in \Sigma
    \end{cases}
\]
În esență, proiecția pe $ \Sigma $ șterge dintr-un șir toate simbolurile
care nu se găsesc în $ \Sigma $.

\emph{Reducerea la dreapta} a simbolului $ a $ din șirul $ w $ este
un șir notat cu $ w \div a $ și definit prin:
\begin{align*}
    \epsilon \div a &= \epsilon \\
    (wb) \div a &= %
    \begin{cases}
        w, & a = b \\
        (w \div a) b, &\text{altfel}
    \end{cases},
\end{align*}
definiție care are sens pentru orice șir $ w $ și simboluri $ a, b $.

Se poate verifica ușor că proiecția și reducerea comută, adică avem:
\[
    \pi_\Sigma(w)\div a = \pi_\Sigma(w \div a),
\]
pentru orice șir $ w $ și simbol $ a $.

Vom defini un \emph{limbaj} printr-un alfabet $ \Sigma $ și o
mulțime de șiruri definite peste $ \Sigma $, adică o submulțime a
$ \Sigma^\ast $. Formal, vom deosebi două limbaje care au aceeași
mulțime de șiruri, dar sînt definite peste alfabete diferite. În
particular, două limbaje care conțin doar simbolurile vide, dar
definite peste alfabete diferite, vor fi considerate diferite.

Limbajele se vor nota cu majuscule din alfabetul latin.

Dacă $ A, B $ sînt două limbaje definite peste același alfabet,
se poate defini \emph{concatenarea} lor $ AB $, care conține
$ \{ uv \mid u \in A, v \in B \} $, peste același alfabet.

\emph{Puterea unui limbaj} $ A $ se definește inductiv:
\[
    A^0 = \{ \epsilon \}, \quad A^{n+1} = A^n A, \forall n \in \NN,
\]
iar iterația $ A^\ast $ se definește prin:
\[
    A^\ast = \bigcup_{n \geq 0} A^n.
\]

În general, putem extinde funcția de proiecție de la șiruri
la limbaje. Fie $ A $ un limbaj, $ \Sigma $ un alfabet. Se definește
proiecția lui $ A $ pe $ \Sigma $ ca fiind limbajul $ \pi_\Sigma(A) $
dat de:
\[
    \pi_\Sigma(A) = \{ \pi_\Sigma(u) \mid u \in A \}.
\]

Dacă $ w $ este un șir, mulțimea:
\[
    \dr{Pref}(w) = \{ u \mid \exists v, uv = w \}
\]
se va numi \emph{mulțimea prefixelor} lui $ w $. Evident,
$ \epsilon, w \in \dr{Pref}(w) $, iar dacă $ A $ este un limbaj
peste $ \Sigma $, definim:
\[
    \dr{Pref}(A) = \bigcup_{w \in A} \dr{Pref}(w).
\]

În general, are loc $ A \seq \dr{Pref}(A) $, iar dacă și incluziunea
reciprocă este adevărată, limbajul se numește \emph{închis la prefixe}.
Rezultă că, pentru orice limbaj $ A $, $ \dr{Pref}(A) $ este un limbaj
închis la prefixe.

Se definește \emph{relația de prefix} ca fiind o relație binară
$ \sqsubseteq {} \in \Sigma^\ast \times \Sigma^\ast $ definită prin:
\[
    u \sqsubseteq w \Leftrightarrow u \in \dr{Pref}(w),
\]

Se poate arăta simplu că relația de prefix este o relație de ordine
pentru orice mulțime de șiruri.

%%%%%%%%%%%%%%%%%%%%%%%%%%%%%%%%%%%%%%%%%%%%%%%%%%%%%%%%%%%%%%%%%%%%%%
\subsection{Dependență și urme}

Vom numi o \emph{dependență} orice relație finită, reflexivă și simetrică,
adică o mulțime finită de perechi ordonate $ D $, astfel încît, dacă
$ (a, b) \in D $, atunci $ (b,a) \in D $ și $ (a, a) \in D $.

Fie acum $ D $ o dependență. Domeniul lui $ D $ se va nota $ \Sigma_D $
și se va numi \emph{alfabetul} lui $ D $. Dacă $ D $ este o dependență,
atunci relația $ I_D = (\Sigma_D \times \Sigma_D) - D $ se numește
\emph{independența} indusa de $ D $. Evident, această relație este
simetrică și nereflexivă. În particular, relația vidă, relația de
identitate și relația completă pe $ \Sigma $ (aceasta din urmă fiind
$ \Sigma \times \Sigma $) sînt dependențe. Prima are alfabetul vid,
a doua este cea mai mică dependență în $ \Sigma $, iar ultima este
cea mai mare dependență în $ \Sigma $.

De exemplu, fie relația:
\[
    D = \{ a, b \}^2 \cup \{ a,c \}^2.
\]
Aceasta este o dependență și avem:
\[
    \Sigma_D = \{ a, b, c \}, \quad I_D = \{ (b,c), (c,b) \}.
\]

Teoria urmelor va avea dependențele ca noțiuni primare. De asemenea,
se mai pot folosi și \emph{alfabete concurente} ca noțiuni primare,
alcătuite din orice pereche $ (\Sigma, D) $, unde $ \Sigma $ este un
alfabet, iar $ D $ este o dependență sau \emph{alfabete suport}
(eng.\ \emph{reliance alphabet}), alcătuite dintr-un triplet format
dintr-un alfabet $ \Sigma $, o dependență $ D $ și independența $ I $
indusă de $ D $.

Ajungem în fine la definiția principală.

\begin{definition}\label{def:trace}
    Fie $ D $ o dependență. Se definește \emph{echivalența de urmă}
    (eng.\ \emph{trace equivalence}) pentru $ D $ ca fiind cea mai
    mică congruență $ \equiv_D $ în monoidul $ \Sigma^\ast_D $ astfel
    încît, pentru orice $ a, b \in \Sigma $ să avem:
    \[
        (a, b) \in I_D \Rightarrow ab \equiv_D ba.
    \]
    Clasele de echivalență din $ \Sigma^\ast_D/\equiv_D $ se numesc
    \emph{urme} (eng.\ \emph{traces}) peste $ D $.

    Urma reprezentată de șirul $ w $ se va nota $ [w]_D $, iar prin
    $ \left[ \Sigma^\ast \right]_D $ vom nota mulțimea factor,
    respectiv $ \left[ \Sigma \right]_D $ va nota mulțimea claselor care
    include și clasa vidă ($\{ [a]_D \mid a \in \Sigma_D \}$).
\end{definition}

Reluînd exemplul de mai sus, pentru dependența:
\[
    D = \{ a, b \}^2 \cup \{ a, c \}^2,
\]
urma peste $ D $ dată de șirul $ abbca $ este:
\[
    [abbca]_D = \{ abbca, abcba, acbba \}.
\]

În general, din definiție, toate șirurile care diferă numai prin ordinea
a două simboluri consecutive determină o singură urmă.

Se definește \emph{monoidul urmelor} $ \mathbb{M}(D) $ ca avînd mulțimea
suport dată de mulțimea factor $ \Sigma^\ast_D / \equiv_D $, iar operația
definită în mod natural. Amintim că există aplicația de proiecție canonică
de la mulțimea $ \Sigma^\ast $ la mulțimea factor, care de fapt induce un
morfism natural de monoizi:
\[
    \varphi_D : \Sigma^\ast \to \mathbb{M}(D), \quad \varphi_D(w) = [w]_D.
\]

Presupunem în continuare că lucrăm cu o dependență fixată $ D $, astfel că
vom omite indicele în cele ce urmează. De asemenea, $ I $ va fi independența
indusă de $ D $ (fixată), $ \Sigma $ va fi domeniul lui $ D $, toate
simbolurilor vor fi din $ \Sigma_D $, toate șirurile vor fi peste $ \Sigma_D $,
iar toate urmele vor fi peste $ D $, dacă nu se specifică altfel.

Implicația $ u \equiv v \Rightarrow \dr{Alph}(u) = \dr{Alph}(v) $ este clară.
Rezultă că definiția $ \dr{Alph}([w]) = \dr{Alph}(w) $ este corectă.

Fie $ \sim : \Sigma^\ast \times \Sigma^\ast $ o relație binară definită
astfel încît $ u \sim v $ dacă și numai dacă există $ x, y \in \Sigma^\ast $
și $ (a,b) \in I $ cu $ u = xaby $ și $ v = xbay $. Cu această definiție,
se poate vedea că $ \equiv $ este închiderea simetrică, reflexivă și
tranzitivă a $ \sim $. Cu alte cuvinte, obținem:
\[
    u \equiv v \Leftrightarrow \exists (w_0, \dots, w_n), w_0 = u, w_n = v,
\]
iar pentru toți $ 0 < i \leq n $ are loc $ w_{i-1} \sim w_i $.

Fie $ [u], [v] $ două urme peste aceeași dependență. Urma $ [u] $ se numește
\emph{prefix} al urmei $ [v] $ (iar $ [v] $ se numește un \emph{dominant}
al lui $ [u] $) dacă există o urmă $ [x] $ astfel încît $ [ux] = [u] $.

E important de știut că:
\begin{proposition}\label{prop:lcm}
    Fie $ [w] $ o urmă, iar $ [u], [v] $ prefixe ale lui $ [w] $. Atunci
    există cel mai mare prefix comun și cel mai mic dominant comun al lui
    $ [u] $, respectiv $ [v] $.
\end{proposition}

\begin{proof}
    \cite[Prop.\ 1.3.5]{mazu}.
\end{proof}

%%%%%%%%%%%%%%%%%%%%%%%%%%%%%%%%%%%%%%%%%%%%%%%%%%%%%%%%%%%%%%%%%%%%%%
\section{Sisteme de rescriere}

Preluăm o prezentare sumară a subiectului din \cite{10.1007/10721959_21},
indicînd pentru mult mai multe detalii Capitolul 3 din \cite{handbook}.

Fie $ \kal{F} $ o mulțime finită de simboluri, care conține și o funcție
de aritate definită pe $ \kal{F} $. Fie $ \kal{X} $ o mulțime numărabilă
de variabile, $ \kal{T}(\kal{F}, \kal{X}) $ o mulțime de termeni, iar
$ \kal{T}(\kal{F}) $ mulțimea termenilor fără variabile. Mulțimea
pozițiilor dintr-un termen $ t $, notată $ \dr{Pos}(t) $, este ordonată
lexicografic. Șirul vid $ \epsilon $ ține poziția cea mai de sus. Dacă
$ p \in \dr{Pos}(t) $, atunci $ t\mid_p $ este subtermenul lui $ t $
de la poziția $ p $, iar $ t[s]_p $ este termenul obținut prin
înlocuirea subtermenului $ t\mid_P $ la poziția $ p $ cu
termenul $ s $. Mulțimea variabilelor din termeul $ t $ se va nota cu
$ \dr{Var}(t) $.

O \emph{substituție} este o aplicație 
\[
    \sigma : \kal{X} \to \kal{T}(\kal{F}, \kal{X}),
\]
care poate fi extinsă în mod unic la un endomorfism al mulțimii
termenilor (cu concatenarea). Domeniul său se definește prin:
\[
    \dr{Dom}(\sigma) = \{ x \in \kal{X} \mid x\sigma \neq x \}.
\]

Un \emph{sistem de rescriere a termenilor} $ \kal{R} $ este o mulțime
de reguli de rescriere de forma $ l \to r $, unde
$ l, r \in \kal{T}(\kal{F}, \kal{X}), l \notin \kal{X} $, iar
$ \dr{Var}(l) \supseteq \dr{Var}(r) $.

Relația $ \to_\kal{R} $ indusă de un sistem de rescriere $ \kal{R} $
pe o mulțime de termeni $ \kal{T}(\kal{F}, \kal{X}) $ se definește
astfel:
\[
    s \to_\kal{R} t \Leftrightarrow \left( \exists l \to r \in \kal{R}, %
        \exists p \in \dr{Pos}(s), \exists \sigma \text{ subst. a.î. } %
    (l\sigma = s\mid_p \land t = s [r\sigma]_p) \right).
\]

Închiderea tranzitivă a relației se va nota $ \to^\ast_\kal{R} $.

%%%%%%%%%%%%%%%%%%%%%%%%%%%%%%%%%%%%%%%%%%%%%%%%%%%%%%%%%%%%%%%%%%%%%%

\section{Analiza automată a protocoalelor}

Preluăm acum din \cite[Cap.\ 3]{schmidt} pentru un exemplu general de
analiză formală a protocoalelor de criptare care folosesc exponențierea
Diffie-Hellman.
