%! TEX root = ../tamarin.tex

\chapter{Introducere}

Tamarin este un produs software care este folosit în modelarea simbolică
și analiza protocoalelor de securitate. Modul de operare este dat de
\begin{itemize}
    \item introducerea unui model de protocol de securitate, ca date de intrare,
        prin specificarea acțiunilor pe care agenții participanți la protocol
        le iau;
    \item o specificație a adversarului;
    \item o specificație a proprietăților dorite ale protocolului.
\end{itemize}

Tamarin oferă o modalitate de a raționa asupra protocoalelor de securitate
prin specificații, atît ale protocolului, cît și ale adversarului, într-un
limbaj expresiv bazat pe reguli de rescriere asupra unor multi-seturi
(i.e.\ într-o logică multi-sortată). Prin aceste reguli, se definește un
sistem de tranziții eticetat, ale cărui stări sînt reprezentate de
cunoștințele adversarului, de mesajele existente pe rețea, informații
despre valorile proaspăt generate (variabilele \emph{fresh}) și
generarea de noi mesaje.

Totodată, putem specifica și ecuațional unii operatori criptografici,
precum logaritmul (sau exponențierea) discret(ă) Diffie-Hellman.

Se oferă două moduri de a construi demonstrații:
\begin{enumerate}[(1)]
    \item Un mod \emph{complet automat}, în care se combină reguli
        de deducție și raționamente ecuaționale, împreună cu argumente
        euristice pentru a ghida demonstrația. Dacă operația se finalizează,
        se returnează fie o demonstrație a corectitudinii, fie un
        contraexemplu, care reprezintă un atac ce încalcă proprietățile
        dorite.
    \item Cum problemele de corectitudine sînt, în general, indecidabile,
        se oferă și un mod \emph{asistat}, interactiv, în care utilizatorul
        ghidează demonstrația pas cu pas, inspectînd graful stărilor și
        luînd decizii combinate cu abordarea automată.
\end{enumerate}

Descrierea formală a metodelor de funcționare a Tamarin este dată
în \cite{schmidt} și \cite{meier}, din care vom prelua cîteva elemente
de bază în capitolul următor. Ideea fundamentală este următoarea:
fie $ E $ o teorie cu egalitate care definește operatori criptografici,
un sistem de rescriere $ R $ asupra unui multi-set, care definește un
protocol și $ \phi $ o formulă care definește o proprietate trasabilă.
Tamarin poate să verifice validitatea sau satisfiabilitatea lui $ \phi $
pentru urmele lui $ R $ modulo $ E $.
