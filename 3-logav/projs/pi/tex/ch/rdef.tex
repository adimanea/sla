% ! TEX root = ../static.tex

\chapter{Definițiile accesibile}

\index{definiții!accesibile}
\index{definiții!distruse}
\index{definiții!generate}
Vom fi interesați de a afla în ce punct al programului \emph{este posibil}
ca variabila $ x $ să fi fost definită atunci cînd parcurgem fiecare
punct $ p $ al programului. O astfel de informație aparent simplă poate fi
foarte utilă, de exemplu, pentru a determina dacă o anume variabilă este,
de fapt, constantă și să o înlocuim cu valoarea ei sau dacă o anume variabilă
este folosită fără a fi definită. De exemplu, putem introduce o
definiție-fantomă (eng.\ \emph{dummy}) a unei variabile în punctul de la
începutul programului și să studiem dacă această definiție se propagă pînă la
un punct în care variabila este utilizată. Dacă da, atunci ea este utilizată
fără a fi definită, situație care trebuie semnalată.

Ca terminologie, vom spune că o definiție $ d $ \emph{se propagă} pînă la
un punct $ p $ dacă există un drum de execuție de la punctul ce urmează imediat
lui $ d $ pînă la $ p $, astfel încît $ d $ să nu fie \emph{distrusă} pe acest
drum. Spunem că o definiție $ d $ a unei variabile $ x $ este \emph{distrusă}
dacă există o altă definiție $ d' $ a lui $ x $ în drumul pe care îl studiem.

\begin{remark}\label{rk:def-loop}
  În analiza drumurilor de propagare a definițiilor, vom ține cont de bucle,
  astfel că în cadrul unei bucle considerăm că definiția nu este distrusă,
  iar punctul ce succede definiția va fi cel de ieșire din buclă.
\end{remark}

Mai adăugăm că variabilele pot apărea ca parametri în proceduri, tablouri
sau referințe indirecte, care sînt, în general, \emph{alias-uri}, astfel că
nu putem spune precis dacă o variabilă este sau nu afectată în mod direct
de o asemenea instrucțiune. Pentru consecvență și pentru a face o analiză
conservatoare, dacă nu este clar dacă o instrucțiune atribuie sau nu o valoare
variabilei $ x $, vom presupune că \emph{o poate face}. Dar pentru simplitate,
vom elimina din studiu cazurile cu alias-uri și vom folosi doar variabile
locale, scalare.

Considerăm exemplul din figura \ref{fig:rdef-ex1}.

\begin{figure}[!htbp]
  \centering
  \small
\begin{BVerbatim}
          +---------+
          | INTRARE |
          +---------+
               |
               V
        +---------------+ B1        
        | d1: i = m - 1 |           gen(B1)  = { d1, d2, d3 }
        | d2: j = n     |           kill(B1) = { d4, d5, d6, d7 }
        | d3: a = u1    |
        +---------------+
               |
               V
        +---------------+ B2        gen(B2)  = { d4, d5 }
        | d4: i = i + 1 |           kill(B2) = { d1, d2, d7 }
        | d5: j = j - 1 |
        +---------------+
          /    |    ^
         /     |    |
B3      /      |    |
+------------+ |    |               gen(B3)  = { d6 }
| d6: a = u2 | |    |               kill(B3) = { d3 }
+------------+ |    |
  |            |    |
  |            |    |
  |            V    |
  |     +------------+ B4           gen(B4)  = { d7 }
  |---> | d7: i = u3 |              kill(B4) = { d1, d4 }
        +------------+
               |
               |
               V
          +--------+
          | IEȘIRE |
          +--------+
\end{BVerbatim}
  \caption{Exemplu pentru definiții accesibile}
  \label{fig:rdef-ex1}
\end{figure}

Să studiem definițiile din blocul \texttt{B2}. Toate definițiile din blocul
\texttt{B1} ajung la începutul blocului \texttt{B2}. Definiția
\texttt{d5:\!\!\!\! j = j - 1} din blocul \texttt{B2} ajunge și la începutul blocului
\texttt{B2}, deoarece nu mai există altă definiție a lui \texttt{j} care să
se găsească în bucla ce duce înapoi în \texttt{B2}. Această definiție, însă,
distruge definiția \texttt{d2:\!\!\!\! j = n}, ceea ce o face să nu ajungă la
\texttt{B3} sau \texttt{B4}.

Pe de altă parte, instrucțiunea \texttt{d4:\!\!\!\! i = i + 1} din blocul \texttt{B2}
nu ajunge la începutul lui \texttt{B2}, deoarece variabila \texttt{i} este
mereu redefinită de \texttt{d7:\!\!\!\! i = u3}.

În fine, definiția \texttt{d6:\!\!\!\! a = u2} ajunge și ea la începutul blocului
\texttt{B2}.

\begin{remark}\label{rk:safe}
  Prin definirea definițiilor accesibile așa cum am făcut-o, pot exista
  inadvertențe, însă toate sînt în direcția \emph{conservativă}, adică mai curînd
  se consideră aspecte redundante, decît să se omită unele. De exemplu, în codul:
  \begin{center}
    \begin{BVerbatim}
      if (a == b) { instrucțiune1 } else if (a == b) { instrucțiune2 }
    \end{BVerbatim}
  \end{center}
  \texttt{instrucțiune2} nu se va atinge niciodată, pentru nicio valoare a lui \texttt{a}.

  Însă în general, problema dacă toate drumurile dintr-un graf de flux al datelor
  sînt parcurse este indecidabilă, astfel că alegem asumpția că ele sînt efectiv
  parcurse, chiar dacă acest lucru poate conduce la situații precum cea de mai
  sus. Eroarea în acest caz este de partea pozitivă, astfel că nu pierdem informație,
  ci cel mult avem o cantitate redundantă.
\end{remark}

%%%%%%%%%%%%%%%%%%%%%%%%%%%%%%%%%%%%%%%%%%%%%%%%%%%%%%%%%%%%%%%%%%%%%%
\section{Ecuații de transfer}

Arătăm acum modul în care se pot formula constrîngerile pentru problema
definițiilor accesibile. Începem prin studiul unei singure instrucțiuni:
\[
  \texttt{d: u = v + w}.
\]

Instrucțiunea generează o definiție \texttt{d} a unei variabile \texttt{u}
și distruge toate celelalte definiții din program care se adresează
variabilei \texttt{u}, dar nu afectează nicio altă definiție. Așadar,
funcția de transfer pentru definiția \texttt{d} poate fi scrisă:
\begin{equation}
  \label{eq:gen-kill}
  f_d(x) = \texttt{gen(d)} \cup (x - \texttt{kill(d)}),
\end{equation}
unde \texttt{gen(d) = \{d\}}, este mulțimea definițiilor
generate de instrucțiune, iar $ \texttt{kill(d)} $ este mulțimea
celorlalte definiții ale lui \texttt{u} din program.

În general funcția de transfer a unui bloc poate fi aflată prin
compunerea tuturor funcțiilor de transfer ale instrucțiunilor din interiorul
blocului. Se poate vedea ușor că prin compunerea a două funcții precum
cea de mai sus (ecuația~\eqref{eq:gen-kill}, pe care o vom numi ecuație
\texttt{gen-kill}) se obține tot o funcție cu aceeași formă.
\index{ecuație!\texttt{gen-kill}}

În general, avem:
\begin{equation}
  \label{eq:fb}
  f_b(x) = \texttt{gen(B)} \cup (x - \texttt{kill(B)},
\end{equation}
unde am notat:
\[
  \texttt{kill(B)} = \texttt{kill(1)} \cup \texttt{kill(2)} \cup \dots \cup \texttt{kill(n)},
\]
iar pentru generare, avem:
\begin{align*}
  \texttt{gen(B)} &= \texttt{gen(n)} \cup (\texttt{gen(n-1)} - \texttt{kill(n)}) \cup %
                    (\texttt{gen(n-2)} - \texttt{kill(n-1)} - \texttt{kill(n)} \cup \\
                  & \dots \cup (\texttt{gen(1)} - \texttt{kill(2)} - \texttt{kill(3)} - %
                    \dots - \texttt{kill(n)})
\end{align*}

\index{definiții!expuse mai jos}
Mulțimea \texttt{gen} care rezultă dintr-un bloc se va numi mulțimea definițiilor
\emph{expuse mai jos}, deoarece această mulțime va conține toate definițiile
din bloc care sînt vizibile și imediat după bloc. Rezultă că o definiție dintr-un
bloc este expusă mai jos dacă și numai dacă nu este distrusă de o definiție a
aceleiași variabile în interiorul aceluiași bloc.

De cealaltă parte, mulțimea \texttt{kill} a unui bloc este pur și simplu reuniunea
tuturor definițiilor distruse de instrucțiuni individuale.

Remarcăm că, în general, o definiție poate apărea atît în mulțimea \texttt{gen},
cît și în \texttt{kill} ale unui bloc. Dacă acesta este cazul, atunci
considerăm că ea este generată, deoarece presupunem că mulțimea \texttt{kill}
acționează după mulțimea \texttt{gen}.

\begin{example}\label{exm:gen-kill}
  Considerăm blocul simplu \texttt{B}:
  \begin{align*}
    \texttt{d1:} \quad & \texttt{a = 3} \\
    \texttt{d2:} \quad & \texttt{a = 4}
  \end{align*}
  În acest caz, \texttt{gen(B) = \{d2\}}, deoarece \texttt{d1} nu este expusă
  mai jos. De asemenea, tot pentru acest bloc \texttt{kill(B) = \{d1, d2\}}, deoarece \texttt{d1}
  distruge \texttt{d2} și reciproc. Totuși, din cauza presupunerii de precedență,
  rezultatul funcției de transfer pentru acest bloc va include mereu \texttt{d2}.
\end{example}

%%%%%%%%%%%%%%%%%%%%%%%%%%%%%%%%%%%%%%%%%%%%%%%%%%%%%%%%%%%%%%%%%%%%%%

\section{Ecuațiile fluxului de control}

Considerăm acum constrîngerile care rezultă din fluxul de control dintre
blocurile de bază. Cum o definiție atinge un punct al programului dacă și numai
dacă există cel puțin un drum prin care definiția să se propage, avem:
\[
  \texttt{OUT[P]} \seq \texttt{IN[B]},
\]
unde \texttt{P} este un punct astfel încît să existe un drum de flux al
controlului pînă la blocul \texttt{B}. Însă, cum o definiție nu poate ajunge la
un punct decît dacă o face pe un drum, rezultă că \texttt{IN[B]} nu
depășește reuniunea definițiilor accesibile din toate blocurile ce-l
precedă pe \texttt{B}. Deci putem presupune că, în general, avem:
\[
  \texttt{IN[B]} = \bigcup_{\texttt{P} \downarrow \texttt{B}} \texttt{OUT[P]},
\]
unde am notat cu $ \texttt{P} \downarrow \texttt{B} $ faptul că punctul \texttt{P}
precede blocul \texttt{B}.

Putem privi această reuniune ca pe un operator \emph{meet} pentru
definițiile accesibile, ca în cazul laticelor.
\index{definiții!accesibile!meet}

%%%%%%%%%%%%%%%%%%%%%%%%%%%%%%%%%%%%%%%%%%%%%%%%%%%%%%%%%%%%%%%%%%%%%%
\section{Algoritm iterativ pentru definiții accesibile}
\index{definiții!accesibile!algoritm}

%%% Local Variables:
%%% mode: latex
%%% TeX-master: "../static"
%%% End:
