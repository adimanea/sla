% ! TEX root = ../static.tex

\chapter{Definițiile accesibile}

Vom fi interesați de a afla în ce punct al programului \emph{este posibil}
ca variabila $ x $ să fi fost definită atunci cînd parcurgem fiecare
punct $ p $ al programului. O astfel de informație aparent simplă poate fi
foarte utilă, de exemplu, pentru a determina dacă o anume variabilă este,
de fapt, constantă și să o înlocuim cu valoarea ei sau dacă o anume variabilă
este folosită fără a fi definită. De exemplu, putem introduce o
definiție-fantomă (eng.\ \emph{dummy}) a unei variabile în punctul de la
începutul programului și să studiem dacă această definiție se propagă pînă la
un punct în care variabila este utilizată. Dacă da, atunci ea este utilizată
fără a fi definită, situație care trebuie semnalată.

Ca terminologie, vom spune că o definiție $ d $ \emph{se propagă} pînă la
un punct $ p $ dacă există un drum de execuție de la punctul ce urmează imediat
lui $ d $ pînă la $ p $, astfel încît $ d $ să nu fie \emph{distrusă} pe acest
drum. Spunem că o definiție $ d $ a unei variabile $ x $ este \emph{distrusă}
dacă există o altă definiție $ d' $ a lui $ x $ în drumul pe care îl studiem.

\begin{remark}\label{rk:def-loop}
  În analiza drumurilor de propagare a definițiilor, vom ține cont de bucle,
  astfel că în cadrul unei bucle considerăm că definiția nu este distrusă,
  iar punctul ce succede definiția va fi cel de ieșire din buclă.
\end{remark}

Mai adăugăm că variabilele pot apărea ca parametri în proceduri, tablouri
sau referințe indirecte, care sînt, în general, \emph{alias-uri}, astfel că
nu putem spune precis dacă o variabilă este sau nu afectată în mod direct
de o asemenea instrucțiune. Pentru consecvență și pentru a face o analiză
conservatoare, dacă nu este clar dacă o instrucțiune atribuie sau nu o valoare
variabilei $ x $, vom presupune că \emph{o poate face}. Dar pentru simplitate,
vom elimina din studiu cazurile cu alias-uri și vom folosi doar variabile
locale, scalare.

%%% Local Variables:
%%% mode: latex
%%% TeX-master: "../static"
%%% End:
